% Ein Konstruktor ist eine besondere Methode innerhalb einer Klasse, welche keinen expliziten Rückgabetyp besitzt und den gleichen Namen trägt wie die Klasse. In dieser Methode werden alle zur Initialisierung der Klasse nötigen Aktionen vollführt, zum Beispiel Objektvariablen belegen. Jede Klasse besitzt einen Default-Konstruktor, welcher keine Parameter annimmt

\subsection{Klassen, Objekte und Methoden}
	\todo{Schreiben}
	
	\subsubsection{Statische Methoden und Attribute}
		\todo{Schreiben}
	% end
	
	\subsubsection{Sichtbarkeit}
		\todo{Schreiben}
	% end
	
	\subsubsection{Abgrenzung: Objektvariable \(\leftrightarrow\) Objektkonstante \(\leftrightarrow\) Klassenvariable \(\leftrightarrow\) Klassenkonstante}
		\todo{Schreiben}
	% end
	
	\subsubsection{Abgrenzung: Objektmethode \(\leftrightarrow\) Klassenmethode}
		\todo{Schreiben}
	% end
	
	\subsubsection{Konstruktoren}
		\todo{Schreiben}
		
		\paragraph{Überladen von Konstruktoren}
			\todo{Schreiben}
		% end
		
		\paragraph{\texttt{this}}
			\todo{Schreiben}
		% end
		
		\paragraph{\texttt{super}}
			\todo{Schreiben}
		% end
	% end
	
	\subsubsection{Initializer-Block}
		\todo{Schreiben}
	% end
	
	\subsubsection{Static-Initializer-Block}
		\todo{Schreiben}
	% end
% end

\subsection{Referenzen}
	\todo{Schreiben}
	
	\subsubsection{Vergleich zu primitiven Daten}
		\todo{Schreiben}
	% end
	
	\subsubsection{Literal \texttt{null}}
		\todo{Schreiben}
	% end
	
	\subsubsection{Sonderfall \texttt{String}}
		\todo{Schreiben}
	% end
	
	\subsubsection{Zuweisen vs. Kopieren}
		\todo{Schreiben}
	% end
	
	\subsubsection{Test auf Gleichheit und Identität}
		\todo{Schreiben}
	% end
	
	\subsubsection{Downcasts}
		\todo{Schreiben}
	% end
% end

\subsection{Vererbung}
	\todo{Schreiben}
	
	\subsubsection{Methoden}
		\todo{Schreiben}
	% end
	
	\subsubsection{Variation der Sichtbarkeit}
		\todo{Schreiben}
	% end
	
	\subsubsection{Variation von Rückgabetyp und Exceptions}
		\todo{Schreiben}
	% end
	
	\subsubsection{Attribute}
		\todo{Schreiben}
	% end
	
	\subsubsection{Finale Klassen}
		\todo{Schreiben}
	% end
% end

\subsection{Abstrakte Klassen}
	\todo{Schreiben}
% end

\subsection{Interfaces}
	\todo{Schreiben}
	
	\subsubsection{Default-Methoden}
		\todo{Schreiben}
	% end
	
	\subsubsection{Funktionale Interfaces}
		\todo{Schreiben}
		
		\paragraph{Interfaces in \texttt{java.util.function}}
			\todo{Schreiben}
		% end
	% end
% end

\subsection{Polymorphie und späte Bindung}
	\todo{Schreiben}
	
	% statischer/dynamischer Typ
% end

\subsection{Verschachtelte Klassen}
	\todo{Schreiben}
	
	\subsubsection{Statische verschachtelte Klassen}
		\todo{Schreiben}
	% end
	
	\subsubsection{Innere Klassen}
		\todo{Schreiben}
	% end
	
	\subsubsection{Anonyme Innere Klassen}
		\todo{Schreiben}
	% end
% end

\subsection{Lambda-Ausdrücke}
	\todo{Schreiben}
	
	\subsubsection{Methoden-Referenzen}
		\todo{Schreiben}
	% end
% end

\subsection{Enumerations}
	\todo{Schreiben}
	
	\subsubsection{Klasse \texttt{java.lang.Enum}}
		\todo{Schreiben}
	% end
	
	\subsubsection{Vererbung}
		\todo{Schreiben}
	% end
% end

\subsection{Metadaten}
	\todo{Schreiben}
	
	% Zur Klasse
	% Zu den Attributen
	% Zu den Methoden
	% Methodentabelle
% end

\subsection{Speicherverwaltung}
	\todo{Schreiben}
% end
