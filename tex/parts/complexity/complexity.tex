Wir haben in den vorhergehenden Kapiteln viel über Programmiersprachen und deren Eigenarten gehört und widmen uns nun einem etwas theoretischeren Thema als die Programmierung an sich: Der Komplexitätstheorie, genauer der Landau-Symbolik.

Betrachten wir hier zum Einstieg folgende Algorithmen, die zum einen die Fakultät einer Ganzzahl berechnen und zum anderen die Fakultät-Fakultät einer Ganzzahl berechnen (d.h. das Produkt der Fakultäten)
\begin{figure}[H]
	\centering
	\caption{Komplexität: Berechnung der Fakultät}
\end{figure}

\todo{Stopped here.}