\implements{Kontrollstrukturen}{kontrollstrukturen}{\racket}

In diesem Abschnitt schauen wir uns an, wie Kontrollstrukturen in \racketText umgesetzt werden. \racketText kennt dabei die Kontrollstrukturen \textit{If} und \textit{Cond} (von \enquote{Conditional}), wobei \textit{Cond} nur eine Vereinfachung von vielen geschalteten Ifs darstellt.

In \racketText gibt es keine Schleifen, da \racketText eine funktionale Programmiersprache ist! Alle Wiederholungen werden über Rekursion \footnote{Siehe \refImpl{rekursion}{\racket}} gelöst.

\subsection{If}
	Das \textit{If} ist die einfachste Form der Verzweigung und hat folgende Form:
	\begin{figure}[H]
		\centering
		\lstinline[language = Racket]|(if <Abfrage> <Wahr-Fall> <Falsch-Fall>)|
	\end{figure}
	Wird der Ausdruck \texttt{<Abfrage>} zu Wahr ausgewertet, so wird das Ergebnis von \texttt{<Wahr-Fall>} zurück gegeben. Ansonsten wir das Ergebnis von \texttt{<Falsch-Fall>} zurück gegeben.
	
	\warning{Bei einem If in \racketText müssen \textit{immer} sowohl Wahr- als auch Falsch-Fall angegeben werden!}
	
	\paragraph{Beispiele}
		\begin{itemize}
			\item \lstinline[language = Racket]|(if (= (modulo x 2) 0) 'even 'odd)| \\
				  Wertet zu \texttt{'even} aus, wenn \texttt{x} gerade ist und sonst zu \texttt{'odd}.
			\item \lstinline[language = Racket]|(if (> x y) x y)| \\
				  Wertet zu dem Maximum von \texttt{x} und \texttt{y} aus (also \( \max \{ x, y \} \)).
		\end{itemize}
	% end
% end

\subsection{Cond}
	Ein \textit{Cond} vereinfacht verschachtelte If-Abfragen immens, wie wir gleich sehen werden. Schauen wir uns dazu folgendes verschachteltes If an:
	\begin{figure}[H]
		\centering
		\begin{lstlisting}[language = Racket]
(if (< x y)
	-1
	(if (> x y)
		1
		0
	)
)
\end{lstlisting}
	\end{figure}

	Und nun noch die allgemeine Syntax von Cond:
	\begin{figure}[H]
		\centering
		\lstinline[language = Racket]|(cond (<Test1> <Ausdruck1>) *@\dots@* (<TestN> <AusdruckN>)) [(else <Ansonsten>)]|
	\end{figure}
	Wobei der gesamte Ausdruck zu \texttt{<AusdruckK>} auswertet genau dann wenn \texttt{<TestK>} Wahr wird und zu \texttt{<Ansonsten>} auswertet, wenn alle Tests negativ ausfallen.
	
	Dann können wir das obige If zu folgendem Code vereinfachen:
	\begin{figure}[H]
		\centering
		\begin{lstlisting}[language = Racket]
(cond
	((< x y) -1)
	((> x y)  1)
	((= x y)  0)
)
\end{lstlisting}
	\end{figure}

	Damit haben wir nun das nötige Handwerkszeug, um komplexe Programme zu schreiben.
% end