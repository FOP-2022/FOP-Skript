Wir werden uns nun als erstes mit \racket\, auseinandersetzen. \racket\, steht für \enquote{How to Design Programs - Teaching Languages} und stellt eine Sprache bereit, die auf Racket und damit auf LISP aufbaut und für die einzelne Sprachfeatures \enquote{deaktiviert} werden können, um die Programmierung an Anfänger heranzuführen. Die Unterschiede zwischen den einzelnen sogenannten \enquote{Sprachlevels} werden wir uns hier nicht genauer angucken, sondern annehmen, dass wir stetig auf dem höchsten Sprachlevel arbeiten.

Im folgenden schauen wir uns \racket\, an und wie die in \ref{c:abstrakte_konzepte} Konzepte in der Sprache implementiert werden.

\section{Lexikalische Bestandteile}
	\subsection{Datentypen}
	\implements{Datentypen}{datentypen}{\racket}
	
	Im folgenden schauen wir uns an, was es in \racket\, für Datentypen gibt:
	\begin{itemize}
		\item Zahlen
			\begin{itemize}
				\item Ganzzahlen
				\item Fließkommazahlen
				\item Brüche
				\item Irrationale (ungenaue) Zahlen
				\item Komplexe Zahlen
			\end{itemize}
		\item Wahrheitswerte
		\item Symbole
		\item Strings
		\item Structs
		\item Listen
	\end{itemize}

	Dabei ist \racket\, aber nicht statisch typisiert, das heißt die Datentypen nicht mit angegeben werden, sondern es ist ausreichend, wenn zur Laufzeit der korrekte Datentyp in einer Variable gespeichert ist (es ist zum Beispiel nicht möglich, Strings zu addieren). Ist nicht der korrekte Datentyp gespeichert, so tritt ein Fehler auf.
	
	\paragraph{Symbole}
		Symbole sind einfache Zeichenketten, die ausschließlich verglichen werden können und weniger Funktionalität als Strings bieten.
		
		Allerdings ist die Verwendung von Symbolen sehr effizient und zu empfehlen, wenn wir mit der produzierten Zeichenkette nichts weiter tun wollen als sie zu vergleichen (dies tritt erstaunlich oft auf, öfter als man im Allgemeinen denkt).
	% end
	
	\paragraph{Listen}
		Listen ist einer der wichtigsten Datentypen in \racket. Wir werden uns diesen wichtigen Datentyp im Abschnitt \ref{sec:racket_lists} genauer anschauen.
	% end
	
	\paragraph{Sondertyp \textit{Struct}}
		Ein \textit{Struct} (eine Struktur) ist von dem Entwickler definierbar und ermöglicht es, komplexe Datentypen zu speichern. Wir werden uns diesen besonderen Datentyp im Abschnitt \ref{sec:structs} anschauen.
	% end
% end

\subsection{Literale}
	\implements{Literalen}{literale}{\racket}
	
	Wie wir Literale im Code ablegen, hängt von dem Datentyp ab, den wir produzieren wollen:
	
	\begin{table}[H]
		\centering
		\begin{tabular}{l | l}
			\textbf{Datentyp} & \textbf{Schreibweise} \\ \hline
			Ganzzahl & \lstinline[language = Racket]|42| \\
			Fließkommazahl & \lstinline[language = Racket]|21.5| \\
			Bruch & \lstinline[language = Racket]|2/3| \\
			Irrationale (ungenaue) Zahl & \lstinline[language = Racket]|#i2.1415| \\
			Komplexe Zahl & \lstinline[language = Racket]|2+5i| \\
			Wahrheitswert & \lstinline[language = Racket]|true|, \lstinline[language = Racket]|false|, \lstinline[language = Racket]|#t|, \lstinline[language = Racket]|#f|, \lstinline[language = Racket]|#true|, \lstinline[language = Racket]|#false| \\
			Symbol & \lstinline[language = Racket]|'symbol|, \lstinline[language = Racket]|'"string as symbol"| \\
		\end{tabular}
		\caption{\racket: Literale verschiedener Datentypen}
	\end{table}

	\paragraph{Symbol-Literale}
		Wenn wir Symbole verwenden, der Text hinter den Symbolen allerdings ein valides Literal eines anderen Datentyps darstellt, so wird das Symbol in den jeweiligen Datentyp umgeformt. Außerdem können wir auch Leerzeichen und Klammern innerhalb eines Symbols verwenden, wenn wir diesen einen Backslash (\(\backslash\)) voranstellen. Wenn wir viele Leerzeichen innerhalb eines Symbols verwenden wollen, können wir um den Inhalt des Symbols Senkrechtstriche setzen.
		
		Somit ist alles folgende äquivalent:
		\begin{itemize}
			\item \lstinline[language = Racket]|'"string as symbol"| \(\iff\) \lstinline[language = Racket]|"string as symbol"|
			\item \lstinline[language = Racket]|'12.34| \(\iff\) \lstinline[language = Racket]|12.34|
			\item \lstinline[language = Racket]|'\ \(| \(\iff\) \texttt{'| (|}
		\end{itemize}
	% end
% end

\subsection{Bezeichner und Konventionen}
	\implements{Bezeichnern und Konventionen}{identifier}{\racket}

	In \racket\, können annähernd alle Zeichen in Bezeichnern genutzt werden, u.a. \texttt{-}, \texttt{?}, usw.. Nicht möglich ist es, eine Zahl als das erste Zeichen eines Bezeichners zu wählen.
	
	Damit sind beispielsweise folgende Bezeichner gültig:
	\begin{itemize}
		\item \texttt{odd?}
		\item \texttt{-}
		\item \texttt{+-123?!}
	\end{itemize}

	\paragraph{Konventionen}
		Bei der Benennung von Variablen und Funktionen sind folgende Konventionen üblich:
		\begin{itemize}
			\item Es werden nur Kleinbuchstaben verwendet.
			\item Einzelne Wortabschnitte werden mit Bindestrichen getrennt (Beispiel: \texttt{is-this-real}).
			\item Zur Benennung von Funktionen gibt es noch weitere Konventionen:
				\begin{itemize}
					\item Funktionen zur Umwandlung von Datentyp A in Datentyp B werden \texttt{A->B} genannt.
					\item Funktionen, deren Rückgabe ein Wahrheitswert ist, wird in Fragezeichen nachgestellt. Beispiel: \texttt{odd?}
					\item \todo{Weiterführen}
				\end{itemize}
		\end{itemize}
	% end
% end

\subsection{Strukturierung des Codes}
	\todo{Schreiben}
% end

% end

\section{Anweisungen}
	\todo{Schreiben}

\subsection{Methodenaufrufe}
	\todo{Schreiben}
% end

\subsection{Konstanten}
	\todo{Schreiben}
% end

\subsection{Operatoren}
	\todo{Schreiben}

	\subsubsection{Arithmetik}
		\todo{Schreiben}
	% end
% end

\subsection{Abfragen/Vergleiche}
	\todo{Schreiben}

	\subsubsection{Gleichheit, Größer-/Kleiner-Gleich}
		\todo{Schreiben}
	% end
	
	\subsubsection{Prädikate}
		\todo{Schreiben}
		
		\begin{itemize}
			\item \lstinline[language = Racket|number?|
			\item \lstinline[language = Racket|real?|
			\item \lstinline[language = Racket|rational?|
			\item \lstinline[language = Racket|integer?|
			\item \lstinline[language = Racket|natural?|
			\item \lstinline[language = Racket|string?|
			\item \lstinline[language = Racket|cons?|, \lstinline[language = Racket|empty?|
		\end{itemize}
	% end
% end

% end

\section{Kontrollstrukturen}
	\implements{Kontrollstrukturen}{kontrollstrukturen}{\racket}

In diesem Abschnitt schauen wir uns an, wie Kontrollstrukturen in \racketText umgesetzt werden. \racketText kennt dabei die Kontrollstrukturen \textit{If} und \textit{Cond} (von \enquote{Conditional}), wobei \textit{Cond} nur eine Vereinfachung von vielen geschalteten Ifs darstellt.

In \racketText gibt es keine Schleifen, da \racketText eine funktionale Programmiersprache ist! Alle Wiederholungen werden über Rekursion \footnote{Siehe \refImpl{recursion}{\racket}} gelöst.

\subsection{If}
	Das \textit{If} ist die einfachste Form der Verzweigung und hat folgende Form:
	\begin{figure}[H]
		\centering
		\lstinline[language = Racket]|(if <Abfrage> <Wahr-Fall> <Falsch-Fall>)|
	\end{figure}
	Wird der Ausdruck \texttt{<Abfrage>} zu Wahr ausgewertet, so wird das Ergebnis von \texttt{<Wahr-Fall>} zurück gegeben. Ansonsten wir das Ergebnis von \texttt{<Falsch-Fall>} zurück gegeben.
	
	\warning{Bei einem If in \racketText müssen \textit{immer} sowohl Wahr- als auch Falsch-Fall angegeben werden!}
	
	\paragraph{Beispiele}
		\begin{itemize}
			\item \lstinline[language = Racket]|(if (= (modulo x 2) 0) 'even 'odd)| \\
				  Wertet zu \texttt{'even} aus, wenn \texttt{x} gerade ist und sonst zu \texttt{'odd}.
			\item \lstinline[language = Racket]|(if (> x y) x y)| \\
				  Wertet zu dem Maximum von \texttt{x} und \texttt{y} aus (also \( \max \{ x, y \} \)).
		\end{itemize}
	% end
% end

\subsection{Cond}
	Ein \textit{Cond} vereinfacht verschachtelte If-Abfragen immens, wie wir gleich sehen werden. Schauen wir uns dazu folgendes verschachteltes If an:
	\begin{figure}[H]
		\centering
		\begin{lstlisting}[language = Racket]
(if (< x y)
	-1
	(if (> x y)
		1
		0
	)
)
\end{lstlisting}
	\end{figure}

	Und nun noch die allgemeine Syntax von Cond:
	\begin{figure}[H]
		\centering
		\lstinline[language = Racket]|(cond (<Test1> <Ausdruck1>) *@\dots@* (<TestN> <AusdruckN>)) [(else <Ansonsten>)]|
	\end{figure}
	Wobei der gesamte Ausdruck zu \texttt{<AusdruckK>} auswertet genau dann wenn \texttt{<TestK>} Wahr wird und zu \texttt{<Ansonsten>} auswertet, wenn alle Tests negativ ausfallen.
	
	Dann können wir das obige If zu folgendem Code vereinfachen:
	\begin{figure}[H]
		\centering
		\begin{lstlisting}[language = Racket]
(cond
	((< x y) -1)
	((> x y)  1)
	((= x y)  0)
)
\end{lstlisting}
	\end{figure}

	Damit haben wir nun das nötige Handwerkszeug, um komplexe Programme zu schreiben.
% end
% end

\section{Funktionen}
	\todo{Schreiben}

\subsection{Bestandteile}
	\todo{Schreiben}
% end

\subsection{Verträge}
	\todo{Schreiben}
% end

\subsection{Rekursion}
	\todo{Schreiben}
% end

% end

\section{Fehlerbehandlung}
	\implements{Fehlerbehandlung}{fehlerbehandlung}{Racket}

In Racket gibt es die zwei typischen grundlegenden Arten von Fehlerbehandlung:
\begin{itemize}
	\item Exceptions in Form von Errors und
	\item Result Codes.
\end{itemize}

\subsection{Result Codes}
	\implements{Result Codes}{resultcodes}{Racket}
	
	In Racket werden Result Codes nicht besonders implementiert, wir können sie nur durch Fallunterscheidungen nutzen.
	
	\paragraph{Beispiel}
		Als Beispiel implementieren wir eine Funktion, welche die reelle Quadratwurzel einer Zahl berechnet. Ist die gegebene Zahl negativ, so gibt die Funktion \(-1\) zurück (Result Code).
		
		\begin{figure}[H]
			\centering
			\begin{lstlisting}[language = Racket]
(define (square-root-positive x)
	(if (< x 0)
		-1
		(sqrt x) ; Die Funktion sqrt gibt fuer negative Werte ein komplexes Ergebnis.
	)
)
\end{lstlisting}
		\end{figure}
	% end
% end

\subsection{Errors}
	\implements{Errors}{exceptions}{Racket}
	
	Außerdem können wir Errors einsetzen, um Fehler anzuzeigen. Diese sind meistens besser geeignet, da die Ausführung direkt abbricht und der Aufrufer nicht prüfen muss, ob ein Fehler aufgetreten ist.
	
	Ein Error lösen wir wie folgt aus:
	\begin{figure}[H]
		\centering
		\lstinline[language = Racket]|(error <Funktionsname> <Fehlermeldung>)|
	\end{figure}
	Der Funktionsname in dem der Fehler aufgetreten ist wird als Symbol übergeben, die Fehlermeldung als String.
	
	\paragraph{Beispiel}
		Wir implementieren folgende Funktion:
		\begin{figure}[H]
			\centering
			\begin{lstlisting}[language = Racket]
(define (square-root-positive x)
	(if (< x 0)
		(error 'square-root "Illegal value for real square root!")
		(sqrt x) ; Die Funktion sqrt gibt fuer negative Werte ein komplexes Ergebnis.
	)
)
			\end{lstlisting}
		\end{figure}
	
		Rufen wir die Funktion nun mit \lstinline[language = Racket]|(square-root-positive -4)| auf, so bekommen wir folgende Fehlermeldung: \texttt{square-root-real: Illegal value for real square root!}
	% end
% end

% end

\section{Datenstrukturen}
	\implements{Datenstrukturen}{datastruct}{Racket}

In diesem Abschnitt schauen wir uns an, was für Datenstrukturen in Racket implementiert werden und wie wir eigene hinzufügen können.

\subsection{Listen}
	\label{sec:racket_lists}

	Listen sind der zentrale Bestandteil von Racket, wie der Name der Ursprungssprache (LISP / List Processing) schon vermuten lässt.
	
	Listen sind in Racket die einzige Möglichkeit, \enquote{beliebig viele} Daten in einem Feld zu speichern und mit Hilfe von Rekursion über diese zu iterieren. Listen werden dabei als einfach gelinkte Listen abgelegt, das heißt eine Liste besteht aus den Kopf (\texttt{first}) und dem Rest der Liste (\texttt{rest}).
	
	Zum Umgang mit diesen Listen sind folgende Funktionen/Konstanten verfügbar (alle Daten können auch ad-hoc von einem Ausdruck berechnet werden):
	\begin{itemize}
		\item \lstinline[language = Racket]|(cons <Elemente> <Liste>)| \\ Funktion. Hängt das Element vorne an die Liste.
		\item \lstinline[language = Racket]|empty| \\ Konstante. entspricht einer leeren Liste und wird zum anlegen einer neuen Liste benötigt (als zweiter Parameter zur \texttt{cons}),
		\item \lstinline[language = Racket]|(list [Elemente])| \\ Funktion. Erstellt eine eine neue Liste, die alle gegebenen Elemente enthält. Die Elemente werden durch Leerzeichen getrennt.
		\item \lstinline[language = Racket]|(first <Liste>)| \\ Funktion. Gibt das erste Element der Liste zurück, also den Kopf.
		\item \lstinline[language = Racket]|(rest <Liste>)| \\ Funktion. Gibt den Rest der Liste (also die gesamte Liste ohne den Kopf) zurück. Ist die Liste leer, gibt es einen Fehler.
		\item \lstinline[language = Racket]|(second <Liste>)|, \lstinline[language = Racket]|(third <Liste>)|, \lstinline[language = Racket]|(fourth <Liste>)|, \lstinline[language = Racket]|(fifth <Liste>)|, \lstinline[language = Racket]|(sixth <Liste>)|, \lstinline[language = Racket]|(seventh <Liste>)|, \lstinline[language = Racket]|(eighth <Liste>)| \\ Funktionen. Geben das zweite/\dots/achte Element der Liste zurück.
		\item \lstinline[language = Racket]|(cons? <Arg>)| \\ Funktion. Gibt an, ob das gegebene Argument eine Liste ist.
		\item \lstinline[language = Racket]|(empty? <Arg)| \\ Funktion. Gibt an, ob das gegebene Argument eine leere Liste ist.
	\end{itemize}
% end

\subsection{Structs}
	\label{sec:racket_structs}

	Structs ermöglichen uns, viele Daten in einer Konstanten (oder einem Parameter) abzulegen und damit komplexe Datenstrukturen zu erstellen.
	
	\subsubsection{Definition}
		Zur Definition eines Struct-Typs wird folgender Code genutzt:
		\begin{figure}[H]
			\centering
			\lstinline[language = Racket]|(define-struct <Name> ([Attribute]))|
		\end{figure}
		Der Name gibt an, unter welchen Namen wir das Struct referenzieren können. Die Attribute definieren, unter welchem Namen wir Daten in dem Struct speichern können. Auf diese können wir anschließend zugreifen. Unterschiedliche Attribute können wir durch Leerzeichen separieren.
		
		\paragraph{Beispiel}
			Legen wir als Beispiel ein Struct zur Speicherung von Daten über einen Studierenden an:
			\begin{figure}[H]
				\centering
				\begin{lstlisting}[language = Racket]
(define-struct student (name matr-number))
\end{lstlisting}
			\end{figure}
		% end
	% end
	
	\subsubsection{Prädikate}
		Um zu Prüfen, ob eine Konstante \texttt{x} vom Typ des Structs \texttt{<Name>} ist, können wir die automatisch generierte Funktion \texttt{<Name>?} nutzen.
		
		\paragraph{Beispiel}
			Um zu prüfen, ob eine Variable \texttt{x} vom Typ \texttt{student} ist, nutzen wir folgende Code:
			\begin{figure}[H]
				\centering
				\begin{lstlisting}[language = Racket]
(student? x)
\end{lstlisting}
			\end{figure}
		% end
	% end
	
	\subsubsection{Nutzung, Attribute und Zugriff}
		Die Erstellung einer \enquote{Instanz} eines Structs \texttt{<Name>} geschieht wie folgt:
		\begin{figure}[H]
			\centering
			\begin{lstlisting}[language = Racket]
(make-<Name> [Parameter-Daten])
\end{lstlisting}
		\end{figure}
		Für die Parameter müssen wir die Daten in der korrekten Reihenfolge wie in der Struct-Definition übergeben.
		
		Um auf bestimmte Attribute eines Structs \texttt{x} zuzugreifen, nutzen wir folgenden Code:
		\begin{figure}[H]
			\centering
			\begin{lstlisting}[language = Racket]
(<Name>-<Attribut> x)
\end{lstlisting}
		\end{figure}
		Dies gibt den Wert des jeweiligen Attributs zurück.
		
		\paragraph{Beispiel}
			Wir nehmen als Beispiel wieder das Studierenden-Struct her. Nun wollen wir eine Funktion anlegen, die den Namen des Studierenden ausgibt, zwei Structs anlegen und die Funktion aufrufen.
			
			\begin{figure}[H]
				\centering
				\begin{lstlisting}[language = Racket]
(define (print-name x) (print (student-name)))

(define fd (make-student "Fabian Damken"  1234567))
(define fk (make-student "Florian Kadner" 8912345))
(define lr (make-student "Lukas Roehrig"  6789123))

(print-name fd)
(print-name fk)
(print-name lr)
\end{lstlisting}
			\end{figure}
		% end
	% end
% end

% end

\section{Funktionen höherer Ordnung}
	\todo{Schreiben}

\subsection{Lambdas}
	\todo{Schreiben}
% end

\subsection{Beispiele}
	\todo{Schreiben}

	\subsubsection{Filter}
		\todo{Schreiben}
	% end
	
	\subsubsection{Map}
		\todo{Schreiben}
	% end
	
	\subsubsection{Fold}
		\todo{Schreiben}
	% end
	
	\subsubsection{Vergleich von zwei Listen}
		\todo{Schreiben}
	% end
% end
% end

\section{Funktionen als Daten/Parameter}
	\todo{Schreiben}
% end

\section{Dokumentation}
	\implements{Dokumentation}{doku}{\racket}

\subsection{Veträge}
	\implements{Verträgen}{vetraege}{\racket}
	
	Nehmen wir an, wir haben eine Funktion \texttt{(moving-average data n)} implementiert, welche den gleitenden Durchschnitt einer Liste \texttt{data} mit den vorherigen \texttt{n} Daten berechnet. Diese Funktion gibt eine Liste mit der Länge \( \text{Length}(\texttt{moving-average}) - (n - 1) \) zurück.
	
	Den Vertrag der Funktion schreiben wir nun wie folgt in den Code:
	\begin{figure}[H]
		\centering
		\begin{lstlisting}[language = Racket]
;; moving-average :: (listof number) number -> (listof number)
(define (moving-average data n) *@\dots@*)
\end{lstlisting}
	\end{figure}
% end

\subsection{Funktionsdokumentation}
	Eine vollständige Funktionsdokumentation besteht aus:
	\begin{itemize}
		\item Einer Beschreibung, was die Methode tut.
		\item Dem Vertrag der Methode, wie oben beschrieben.
		\item Mindestens einem Nutzungsbeispiel.
	\end{itemize}

	Somit ist die folgende Methode korrekt dokumentiert:
	\begin{figure}[H]
		\centering
		\begin{lstlisting}[language = Racket]
;; factorial :: number -> number
;;
;; Berechnet die Fakultaet einer gegebenen natuerlichen Zahl.
;;
;; Beispiele:
;;   (factorial 3) -> 2
;;   (factorial 5) -> 120
(define (factorial n)
	(if (= n 1)
		1
		(* n (factorial (- n 1)))
	)
)
\end{lstlisting}
	\end{figure}
% end
% end

\section{Testen}
	\implements{Tests}{testing}{Racket}

Auch in Racket ist es natürlich möglich, Tests zu schreiben.

Hierzu haben wir drei grundlegende Funktionen zur Verfügung, die wie folgt heißen (auch hier wieder nur eine Auswahl, eine gesamte Liste steht in Abschnitt \ref{sec:racket_summary}):
\begin{itemize}
	\item \lstinline[language = Racket]|(check-expect <Ausdruck> <Erwartetes Ergebnis>)| Testet, ob der Ausdruck das erwartete Ergebnis produziert.
	\item \lstinline[language = Racket]|(check-within <Ausdruck> <Erwartetes Ergebnis> <Delta>)| Testet, ob der Ausdruck das erwartete Ergebnis \( \pm\text{Delta} \) produziert.
	\item \lstinline[language = Racket]|(check-error <Ausdruck> <Erwartete Fehlermeldung>)| Testet, ob der Ausdruck einen Fehler mit der erwarteten Meldung produziert.
\end{itemize}

% end

\section{Zusammenfassung}
	\label{sec:racket_summary}

\todo{Schreiben}
% end
