Wir werden uns nun als erstes mit Racket auseinandersetzen. Racket baut auf LISP auf und stellt einen Dialekt dieser funktionalen Sprache dar.

Im folgenden schauen wir uns Racket an und wie die in \ref{c:abstrakte_konzepte} Konzepte in der Sprache implementiert werden.

\section{Lexikalische Bestandteile}
	\subsection{Datentypen}
	\implements{Datentypen}{datentypen}{\racket}
	
	Im folgenden schauen wir uns an, was es in \racket\, für Datentypen gibt:
	\begin{itemize}
		\item Zahlen
			\begin{itemize}
				\item Ganzzahlen
				\item Fließkommazahlen
				\item Brüche
				\item Irrationale (ungenaue) Zahlen
				\item Komplexe Zahlen
			\end{itemize}
		\item Wahrheitswerte
		\item Symbole
		\item Strings
		\item Structs
		\item Listen
	\end{itemize}

	Dabei ist \racket\, aber nicht statisch typisiert, das heißt die Datentypen nicht mit angegeben werden, sondern es ist ausreichend, wenn zur Laufzeit der korrekte Datentyp in einer Variable gespeichert ist (es ist zum Beispiel nicht möglich, Strings zu addieren). Ist nicht der korrekte Datentyp gespeichert, so tritt ein Fehler auf.
	
	\paragraph{Symbole}
		Symbole sind einfache Zeichenketten, die ausschließlich verglichen werden können und weniger Funktionalität als Strings bieten.
		
		Allerdings ist die Verwendung von Symbolen sehr effizient und zu empfehlen, wenn wir mit der produzierten Zeichenkette nichts weiter tun wollen als sie zu vergleichen (dies tritt erstaunlich oft auf, öfter als man im Allgemeinen denkt).
	% end
	
	\paragraph{Listen}
		Listen ist einer der wichtigsten Datentypen in \racket. Wir werden uns diesen wichtigen Datentyp im Abschnitt \ref{sec:racket_lists} genauer anschauen.
	% end
	
	\paragraph{Sondertyp \textit{Struct}}
		Ein \textit{Struct} (eine Struktur) ist von dem Entwickler definierbar und ermöglicht es, komplexe Datentypen zu speichern. Wir werden uns diesen besonderen Datentyp im Abschnitt \ref{sec:structs} anschauen.
	% end
% end

\subsection{Literale}
	\implements{Literalen}{literale}{\racket}
	
	Wie wir Literale im Code ablegen, hängt von dem Datentyp ab, den wir produzieren wollen:
	
	\begin{table}[H]
		\centering
		\begin{tabular}{l | l}
			\textbf{Datentyp} & \textbf{Schreibweise} \\ \hline
			Ganzzahl & \lstinline[language = Racket]|42| \\
			Fließkommazahl & \lstinline[language = Racket]|21.5| \\
			Bruch & \lstinline[language = Racket]|2/3| \\
			Irrationale (ungenaue) Zahl & \lstinline[language = Racket]|#i2.1415| \\
			Komplexe Zahl & \lstinline[language = Racket]|2+5i| \\
			Wahrheitswert & \lstinline[language = Racket]|true|, \lstinline[language = Racket]|false|, \lstinline[language = Racket]|#t|, \lstinline[language = Racket]|#f|, \lstinline[language = Racket]|#true|, \lstinline[language = Racket]|#false| \\
			Symbol & \lstinline[language = Racket]|'symbol|, \lstinline[language = Racket]|'"string as symbol"| \\
		\end{tabular}
		\caption{\racket: Literale verschiedener Datentypen}
	\end{table}

	\paragraph{Symbol-Literale}
		Wenn wir Symbole verwenden, der Text hinter den Symbolen allerdings ein valides Literal eines anderen Datentyps darstellt, so wird das Symbol in den jeweiligen Datentyp umgeformt. Außerdem können wir auch Leerzeichen und Klammern innerhalb eines Symbols verwenden, wenn wir diesen einen Backslash (\(\backslash\)) voranstellen. Wenn wir viele Leerzeichen innerhalb eines Symbols verwenden wollen, können wir um den Inhalt des Symbols Senkrechtstriche setzen.
		
		Somit ist alles folgende äquivalent:
		\begin{itemize}
			\item \lstinline[language = Racket]|'"string as symbol"| \(\iff\) \lstinline[language = Racket]|"string as symbol"|
			\item \lstinline[language = Racket]|'12.34| \(\iff\) \lstinline[language = Racket]|12.34|
			\item \lstinline[language = Racket]|'\ \(| \(\iff\) \texttt{'| (|}
		\end{itemize}
	% end
% end

\subsection{Bezeichner und Konventionen}
	\implements{Bezeichnern und Konventionen}{identifier}{\racket}

	In \racket\, können annähernd alle Zeichen in Bezeichnern genutzt werden, u.a. \texttt{-}, \texttt{?}, usw.. Nicht möglich ist es, eine Zahl als das erste Zeichen eines Bezeichners zu wählen.
	
	Damit sind beispielsweise folgende Bezeichner gültig:
	\begin{itemize}
		\item \texttt{odd?}
		\item \texttt{-}
		\item \texttt{+-123?!}
	\end{itemize}

	\paragraph{Konventionen}
		Bei der Benennung von Variablen und Funktionen sind folgende Konventionen üblich:
		\begin{itemize}
			\item Es werden nur Kleinbuchstaben verwendet.
			\item Einzelne Wortabschnitte werden mit Bindestrichen getrennt (Beispiel: \texttt{is-this-real}).
			\item Zur Benennung von Funktionen gibt es noch weitere Konventionen:
				\begin{itemize}
					\item Funktionen zur Umwandlung von Datentyp A in Datentyp B werden \texttt{A->B} genannt.
					\item Funktionen, deren Rückgabe ein Wahrheitswert ist, wird in Fragezeichen nachgestellt. Beispiel: \texttt{odd?}
				\end{itemize}
		\end{itemize}
	% end
% end

\subsection{Strukturierung des Codes}
	In \racketText werden an allen Stellen Klammern verwendet. Zur Strukturierung ist es gut zu wissen, dass der Typ der Klammer (rund, geschweift, eckig) keinen Einfluss auf die Funktionalität hat, sofern der identische Typ zur Schließung verwendet wird.
	
	Das heißt, die folgenden Codes sind äquivalent:
	\begin{itemize}
		\item \lstinline[language = Racket]|(add 1 2 3)|
		\item \lstinline[language = Racket]|{add 1 2 3}|
		\item \lstinline[language = Racket]|[add 1 2 3]|
	\end{itemize}

	Dadurch kann der Quellcode an vielen Stellen übersichtlicher gestaltet werden.
% end

% end

\section{Anweisungen}
	\todo{Schreiben}

\subsection{Methodenaufrufe}
	\todo{Schreiben}
% end

\subsection{Konstanten}
	\todo{Schreiben}
% end

\subsection{Operatoren}
	\todo{Schreiben}

	\subsubsection{Arithmetik}
		\todo{Schreiben}
	% end
% end

\subsection{Abfragen/Vergleiche}
	\todo{Schreiben}

	\subsubsection{Gleichheit, Größer-/Kleiner-Gleich}
		\todo{Schreiben}
	% end
	
	\subsubsection{Prädikate}
		\todo{Schreiben}
		
		\begin{itemize}
			\item \lstinline[language = Racket|number?|
			\item \lstinline[language = Racket|real?|
			\item \lstinline[language = Racket|rational?|
			\item \lstinline[language = Racket|integer?|
			\item \lstinline[language = Racket|natural?|
			\item \lstinline[language = Racket|string?|
			\item \lstinline[language = Racket|cons?|, \lstinline[language = Racket|empty?|
		\end{itemize}
	% end
% end

% end

\section{Kontrollstrukturen}
	\todo{Schreiben}

\subsection{Verzweigungen}
	\todo{Schreiben} % if, cond
% end

% end

\section{Funktionen}
	\implements{Funktionen}{methoden}{Racket}

In diesem Abschnitt schauen wir uns an, wie Methoden als Funktionen in Racket umgesetzt werden.

Hierzu müssen wir verstehen, was eine Funktion in Racket genau ist: Eine deklarative Beschreibung dessen, was mit den Eingabedaten getan werden soll und wir das Ergebnis aussehen soll. Eine Rückgabe eines Wertes gibt es an sich nicht, die Funktion wird einfach ausgewertet und der entstehende Wert zurück gegeben.

\subsection{Bestandteile} % Name, Parameter, Rückgabe
	Eine Funktion definieren wir wie folgt:
	\begin{figure}[H]
		\centering
		\lstinline[language = Racket]|(define (<Name> [Parameter-Bezeichner]) <Ausdruck>)|
	\end{figure}
	Der Name muss dabei ein gültiger Bezeichner sein, die Parameter werden durch Leerzeichen getrennt hintereinander geschrieben und vom Aufrufer mit Daten gefüllt. Der gegebene Ausdruck kann dann die Parameter-Konstanten nutzen, um das Ergebnis zu berechnen.
	
	\paragraph{Beispiel}
		Schauen wir uns folgendes Beispiel an, welches den Durchschnittswert von 5 Zahlen berechnet:
		\begin{figure}[H]
			\centering
			\begin{lstlisting}[language = Racket]
(define (average a b c d e)
	(/ (+ a b c d e) 5)
)
\end{lstlisting}
		\end{figure}
	% end
% end

\subsection{Verträge}
	Verträge sind Teil der Dokumentation, siehe \refImpl{doku}{Racket}.
% end

\subsection{Rekursion}
	\implements{Rekursion}{rekursion}{Racket}
	
	In Racket ist Rekursion die einzige Möglichkeit, wie wir Code doppelt ausführen können. Die Nutzung der Rekursion ist, da Racket eine funktionale Sprache ist, sehr mathematisch, wie wir an folgendem Beispiel zur Berechnung der Fakultät sehen:
	\begin{figure}[H]
		\centering
		\begin{lstlisting}[language = Racket]
(define (factorial n)
	(if (= n 1)
		1
		(* n (factorial (- n 1)))
	)
)
\end{lstlisting}
	\end{figure}
% end

% end

\section{Fehlerbehandlung}
	\todo{Schreiben}

\subsection{Errors}
	\todo{Schreiben}
% end

\subsection{Result Codes}
	\todo{Schreiben}
% end
% end

\section{Datenstrukturen}
	\todo{Schreiben}

\subsection{Listen}
	\todo{Schreiben}
	
	\begin{itemize}
		\item \lstinline[language = Racket]|list|
		\item \lstinline[language = Racket]|cons|
		\item \lstinline[language = Racket]|first|, \lstinline[language = Racket]|second|, \lstinline[language = Racket]|third|, \lstinline[language = Racket]|fourth|, \lstinline[language = Racket]|fifth|, \lstinline[language = Racket]|sixth|, \lstinline[language = Racket]|seventh|, \lstinline[language = Racket]|eigth|
		\item \lstinline[language = Racket]|rest|
		\item \lstinline[language = Racket]|empty|
	\end{itemize}
% end

\subsection{Structs}
	\todo{Schreiben}
	
	\subsubsection{Definition}
		\todo{Schreiben}
	% end
	
	\subsubsection{Prädikate}
		\todo{Schreiben}
	% end
	
	\subsubsection{Nutzung, Attribute und Zugriff}
		\todo{Schreiben}
	% end
% end

% end

\section{Funktionen höherer Ordnung}
	\todo{Schreiben}

\subsection{Lambdas}
	\todo{Schreiben}
% end

\subsection{Beispiele}
	\todo{Schreiben}

	\subsubsection{Filter}
		\todo{Schreiben}
	% end
	
	\subsubsection{Map}
		\todo{Schreiben}
	% end
	
	\subsubsection{Fold}
		\todo{Schreiben}
	% end
	
	\subsubsection{Vergleich von zwei Listen}
		\todo{Schreiben}
	% end
% end
% end

\section{Dokumentation}
	\todo{Schreiben}

\subsection{Veträge}
	\todo{Schreiben}
% end
% end

\section{Testen}
	\todo{Schreiben}

% check-expect, check-within, check-error

% end

\section{Zusammenfassung}
	\label{sec:racket_summary}

\subsubsection{Arithmetik}
	\begin{table}[H]
		\centering
		\begin{tabular}{l l}
			\textbf{Name} & \textbf{Vertrag}
			\csvreader[head to column names]{parts/racket/summary-arithmetic.csv}{}{\\ \name & \texttt{\contract}}
		\end{tabular}
		\caption{Racket: Arithmetik}
	\end{table}
% end

\subsubsection{Entscheidungen}
	\begin{table}[H]
		\centering
		\begin{tabular}{l l}
			\textbf{Name} & \textbf{Vertrag}
			\csvreader[head to column names]{parts/racket/summary-decisions.csv}{}{\\ \name & \texttt{\contract}}
		\end{tabular}
		\caption{Racket: Entscheidungen}
	\end{table}
% end

\subsubsection{Logik}
	\begin{table}[H]
		\centering
		\begin{tabular}{l l}
			\textbf{Name} & \textbf{Vertrag}
			\csvreader[head to column names]{parts/racket/summary-logic.csv}{}{\\ \name & \texttt{\contract}}
		\end{tabular}
		\caption{Racket: Logik}
	\end{table}
% end

\subsubsection{Datenstrukturen}
	\begin{table}[H]
		\centering
		\begin{tabular}{l l}
			\textbf{Name} & \textbf{Vertrag}
			\csvreader[head to column names]{parts/racket/summary-structs.csv}{}{\\ \name & \texttt{\contract}}
		\end{tabular}
		\caption{Racket: Datenstrukturen}
	\end{table}
% end

\subsubsection{Tests}
	\begin{table}[H]
		\centering
		\begin{tabular}{l l}
			\textbf{Name} & \textbf{Vertrag}
			\csvreader[head to column names]{parts/racket/summary-tests.csv}{}{\\ \name & \texttt{\contract}}
		\end{tabular}
		\caption{Racket: Tests}
	\end{table}
% end

\subsubsection{Sonstiges}
	\begin{table}[H]
		\centering
		\begin{tabular}{l l}
			\textbf{Name} & \textbf{Vertrag}
			\csvreader[head to column names]{parts/racket/summary-other.csv}{}{\\ \name & \texttt{\contract}}
		\end{tabular}
		\caption{Racket: Sonstiges}
	\end{table}
% end
% end
