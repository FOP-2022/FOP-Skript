\implements{Dokumentation}{doku}{\racket}

\subsection{Veträge}
	\implements{Verträgen}{vetraege}{\racket}
	
	Nehmen wir an, wir haben eine Funktion \texttt{(moving-average data n)} implementiert, welche den gleitenden Durchschnitt einer Liste \texttt{data} mit den vorherigen \texttt{n} Daten berechnet. Diese Funktion gibt eine Liste mit der Länge \( \text{Length}(\texttt{moving-average}) - (n - 1) \) zurück.
	
	Den Vertrag der Funktion schreiben wir nun wie folgt in den Code:
	\begin{figure}[H]
		\centering
		\begin{lstlisting}[language = Racket]
;; moving-average :: (listof number) number -> (listof number)
(define (moving-average data n) *@\dots@*)
\end{lstlisting}
	\end{figure}
% end

\subsection{Funktionsdokumentation}
	Eine vollständige Funktionsdokumentation besteht aus:
	\begin{itemize}
		\item Einer Beschreibung, was die Methode tut.
		\item Dem Vertrag der Methode, wie oben beschrieben.
		\item Mindestens einem Nutzungsbeispiel.
	\end{itemize}

	Somit ist die folgende Methode korrekt dokumentiert:
	\begin{figure}[H]
		\centering
		\begin{lstlisting}[language = Racket]
;; factorial :: number -> number
;;
;; Berechnet die Fakultaet einer gegebenen natuerlichen Zahl.
;;
;; Beispiele:
;;   (factorial 3) -> 2
;;   (factorial 5) -> 120
(define (factorial n)
	(if (= n 1)
		1
		(* n (factorial (- n 1)))
	)
)
\end{lstlisting}
	\end{figure}
% end