\implements{Fehlerbehandlung}{fehlerbehandlung}{Racket}

In Racket gibt es die zwei typischen grundlegenden Arten von Fehlerbehandlung:
\begin{itemize}
	\item Exceptions in Form von Errors und
	\item Result Codes.
\end{itemize}

\subsection{Result Codes}
	\implements{Result Codes}{resultcodes}{Racket}
	
	In Racket werden Result Codes nicht besonders implementiert, wir können sie nur durch Fallunterscheidungen nutzen.
	
	\paragraph{Beispiel}
		Als Beispiel implementieren wir eine Funktion, welche die reelle Quadratwurzel einer Zahl berechnet. Ist die gegebene Zahl negativ, so gibt die Funktion \(-1\) zurück (Result Code).
		
		\begin{figure}[H]
			\centering
			\begin{lstlisting}[language = Racket]
(define (square-root-real x)
	(if (< x 0)
		-1
		(sqrt x) ; Die Funktion sqrt gibt fuer negative Werte ein komplexes Ergebnis.
	)
)
\end{lstlisting}
		\end{figure}
	% end
% end

\subsection{Errors}
	\implements{Errors}{exceptions}{Racket}
	
	Außerdem können wir Errors einsetzen, um Fehler anzuzeigen. Diese sind meistens besser geeignet, da die Ausführung direkt abbricht und der Aufrufer nicht prüfen muss, ob ein Fehler aufgetreten ist.
	
	Ein Error lösen wir wie folgt aus:
	\begin{figure}[H]
		\centering
		\lstinline[language = Racket]|(error <Funktionsname> <Fehlermeldung>)|
	\end{figure}
	Der Funktionsname in dem der Fehler aufgetreten ist wird als Symbol übergeben, die Fehlermeldung als String.
	
	\paragraph{Beispiel}
		Wir implementieren folgende Funktion:
		\begin{figure}[H]
			\centering
			\begin{lstlisting}[language = Racket]
(define (square-root-real x)
	(if (< x 0)
		(error 'square-root "Illegal value for real square root!")
		(sqrt x) ; Die Funktion sqrt gibt fuer negative Werte ein komplexes Ergebnis.
	)
)
			\end{lstlisting}
		\end{figure}
	
		Rufen wir die Funktion nun mit \lstinline[language = Racket]|(square-root-real -4)| auf, so bekommen wir folgende Fehlermeldung: \texttt{square-root-real: Illegal value for real square root!}
	% end
% end
