\introduces{von I/O}{io}

Mit I/O (Input/Output, Eingabe/Ausgabe) wird im Allgemeinen das Lesen von Daten (Input, Eingabe) und das Schreiben von Daten (Output, Ausgabe) bezeichnet.

Dies kann das Erstellen von Ordnern im Dateisystem sein, das Lesen von Dateien oder das Schreiben selbiger.

Jede Sprache kann unterschiedlich viele unterschiedliche I/O-Operationen durchführen, die meisten unterstützen aber mindestens das Lesen und Schreiben von Dateien.

\subsection{Allgemeiner Aufbau}
	\warning{In diesem Abschnitt schauen wir uns den allgemeinen Aufbau von Lese-/Schreiboperationen an, wie er in den meisten Sprachen zu finden ist. Es gibt aber auch Sprachen, bei denen dies grundlegend anders funktioniert.}
	
	\paragraph{Lesen}
		\begin{enumerate}
			\item Die Datei wird \textit{geöffnet} (\textit{open}). Hierbei wird überprüft, ob die Datei überhaupt existiert, ob das Programm die Datei Lesen darf, u.v.m..
			\item Die Daten werden (auf irgendeine Weise) gelesen\dots
			\item Die Datei wird \textit{geschlossen} (\textit{close}). Dabei werden die Ressourcen wieder freigegeben, sodass ein anderes Programm die Datei lesen kann, die Datei gelöscht werden kann, etc..
		\end{enumerate}
		
		\warning{Der Letzte Schritt (close) ist mit Abstand am wichtigsten, da hiermit sichergestellt wird, dass das umliegende System intakt bleibt. Eine Datei sollte \textit{immer} geschlossen werden, unabhängig ob bei dem Lesen ein Fehler aufgetreten ist.}
	% end
	
	\paragraph{Schreiben}
		Der Prozess, um eine Datei zu Schreiben ist dem Prozess zum Lesen sehr ähnlich, wie wir im folgenden sehen werden.
	
		\begin{enumerate}
			\item Die Datei wird \textit{geöffnet} (\textit{open}). Hierbei wird überprüft, ob die Datei überhaupt existiert, ob das Programm die Datei Schreiben darf, u.v.m..
			\item Die Daten werden (auf irgendeine Weise) schreiben\dots
			\item Die Datei wird \textit{geschlossen} (\textit{close}). Dabei werden die Ressourcen wieder freigegeben, sodass ein anderes Programm die Datei lesen (oder schreiben) kann, die Datei gelöscht werden kann, etc..
		\end{enumerate}
		
		\warning{Der Letzte Schritt (close) ist mit Abstand am wichtigsten, da hiermit sichergestellt wird, dass das umliegende System intakt bleibt. Eine Datei sollte \textit{immer} geschlossen werden, unabhängig ob bei dem Schreiben ein Fehler aufgetreten ist.}
	% end
% end