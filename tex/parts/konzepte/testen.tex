\introduces{von Testen}{testing}

Das Durchführen und Erstellen von Tests in der Softwareentwicklung sehr wichtig, stellt uns aber auch vor große Herausforderungen, zum Beispiel:
\begin{itemize}
	\item \textit{Was ist überhaupt ein Test?}
	\item \textit{Was ist eigentlich ein guter Test?}
	\item \textit{Wie kann ein sehr großes System getestet werden?}
	\item Was muss beim Testen beachtet werden?
	\item Wie können Tests automatisiert werden?
	\item \dots
\end{itemize}

Wir werden hier nur einige grundlegende Fragen (kursiv) beantworten und nicht sehr tief in die Materie hinabsteigen. Hierzu gibt es die Vorlesung \textit{Software Engineering} im dritten Semester und das Seminar \textit{Programmanalyse und Software-Tests} (Wahlbereich).

\paragraph{Was ist überhaupt ein Test?}
	Ein Test enthält eine systematische Beschreibung von:
	\begin{itemize}
		\item dem erwarteten Anfangszustand des Systems,
		\item den auszuführenden Schritten und
		\item dem erwarteten Endergebnis.
		\item Außerdem wird beschrieben, wie das Endergebnis validiert werden kann (beispielsweise durch Vergleich mit dem erwarteten Wert).
	\end{itemize}
	Außerdem muss ein Test wiederholbar sein.
	
	Dann kann einem Menschen oder Computer aufgetragen werden, die Schritte automatisiert durchzuführen und einen Bericht über das Ergebnis zu erstellen.
% end

\paragraph{Was ist eigentlich ein guter Test?}
	Ein Test sollte einige Parameter für den Standardfall vorhalten, aber vor allem Randfälle testen. Tests sollen somit dafür sorgen, dass möglichst alle Teile eines Systems getestet werden.
	
	\textbf{Beispiel:} Wenn die Division getestet werden soll, sind Tests wie \(\frac{4}{2}\) zwar interessant, aber was tut das System bei \(\frac{4}{0}\)? Oder berechnet es die Nachkommastellen von \(\frac{4}{3}\) korrekt?
% end

\paragraph{Was muss beim Testen beachtet werden?}
	Was genau beim Testen beachtet werden muss, hängt natürlich von dem System ab. Einige Dinge lassen sich aber allgemein sagen:
	\begin{itemize}
		\item Beim Vergleichen von Fließkommazahlen muss darauf geachtet werden, dass keine Äquivalenz-Vergleiche genutzt werden, da auch identische Berechnungen zu leicht anders gerundeten Zahlen führen können. Dies ist der internen Darstellung von Fließkommazahlen im Speicher zu schulden.
		\item Somit sollte nicht auf Gleichheit geprüft werden, sondern ob das erhaltene Ergebnis \(a\) in einem bestimmten Radius \(\varepsilon\) um das erwartete Ergebnis \(e\) liegt:
			\begin{equation*}
				a \in [e - \varepsilon, e + \varepsilon] \quad\iff\quad e - \varepsilon \leq a \leq e + \varepsilon
			\end{equation*}
		\item Dabei entspricht \(\varepsilon\) dem maximalen Wert, um den das Ergebnis abweichen darf (also der Genauigkeit).
		\item Meistens ist ein Wert wie \( \varepsilon = 10^{-8} = 0.00000001 \) vernünftig.
	\end{itemize}
% end