\todo{Schreiben}

\section{Aufbau}
	Dieses Skript ist in folgende Kapitel gegliedert:
	\begin{itemize}
		\item[\ref{c:abstrakte_konzepte}] Abstrakte Konzepte \\ In diesem Kapitel werden abstrakte Konzepte der Programmierung eingeführt, d.h. es wird über keine Programmiersprache an sich gesprochen.
		\item[\ref{c:racket}] \racket \\ Dieses Kapitel führt in die funktionale (\ref{sec:paradigma_imperativ}) Programmiersprache \racket ein, indem die im vorherigen Kapitel eingeführten Konzepte auf die Sprache angewendet werden.
		\item[\ref{c:java}] Java \\ Ebenso wir im Kapitel über \racket, nur werden hier die Konzepte auf die objektorientierte (\refIntr{oop}) und imperative (\ref{sec:paradigma_imperativ}) Programmiersprache Java angewendet.
		\item[\ref{c:abstraktion}] Abstraktion \\ Dieses Kapitel behandelt unterschiedliche Methodiken der Abstraktion, wie sie in funktionaler und objektorientierter Programmierung eingesetzt werden.
		\item[\ref{c:komplexitaet}] Komplexität und Landau-Symbolik \\ Dieses Kapitel führt in die Komplexität von Algorithmen und die Landau-Symbolik ein.
		\item[\ref{c:glossar}] Glossar \\ Ein Glossar, welches synchron zu dem Glossar im jeweiligen Moodle-Kurs ist.
	\end{itemize}
% end