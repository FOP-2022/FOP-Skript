\documentclass[a4paper, 11pt, accentcolor = tud3b]{tudreport}

% Core packages.
\usepackage[T1]{fontenc}
\usepackage[utf8]{inputenc}
\usepackage[ngerman]{babel}
% Other packages.
\usepackage[linesnumbered, ruled]{algorithm2e}
\usepackage{cite}
\usepackage{enumitem}
\usepackage{tabto}
\usepackage[mathcal]{euscript} % Get readable mathcal font.
\usepackage{hyperref}
\usepackage{listings}
\usepackage{mathtools}
\usepackage{xspace}
\usepackage[disable]{todonotes}
\usepackage{tcolorbox}
\usepackage[german = quotes]{csquotes}
\usepackage{tikz}
\usepackage{wrapfig}
\usepackage{stmaryrd}
\usepackage{float}
\usepackage{multicol}
\usepackage{csvsimple}
\usepackage{wrapfig}
\tcbuselibrary{skins}
\usetikzlibrary{arrows.meta, shapes, backgrounds, calc}

% Basic information.
\title{Funktionale und objektorientierte Programmierkonzepte \\ Version 1.0}
\subtitle{Fabian Damken (\href{mailto:fabian.damken@stud.tu-darmstadt.de?subject=[FoP-Skript]}{fabian.damken@stud.tu-darmstadt.de})}
\author{Fabian Damken}
\date{\today}

% Description-list styling.
\SetLabelAlign{parright}{\parbox[t]{\labelwidth}{\raggedleft#1}}
\setlist[description]{style = multiline, leftmargin = 4cm, align = parright}

\colorlet{lstcomments}{tud4c}
\colorlet{lstkeywords}{tud9d}
\colorlet{lstlinenumbers}{tud0c}
\colorlet{lststrings}{tud2c}
\colorlet{functionalmarker}{tud2b}
\colorlet{imperativemarker}{tud4b}
\colorlet{oopmarker}{tud6b}
\colorlet{errorred}{tud9b}
\colorlet{changedpurple}{tud11a}

\tikzset{> = { Latex[length = 2mm] }}

\lstdefinelanguage{Racket}{
	morekeywords = {
		list,
		cons,
		first,
		second,
		third,
		fourth,
		fifth,
		sixth,
		seventh,
		eighth,
		rest,
		empty,
		number,
		real,
		rational,
		integer,
		natural,
		string,
		true,
		false,
		\#t,
		\#f,
		\#true,
		\#false,
		define,
		remainder,
		if,
		cond,
		error,
		define-struct,
		print,
		lambda,
		else
	},
	sensitive = true,
	morecomment=[l]{;},
	morestring=[b]",
	keepspaces = true
}
\lstdefinestyle{base}{
	moredelim = **[is][\color{errorred}]{@@!}{@@@},
	moredelim = **[is][\color{changedpurple}]{@@?}{@@@}
}
\lstset{
	backgroundcolor = \color{white},
	basicstyle = \ttfamily\normalsize\color{black},
	breakatwhitespace = true,
	breaklines = true,
	breakautoindent = true,
	commentstyle = \color{lstcomments},
	escapeinside = {{*@}{@*}},
	keywordstyle = \color{lstkeywords},
	numbers = left,
	numberstyle = \tiny\color{lstlinenumbers},
	showstringspaces = false,
	stringstyle = \color{lststrings},
	tabsize = 4
}

% New commands.
\providecommand{\textbox}[1]{
	\begin{figure}[H]
		\centering
		\fbox{\parbox[c]{0.75\textwidth}{#1}}
	\end{figure}
}
\providecommand{\info}[1]{\textbox{\textbf{Info:} #1}}
\providecommand{\warning}[1]{\textbox{\textbf{Warnung:} #1}}
\providecommand{\funfact}[1]{\textbox{\textbf{Fun Fact:} #1}}
\providecommand{\memo}[1]{\textbox{\textbf{Merksatz:} #1}}
\providecommand{\forwhich}{\ensuremath{{\,\vert\,}}}
\providecommand{\abs}[1]{\ensuremath{{\lvert #1 \rvert}}}

\providecommand{\qed}{{\hfill q.e.d.}}

\providecommand{\subsubparagraph}[1]{\hspace{1cm} \textbf{#1:}}

\providecommand{\definition}[2]{\subparagraph{Definition (#1)} #2}
\providecommand{\notation}[2]{\subparagraph{Notation (#1)} #2}
\providecommand{\theorem}[1]{\subparagraph{Theorem} #1}
\providecommand{\intuition}[1]{\subsubparagraph{Intuition} #1}

\providecommand{\functional}{%
	\begin{tcolorbox}[standard jigsaw, colframe = functionalmarker, colback = tud2a, width = 2.3cm, height = 0.5cm, top = -2pt, left = 5pt, boxrule = 3pt, opacityback = 0.8]
		\footnotesize Funktional
	\end{tcolorbox}
}
\providecommand{\imperative}{%
	\begin{tcolorbox}[standard jigsaw, colframe = imperativemarker, colback = tud4a, width = 2.25cm, height = 0.5cm, top = -2pt, left = 7pt, boxrule = 3pt, opacityback = 0.8]
		\footnotesize Imperativ
	\end{tcolorbox}
}
\providecommand{\oop}{%
	\begin{tcolorbox}[standard jigsaw, colframe = oopmarker, colback = tud6a, width = 3.1cm, height = 0.5cm, top = -2pt, left = 6pt, boxrule = 3pt, opacityback = 0.8]
		\footnotesize Objektorientiert
	\end{tcolorbox}
}
\providecommand{\functionalMark}{%
	\raggedleft
	\functional
	\raggedright
}
\providecommand{\imperativeMark}{%
	\raggedleft
	\imperative
	\raggedright
}
\providecommand{\oopMark}{%
	\raggedleft
	\oop
	\raggedright
}
%\providecommand{\functionalMark}{\flushright \functional \flushleft}
%\providecommand{\imperativeMark}{\flushright \imperative \flushleft}
%\providecommand{\oopMark}{\flushright \oop \flushleft}

\providecommand{\introduces}[2]{\label{def:#2} Dieser Abschnitt führt das Konzept #1 ein.}
\providecommand{\implements}[3]{\label{imp:#2_#3} Dieser Abschnitt beschreibt die Implementierung von #1 in #3. Für die abstrakte Definition siehe \ref{def:#2}.}
\providecommand{\refIntr}[1]{\ref{def:#1}}
\providecommand{\refImpl}[2]{\ref{imp:#1_#2}}

\providecommand{\HREF}[1]{\href{#1}{#1}}

\begin{document}
	\bibliographystyle{alpha}

    \maketitle
    \tableofcontents
    \listoftodos

    \chapter{Einführung}
	    \label{c:einfuehrung}
    
        \todo{Schreiben}

\section{Aufbau}
	Dieses Skript ist in folgende Kapitel gegliedert:
	\begin{itemize}
		\item[\ref{c:abstrakte_konzepte}] Abstrakte Konzepte \\ In diesem Kapitel werden abstrakte Konzepte der Programmierung eingeführt, d.h. es wird über keine Programmiersprache an sich gesprochen.
		\item[\ref{c:racket}] \racket \\ Dieses Kapitel führt in die funktionale (\ref{sec:paradigma_funktional}) Programmiersprache \racket ein, indem die im vorherigen Kapitel eingeführten Konzepte auf die Sprache angewendet werden.
		\item[\ref{c:java}] Java \\ Ebenso wir im Kapitel über \racket, nur werden hier die Konzepte auf die objektorientierte (\ref{sec:paradigma_oop}) und imperative (\ref{sec:paradigma_imperativ}) Programmiersprache Java angewendet.
		\item[\ref{c:vergleich_racket_java}] Vergleich \racket $ \leftrightarrow $ Java \\ Die Gemeinsamkeiten und Unterschiede, welche sich nicht direkt aus der Abstraktion im Kapitel \ref{c:abstrakte_konzepte} ergeben.
		\item[\ref{c:glossar}] Glossar \\ Ein Glossar, welches synchron zu dem Glossar im jeweiligen Moodle-Kurs ist.
	\end{itemize}
% end
    % end

	\chapter{Abstrakte Konzepte}
		\label{c:abstrakte_konzepte}
	
		Dieses Kapitel führt ein in die abstrakten Konzepte, welche hinter eine Programmiersprache stehen.

Da sich nicht alle Konzepte auf alle Programmierparadigmen \footnote{Siehe \ref{sec:paradigmen}} anwenden lassen, ist jeder abschnitt mit
\begin{itemize}
	\item[] \functional
	\item[] \imperative
	\item[] \oop
\end{itemize}
gekennzeichnet, je nachdem, auf welches Paradigma sich das vorgestellte Konzept anwenden lässt. Die Markierungen werden am rechten Rand angebracht sein, sodass diese leicht zu finden sind.

\section{Programmierparadigmen}
	\label{sec:paradigmen}
	
	\subsection{Deklarativ}
		\label{sec:paradigma_deklarativ}
	
		Der der Deklarativen Programmierung steht die Beschreibung des Problems im Vordergrund, die Lösung wird hier meist automatisiert gefunden.
		
		Es steht somit im Vordergrund, \textit{welches} Problem gelöst werden soll und nicht \textit{wie} ein Problem gelöst werden soll. Hierdurch ist eine genaue Trennung von Problem und Implementierung möglich, was bei imperativen Programmiersprachen (\ref{sec:paradigma_imperativ}) gar nicht oder zumindest nicht trivial möglich ist.
		
		Das Paradigma der deklarativen Programmierung kann in weitere unterteilt werden, beispielsweise in funktionale (\ref{sec:paradigma_funktional}) und logische Sprachen. Logische Sprachen werden hier nicht weiter ausgeführt.
		
		\paragraph{Beispiele}
			\begin{itemize}
				\item SQL, Cypher (Abfragesprachen)
				\item Lisp, Racket, Haskell (Funktionale Sprachen)
				\item Prolog (Logische Sprache)
			\end{itemize}
		% end
	% end
	
	\subsection{Funktional}
		\label{sec:paradigma_funktional}
	
		Funktionale Programmiersprache sind Ausarbeitungen von deklarativen Sprachen (\ref{sec:paradigma_deklarativ}), bei denen ebenfalls die Beschreibung des Problems im Vordergrund steht. Sie werden oftmals zur Beschreibung von mathematischen Problem verwendet.
		
		In diesen Sprachen wir auf Konstrukte wie Schleifen (\ref{sec:konzept_schleife}) und Variablen (\ref{sec:konzept_variablen}) verzichtet, wodurch Seiteneffekte \todo{Beschreiben} verhindert werden und die Implementierung zur Lösung eines Problems robuster wird.
		
		Zur Abgrenzung von funktionalen Sprachen zu imperativen Sprachen siehe \ref{sec:paradigma_abgrenzung_funktional_imperativ}.
		
		\paragraph{Beispiele}
			\begin{itemize}
				\item LISP
				\item Racket
				\item Haskell
			\end{itemize}
		% end
	% end
	
	\subsection{Imperativ}
		\label{sec:paradigma_imperativ}
	
		Imperative Programmiersprachen sehen vor, dass der Entwickler beschreibt, \textit{wie} ein Problem zu lösen ist, wobei die Beschreibung des eigentlichen Problems (das \enquote{\textit{Was}}) fallen gelassen wird. Ein Programm besteht \enquote{aus einer Folge von Anweisungen [\dots], die vorgeben, in welcher Reihenfolge was vom Computer getan werden soll}. ~\cite{andreas2005grundkurs}
		
		Im Gegensatz zu deklarativen und funktionalen Sprachen ist die Korrektheit eines Algorithmus weniger offensichtlich und es werden Kontrollstrukturen wie Schleifen (\ref{sec:konzept_schleife}) und Variablen (\ref{sec:konzept_variablen}) eingeführt.
		
		Zur Abgrenzung von imperativen und funktionalen Sprachen siehe \ref{sec:paradigma_abgrenzung_funktional_imperativ}.
		
		\paragraph{Beispiele}
			\begin{itemize}
				\item Java
				\item C/C++
				\item Assembler
			\end{itemize}
		% end
	% end
	
	\subsection{Objektorientiert}
		\todo{Inhalt}
	% end
	
	\subsection{Abgrenzung Funktional $ \leftrightarrow $ Imperativ}
		\label{sec:paradigma_abgrenzung_funktional_imperativ}
		
		Die Abgrenzung von funktionalen und imperativen Sprachen lässt sich am besten anhand eines Beispiels erläutern:
		
		Gegeben sei das mathematische Problem, die Fakultät einer beliebigen natürlichen Zahl $ n \in \mathbb{N} _ 0 $ zu bestimmen. Mathematisch wird das Problem wie folgt rekursiv definiert:
		\begin{equation*}
			n! = f(n) = \begin{cases*}
				1 & \text{ falls } n = 0 \\
				n \cdot f(n - 1) & \text{ falls } n > 0 \\
			\end{cases*}
		\end{equation*}
		
		In einer (fiktionalen) funktionalen Sprache kann das Problem folgendermaßen implementiert werden:
		\begin{figure}[H]
			\centering
			\begin{lstlisting}
f(0) := 1
f(n) := n * f(n - 1)
			\end{lstlisting}
			\caption{Funktionale Implementierung der Fakultät}
		\end{figure}
		
		In einer (ebenfalls fiktionalen) imperativen Sprache kann das Problem folgendermaßen implementiert werden:
		\begin{figure}[H]
			\centering
			\begin{lstlisting}
function f(n)
	num = 1
	for i in 1..n
		num = num * i
	endloop
endfunction
			\end{lstlisting}
			\caption{Imperative Implementierung der Fakultät}
		\end{figure}
	% end
% end

	% end

    \chapter{Racket}
	    \label{c:racket}
    
        Wir werden uns nun als erstes mit Racket auseinandersetzen. Racket baut auf LISP auf und stellt einen Dialekt dieser funktionalen Sprache dar.

Im folgenden schauen wir uns Racket an und wie die in \ref{c:abstrakte_konzepte} Konzepte in der Sprache implementiert werden.

\section{Lexikalische Bestandteile}
	\subsection{Datentypen}
	\implements{Datentypen}{datentypen}{\racket}
	
	Im folgenden schauen wir uns an, was es in \racket\, für Datentypen gibt:
	\begin{itemize}
		\item Zahlen
			\begin{itemize}
				\item Ganzzahlen
				\item Fließkommazahlen
				\item Brüche
				\item Irrationale (ungenaue) Zahlen
				\item Komplexe Zahlen
			\end{itemize}
		\item Wahrheitswerte
		\item Symbole
		\item Strings
		\item Structs
		\item Listen
	\end{itemize}

	Dabei ist \racket\, aber nicht statisch typisiert, das heißt die Datentypen nicht mit angegeben werden, sondern es ist ausreichend, wenn zur Laufzeit der korrekte Datentyp in einer Variable gespeichert ist (es ist zum Beispiel nicht möglich, Strings zu addieren). Ist nicht der korrekte Datentyp gespeichert, so tritt ein Fehler auf.
	
	\paragraph{Symbole}
		Symbole sind einfache Zeichenketten, die ausschließlich verglichen werden können und weniger Funktionalität als Strings bieten.
		
		Allerdings ist die Verwendung von Symbolen sehr effizient und zu empfehlen, wenn wir mit der produzierten Zeichenkette nichts weiter tun wollen als sie zu vergleichen (dies tritt erstaunlich oft auf, öfter als man im Allgemeinen denkt).
	% end
	
	\paragraph{Listen}
		Listen ist einer der wichtigsten Datentypen in \racket. Wir werden uns diesen wichtigen Datentyp im Abschnitt \ref{sec:racket_lists} genauer anschauen.
	% end
	
	\paragraph{Sondertyp \textit{Struct}}
		Ein \textit{Struct} (eine Struktur) ist von dem Entwickler definierbar und ermöglicht es, komplexe Datentypen zu speichern. Wir werden uns diesen besonderen Datentyp im Abschnitt \ref{sec:structs} anschauen.
	% end
% end

\subsection{Literale}
	\implements{Literalen}{literale}{\racket}
	
	Wie wir Literale im Code ablegen, hängt von dem Datentyp ab, den wir produzieren wollen:
	
	\begin{table}[H]
		\centering
		\begin{tabular}{l | l}
			\textbf{Datentyp} & \textbf{Schreibweise} \\ \hline
			Ganzzahl & \lstinline[language = Racket]|42| \\
			Fließkommazahl & \lstinline[language = Racket]|21.5| \\
			Bruch & \lstinline[language = Racket]|2/3| \\
			Irrationale (ungenaue) Zahl & \lstinline[language = Racket]|#i2.1415| \\
			Komplexe Zahl & \lstinline[language = Racket]|2+5i| \\
			Wahrheitswert & \lstinline[language = Racket]|true|, \lstinline[language = Racket]|false|, \lstinline[language = Racket]|#t|, \lstinline[language = Racket]|#f|, \lstinline[language = Racket]|#true|, \lstinline[language = Racket]|#false| \\
			Symbol & \lstinline[language = Racket]|'symbol|, \lstinline[language = Racket]|'"string as symbol"| \\
		\end{tabular}
		\caption{\racket: Literale verschiedener Datentypen}
	\end{table}

	\paragraph{Symbol-Literale}
		Wenn wir Symbole verwenden, der Text hinter den Symbolen allerdings ein valides Literal eines anderen Datentyps darstellt, so wird das Symbol in den jeweiligen Datentyp umgeformt. Außerdem können wir auch Leerzeichen und Klammern innerhalb eines Symbols verwenden, wenn wir diesen einen Backslash (\(\backslash\)) voranstellen. Wenn wir viele Leerzeichen innerhalb eines Symbols verwenden wollen, können wir um den Inhalt des Symbols Senkrechtstriche setzen.
		
		Somit ist alles folgende äquivalent:
		\begin{itemize}
			\item \lstinline[language = Racket]|'"string as symbol"| \(\iff\) \lstinline[language = Racket]|"string as symbol"|
			\item \lstinline[language = Racket]|'12.34| \(\iff\) \lstinline[language = Racket]|12.34|
			\item \lstinline[language = Racket]|'\ \(| \(\iff\) \texttt{'| (|}
		\end{itemize}
	% end
% end

\subsection{Bezeichner und Konventionen}
	\implements{Bezeichnern und Konventionen}{identifier}{\racket}

	In \racket\, können annähernd alle Zeichen in Bezeichnern genutzt werden, u.a. \texttt{-}, \texttt{?}, usw.. Nicht möglich ist es, eine Zahl als das erste Zeichen eines Bezeichners zu wählen.
	
	Damit sind beispielsweise folgende Bezeichner gültig:
	\begin{itemize}
		\item \texttt{odd?}
		\item \texttt{-}
		\item \texttt{+-123?!}
	\end{itemize}

	\paragraph{Konventionen}
		Bei der Benennung von Variablen und Funktionen sind folgende Konventionen üblich:
		\begin{itemize}
			\item Es werden nur Kleinbuchstaben verwendet.
			\item Einzelne Wortabschnitte werden mit Bindestrichen getrennt (Beispiel: \texttt{is-this-real}).
			\item Zur Benennung von Funktionen gibt es noch weitere Konventionen:
				\begin{itemize}
					\item Funktionen zur Umwandlung von Datentyp A in Datentyp B werden \texttt{A->B} genannt.
					\item Funktionen, deren Rückgabe ein Wahrheitswert ist, wird in Fragezeichen nachgestellt. Beispiel: \texttt{odd?}
				\end{itemize}
		\end{itemize}
	% end
% end

\subsection{Strukturierung des Codes}
	In \racketText werden an allen Stellen Klammern verwendet. Zur Strukturierung ist es gut zu wissen, dass der Typ der Klammer (rund, geschweift, eckig) keinen Einfluss auf die Funktionalität hat, sofern der identische Typ zur Schließung verwendet wird.
	
	Das heißt, die folgenden Codes sind äquivalent:
	\begin{itemize}
		\item \lstinline[language = Racket]|(add 1 2 3)|
		\item \lstinline[language = Racket]|{add 1 2 3}|
		\item \lstinline[language = Racket]|[add 1 2 3]|
	\end{itemize}

	Dadurch kann der Quellcode an vielen Stellen übersichtlicher gestaltet werden.
% end

% end

\section{Anweisungen}
	\todo{Schreiben}

\subsection{Methodenaufrufe}
	\todo{Schreiben}
% end

\subsection{Konstanten}
	\todo{Schreiben}
% end

\subsection{Operatoren}
	\todo{Schreiben}

	\subsubsection{Arithmetik}
		\todo{Schreiben}
	% end
% end

\subsection{Abfragen/Vergleiche}
	\todo{Schreiben}

	\subsubsection{Gleichheit, Größer-/Kleiner-Gleich}
		\todo{Schreiben}
	% end
	
	\subsubsection{Prädikate}
		\todo{Schreiben}
		
		\begin{itemize}
			\item \lstinline[language = Racket|number?|
			\item \lstinline[language = Racket|real?|
			\item \lstinline[language = Racket|rational?|
			\item \lstinline[language = Racket|integer?|
			\item \lstinline[language = Racket|natural?|
			\item \lstinline[language = Racket|string?|
			\item \lstinline[language = Racket|cons?|, \lstinline[language = Racket|empty?|
		\end{itemize}
	% end
% end

% end

\section{Kontrollstrukturen}
	\todo{Schreiben}

\subsection{Verzweigungen}
	\todo{Schreiben} % if, cond
% end

% end

\section{Funktionen}
	\implements{Funktionen}{methoden}{Racket}

In diesem Abschnitt schauen wir uns an, wie Methoden als Funktionen in Racket umgesetzt werden.

Hierzu müssen wir verstehen, was eine Funktion in Racket genau ist: Eine deklarative Beschreibung dessen, was mit den Eingabedaten getan werden soll und wir das Ergebnis aussehen soll. Eine Rückgabe eines Wertes gibt es an sich nicht, die Funktion wird einfach ausgewertet und der entstehende Wert zurück gegeben.

\subsection{Bestandteile} % Name, Parameter, Rückgabe
	Eine Funktion definieren wir wie folgt:
	\begin{figure}[H]
		\centering
		\lstinline[language = Racket]|(define (<Name> [Parameter-Bezeichner]) <Ausdruck>)|
	\end{figure}
	Der Name muss dabei ein gültiger Bezeichner sein, die Parameter werden durch Leerzeichen getrennt hintereinander geschrieben und vom Aufrufer mit Daten gefüllt. Der gegebene Ausdruck kann dann die Parameter-Konstanten nutzen, um das Ergebnis zu berechnen.
	
	\paragraph{Beispiel}
		Schauen wir uns folgendes Beispiel an, welches den Durchschnittswert von 5 Zahlen berechnet:
		\begin{figure}[H]
			\centering
			\begin{lstlisting}[language = Racket]
(define (average a b c d e)
	(/ (+ a b c d e) 5)
)
\end{lstlisting}
		\end{figure}
	% end
% end

\subsection{Verträge}
	Verträge sind Teil der Dokumentation, siehe \refImpl{doku}{Racket}.
% end

\subsection{Rekursion}
	\implements{Rekursion}{rekursion}{Racket}
	
	In Racket ist Rekursion die einzige Möglichkeit, wie wir Code doppelt ausführen können. Die Nutzung der Rekursion ist, da Racket eine funktionale Sprache ist, sehr mathematisch, wie wir an folgendem Beispiel zur Berechnung der Fakultät sehen:
	\begin{figure}[H]
		\centering
		\begin{lstlisting}[language = Racket]
(define (factorial n)
	(if (= n 1)
		1
		(* n (factorial (- n 1)))
	)
)
\end{lstlisting}
	\end{figure}
% end

% end

\section{Fehlerbehandlung}
	\todo{Schreiben}

\subsection{Errors}
	\todo{Schreiben}
% end

\subsection{Result Codes}
	\todo{Schreiben}
% end
% end

\section{Datenstrukturen}
	\todo{Schreiben}

\subsection{Listen}
	\todo{Schreiben}
	
	\begin{itemize}
		\item \lstinline[language = Racket]|list|
		\item \lstinline[language = Racket]|cons|
		\item \lstinline[language = Racket]|first|, \lstinline[language = Racket]|second|, \lstinline[language = Racket]|third|, \lstinline[language = Racket]|fourth|, \lstinline[language = Racket]|fifth|, \lstinline[language = Racket]|sixth|, \lstinline[language = Racket]|seventh|, \lstinline[language = Racket]|eigth|
		\item \lstinline[language = Racket]|rest|
		\item \lstinline[language = Racket]|empty|
	\end{itemize}
% end

\subsection{Structs}
	\todo{Schreiben}
	
	\subsubsection{Definition}
		\todo{Schreiben}
	% end
	
	\subsubsection{Prädikate}
		\todo{Schreiben}
	% end
	
	\subsubsection{Nutzung, Attribute und Zugriff}
		\todo{Schreiben}
	% end
% end

% end

\section{Funktionen höherer Ordnung}
	\todo{Schreiben}

\subsection{Lambdas}
	\todo{Schreiben}
% end

\subsection{Beispiele}
	\todo{Schreiben}

	\subsubsection{Filter}
		\todo{Schreiben}
	% end
	
	\subsubsection{Map}
		\todo{Schreiben}
	% end
	
	\subsubsection{Fold}
		\todo{Schreiben}
	% end
	
	\subsubsection{Vergleich von zwei Listen}
		\todo{Schreiben}
	% end
% end
% end

\section{Dokumentation}
	\todo{Schreiben}

\subsection{Veträge}
	\todo{Schreiben}
% end
% end

\section{Testen}
	\todo{Schreiben}

% check-expect, check-within, check-error

% end

\section{Zusammenfassung}
	\label{sec:racket_summary}

\subsubsection{Arithmetik}
	\begin{table}[H]
		\centering
		\begin{tabular}{l l}
			\textbf{Name} & \textbf{Vertrag}
			\csvreader[head to column names]{parts/racket/summary-arithmetic.csv}{}{\\ \name & \texttt{\contract}}
		\end{tabular}
		\caption{Racket: Arithmetik}
	\end{table}
% end

\subsubsection{Entscheidungen}
	\begin{table}[H]
		\centering
		\begin{tabular}{l l}
			\textbf{Name} & \textbf{Vertrag}
			\csvreader[head to column names]{parts/racket/summary-decisions.csv}{}{\\ \name & \texttt{\contract}}
		\end{tabular}
		\caption{Racket: Entscheidungen}
	\end{table}
% end

\subsubsection{Logik}
	\begin{table}[H]
		\centering
		\begin{tabular}{l l}
			\textbf{Name} & \textbf{Vertrag}
			\csvreader[head to column names]{parts/racket/summary-logic.csv}{}{\\ \name & \texttt{\contract}}
		\end{tabular}
		\caption{Racket: Logik}
	\end{table}
% end

\subsubsection{Datenstrukturen}
	\begin{table}[H]
		\centering
		\begin{tabular}{l l}
			\textbf{Name} & \textbf{Vertrag}
			\csvreader[head to column names]{parts/racket/summary-structs.csv}{}{\\ \name & \texttt{\contract}}
		\end{tabular}
		\caption{Racket: Datenstrukturen}
	\end{table}
% end

\subsubsection{Tests}
	\begin{table}[H]
		\centering
		\begin{tabular}{l l}
			\textbf{Name} & \textbf{Vertrag}
			\csvreader[head to column names]{parts/racket/summary-tests.csv}{}{\\ \name & \texttt{\contract}}
		\end{tabular}
		\caption{Racket: Tests}
	\end{table}
% end

\subsubsection{Sonstiges}
	\begin{table}[H]
		\centering
		\begin{tabular}{l l}
			\textbf{Name} & \textbf{Vertrag}
			\csvreader[head to column names]{parts/racket/summary-other.csv}{}{\\ \name & \texttt{\contract}}
		\end{tabular}
		\caption{Racket: Sonstiges}
	\end{table}
% end
% end

    % end

    %\chapter{Java}
	%    \label{c:java}
    %
    %    \section{Lexikalische Bestandteile}
	\subsection{Datentypen}
	\implements{Datentypen}{datentypen}{Java}
	
	\todo{Statische Typisierung}
	
	In Java existieren viele Datentypen, die in zwei \footnote{Durch Project Valhalla \HREF{http://openjdk.java.net/projects/valhalla/} werden sich hier einige Dinge ändern.} Kategorien unterteilt werden können: \todo{Ist das mit Valhalla korrekt?}
	\begin{itemize}
		\item Primitive Datentypen
		\item Objektreferenzen
	\end{itemize}
	
	\paragraph{Primitive Datentypen}
		Einer der Unterschiede zwischen primitiven Datentypen und Objektreferenzen ist, dass Daten, welche in primitiven Datentypen gespeichert sind, mit Pass-by-Value weitergegeben werden. \todo{Pass-by-Value erklären.} Außerdem sind ist die Anzahl an primitiven Datentypen begrenzt und die Datentypen sind von vornherein festgelegt. Ferner gibt es große Unterschiede bei der Behandlung von Konstanten, die wir später betrachten werden. \todo{Primitive vs. Objekte: Konstanten} als ersten Anhaltspunkt eignet sich, dass primitive Datentypen mit einem kleinen Buchstaben und Objektreferenztypen mit einem großen Buchstaben beginnen.
		
		Es existieren folgende primitive Datentypen:
		\begin{table}[H]
			\centering
			\begin{tabular}{l | l | l | l}
				Schlüsselwort    & Typ            & Beschreibung               & Wertebereich                                                      \\ \hline
				\texttt{byte}    & Ganzzahl       & Vorzeichenbehaftet, 8 Bit  & \( -(2 ^ { 7}) \) bis \( 2 ^ { 7} - 1 \)                          \\
				\texttt{short}   & Ganzzahl       & Vorzeichenbehaftet, 16 Bit & \( -(2 ^ {15}) \) bis \( 2 ^ {15} - 1 \)                          \\
				\texttt{int}     & Ganzzahl       & Vorzeichenbehaftet, 32 Bit & \( -(2 ^ {31}) \) bis \( 2 ^ {31} - 1 \)                          \\
				\texttt{long}    & Ganzzahl       & Vorzeichenbehaftet, 64 Bit & \( -(2 ^ {63}) \) bis \( 2 ^ {63} - 1 \)                          \\
				\texttt{float}   & Fließkommazahl & einfache Genauigkeit       & \( 1,4 \cdot 10 ^ {-45} \) bis \( \approx 3,4 \cdot 10 ^ {38} \)  \\
				\texttt{double}  & Fließkommazahl & doppelte Genauigkeit       & \( 4,9 \cdot 10 ^ {324} \) bis \( \approx 1,8 \cdot 10 ^ {308} \) \\
				\texttt{char}    & Charakter      & Unicode-Code, 16 Bit       & \( 0 \) bis \( 2 ^ {16} - 1 \)                                    \\
				\texttt{boolean} & Wahrheitswert  &                            & \texttt{true}/\texttt{false}
			\end{tabular}
			\caption{Liste der primitiven Datentypen in Java}
		\end{table}
		
		Hierbei fällt auf, dass es in Java keinen eingebauten Datentyp für vorzeichenfreie Zahlen (\enquote{unsigned}) gibt. Dies kann bei der Verarbeitung von Binärdaten (beispielsweise bei Netzwerkkommunikation) zu Fehlern führen.
	% end
	
	\todo{Objektreferenzen und String}
% end

\subsection{Literale}
	\implements{Literalen}{literale}{Java}
	
	In Java gibt es Schreibweisen für Literale für alle Datentypen, wobei die Erstellung von Objekten einen Sonderfall darstellt und nicht vollständig als Literal bezeichnet werden kann (es können zwar alle Argumente fest im Code stehen, das Objekt selbst wird allerdings erst zur Laufzeit erstellt).
	
	In der folgenden Tabelle sind sämtliche syntaktische Methoden zu Definition von Literalen gelistet:
	\begin{table}[H]
		\centering
		\begin{tabular}{l | l}
			Datentyp         & Primäre Schreibweise \\
			\hline
			\texttt{byte}    & \texttt{123}, \texttt{-123} \\
			\texttt{short}   & \texttt{1234}, \texttt{1234} \\
			\texttt{int}     & \texttt{12345}, \texttt{12345} \\
			\texttt{long}    & \texttt{123456}, \texttt{123456} \\
			\texttt{float}   & \texttt{12.34F}, \texttt{0.34F}/\texttt{.34F} \\
			\texttt{double}  & \texttt{123.456}, \texttt{0.456}/\texttt{.456} \\
			\texttt{char}    & \texttt{'a'} \\
			\texttt{boolean} & \texttt{true}, \texttt{false} \\
			\texttt{String}  & \texttt{"Hello, World!"} \\
			\texttt{Object}  & \texttt{null} \\
		\end{tabular}
	\end{table}
	
	Bei Literalen von Zahlen gibt es außerdem folgende Besonderheiten:
	\begin{itemize}
		\item Bei einem \texttt{float}-Literal muss ein \enquote{\texttt{F}} am Ende des Literals angehängt werden, damit das Literal als \texttt{float} und nicht als \texttt{double} interpretiert wird. Die Groß-/Kleinschreibung ist irrelevant.
		\item Bei einem \texttt{long}-Literal kann ein \enquote{\texttt{L}} am Ende des Literals angehängt werden, damit das Literal als \texttt{long} und nicht als \texttt{int} interpretiert wird. Die Groß-/Kleinschreibung ist irrelevant, aufgrund der Ähnlichkeit von \enquote{\texttt{l}} und \enquote{\texttt{1}} wir allerdings ein großes \enquote{\texttt{L}} empfohlen.
		\item Bei allen Ganzzahlen (\texttt{byte}, \texttt{short}, \texttt{int}, \texttt{long}) können die Zahlen mit den Zahlensystemen Binär, Oktal, Dezimal und Hexadezimal eingegeben werden, wobei Dezimal sinnvollerweise der Standard ist. Zur Nutzung hiervon müssen den Werten bestimmte Zeichenketten vorangestellt werden. Dies sind \texttt{0b} für Binär, \texttt{0} für Oktal, nichts für Dezimal und \texttt{0x} für Hexadezimal. \\ Das heißt, die folgenden Literale sind äquivalent:
			\begin{itemize}
				\item \texttt{0b101010}
				\item \texttt{052}
				\item \texttt{42}
				\item \texttt{0x2A}
			\end{itemize}
			Wobei auch hier die Groß-/Kleinschreibung irrelevant ist, für den Prefix allerdings die Kleinschreibung und für die Zahl die Großschreibung empfohlen wird.
	\end{itemize}
	
	\warning{Wird bei Zahlen eine \texttt{0} vorangestellt, wird die Zahl Oktal interpretiert! Das heißt es gilt \texttt{010 \(\neq\) 10}.}
	
	\todo{Schreiben}
	
	% Primitive, null
% end

\subsection{Schlüsselwörter}
	\implements{Schlüsselwörtern}{keywords}{Java}
	
	In Java existieren folgende Schlüsselwörter (kursiv geschriebene Themen werden wir nicht ausführlicher betrachten):
	\begin{description}
        \item[\texttt{abstract}] Markiert eine\dots
	        \begin{description}
	        	\item[Klasse] das heißt, diese kann abstrakte Methoden enthalten.
	        	\item[Methode] die von Unterklassen implementiert werden muss.
	        \end{description}
        \item[\texttt{continue}] Fährt in einer Schleife mit dem nächsten Element fort.
        \item[\texttt{for}] Leitet eine for-Schleife ein.
        \item[\texttt{new}] Operator zur Erstellung eines neuen Objektes einer Klasse.
        \item[\texttt{switch}] Leitet eine switch-Anweisung ein.
        \item[\textit{\texttt{assert}}] \textit{Legt bestimmte Bedingungen fest, die für Parameter gelten müssen. Gelten diese wird, wird ein Fehler ausgelöst.} \todo{Wird das behandelt?}
        \item[\texttt{default}]
	        \begin{itemize}
	        	\item Default-Fall in einer switch-Anweisung.
	        	\item Definition einer Default-Methode innerhalb eines Interfaces.
	        	\item \textit{Definition des Default-Wertes einer Methode in einer Annotation}
	        \end{itemize}
        \item[\texttt{if}] Leitet eine if-Verzweigung ein.
        \item[\texttt{package}] Definition des Packages einer Klasse.
        \item[\texttt{synchronized}] Markiert eine Methode oder einen Codeblock als synchron, das heißt es kann maximal ein Thread zur gleichen Zeit die Methode \enquote{betreten}. \todo{Wird das behandelt?}
        \item[\texttt{boolean}] Datentyp.
        \item[\texttt{do}] Leitet eine do-while-Schleife ein.
        \item[\texttt{goto}] Reserviert. Löst ausschließlich einen Compilefehler aus.
        \item[\texttt{private}] Markiert eine Klasse, einen Konstruktor, eine Methode oder ein Attribut als privat.
        \item[\texttt{this}] Referenz auf die Instanz des aktuellen Objektes.
        \item[\texttt{break}] Bricht die Ausführung einer Schleife ab.
        \item[\texttt{double}] Datentyp.
        \item[\texttt{implements}] Implementiert ein Interface.
        \item[\texttt{protected}] Markiert eine Klasse, einen Konstruktor, eine Methode oder ein Attribut als protected.
        \item[\texttt{throw}] Wirft eine Instanz einer Exception.
        \item[\texttt{byte}] Datentyp.
        \item[\texttt{else}] Leitet einen else-Block ein.
        \item[\texttt{import}] Importiert eine Klasse/Methode aus einem anderen Paket.
        \item[\texttt{public}]  Markiert eine Klasse, einen Konstruktor, eine Methode oder ein Attribut als public.
        \item[\texttt{throws}] Definiert, dass ein Konstruktor/eine Methode eine bestimmte Exception werfen kann.
        \item[\texttt{case}] Leitet einen Fall eines switch-Ausdruckes ein.
        \item[\texttt{enum}] Leitet die Definition eines Enums ein.
        \item[\texttt{instanceof}] Operator zum Prüfen, ob eine Instanz eine Instanz einer anderen Klasse ist.
        \item[\texttt{return}] Gibt einen Wert zurück und bricht die Ausführung der Methode/des Konstruktors ab.
        \item[\textit{\texttt{transient}}] \textit{Definiert, dass ein bestimmte Attribut einer Instanz nicht mit serialisiert wird.}
        \item[\texttt{catch}] Leitet einen catch-Block ein.
        \item[\texttt{extends}]
	        \begin{description}
	        	\item[Klasse] Erweitert eine bestehende (möglicherweise abstrakte) Klasse.
	        	\item[Interface] Erweitert ein bestehendes Interface.
	        \end{description}
        \item[\texttt{int}] Datentyp.
        \item[\texttt{short}] Datentyp.
        \item[\texttt{try}] Leitet einen try-Block ein.
        \item[\texttt{char}] Datentyp.
        \item[\texttt{final}]
	        \begin{description}
	        	\item[Klasse] Die Klasse ist nicht vererbbar.
	        	\item[Methode] Die Methode ist nicht überschreibbar.
	        	\item[Variable] Die Variable ist nur einmal zuweisbar.
	        \end{description}
        \item[\texttt{interface}] Leitet die Definition eines Interfaces ein.
        \item[\texttt{static}] Markiert eine innere Klasse, eine Methode oder ein Attribut als statisch.
        \item[\texttt{void}] \enquote{Datentyp} als Platzhalter für \enquote{Nichts}.
        \item[\texttt{class}] Leitet die Definition einer Klasse ein.
        \item[\texttt{finally}] Leitet einen finally-Block ein.
        \item[\texttt{long}] Datentyp.
        \item[\textit{\texttt{strictfp}}] \textit{Legt fest, dass innerhalb einer Methode/einer Klasse nur strikte mathematische Operationen verwendet werden, sodass diese nicht optimiert werden sollen (es wird sich strikt an den Standard gehalten).}
        \item[\textit{\texttt{volatile}}] \textit{Markiert ein Attribut, sodass Modifikationen an diesem atomar durchgeführt werden.}
        \item[\texttt{const}] Reserviert. Löst ausschließlich einen Compilefehler aus.
        \item[\texttt{float}] Datentyp.
        \item[\textit{\texttt{native}}] \textit{Markiert die Implementierung einer Methode als nativ, das heißt, die Implementierung liegt in nativem Code (C/C++) vor. Siehe JNI (Java Native Interface).}
        \item[\texttt{super}] Referenz auf die Instanz der Oberklasse des aktuellen Objektes.
        \item[\texttt{while}] Leitet eine while-Schleife ein.
	\end{description}
	
	Die genaue Bedeutung der obigen Schlüsselwörter werden wir in den jeweiligen Kapiteln genauer betrachten.
% end

\subsection{Bezeichner und Konventionen}
	\implements{Bezeichnern und Namenskonventionen}{identifier}{Java}
	
	In Java können Zeichenketten als Bezeichner gelten, wenn sie folgenden Bedingungen genügen:
	\begin{itemize}
		\item Sie bestehen nur aus \texttt{a} bis \texttt{z}, \texttt{0} bis \texttt{9}, \texttt{\_} oder \texttt{\$}.
		\item Sie beginnen nur mit \texttt{a} bis \texttt{z}, \texttt{\_} oder \texttt{\$}.
	\end{itemize}
	
	Zur Benennung sind außerdem folgende Konventionen zu empfehlen:
	\begin{itemize}
		\item Namen von Klassen beginnen mit einem Großbuchstaben.
		\item Namen von Methoden/Parametern/Variablen/etc. beginnen mit einem Kleinbuchstaben.
		\item Namen von Klassen sollen Subjekte und Objekte sein. Beispiel: \enquote{\texttt{User}}
		\item Namen von Methoden sollen mit einem Adjektiv beginnen. Beispiel: \enquote{\texttt{generateAccessToken}}
	\end{itemize}
	
	\info{Die oben Bedingungen, wann eine Zeichenkette als Bezeichner dienen kann, stellen Vereinfachungen dar. Streng genommen können alle Zeichen verwendet werden, für die die Methoden \texttt{Character.isJavaIdentifierStart(char)} bzw. \texttt{Character.isJavaIdentifierPart(char)} den Wert \texttt{true} ergeben. Damit \textit{wären} auch Bezeichner wie \enquote{\texttt{\(\Delta\Psi\)}} möglich.}
% end

\subsection{Operatoren}
	\implements{Operatoren}{lexOperatoren}{Java}
	
	In Java existieren die folgenden Operatoren, die genaue Bedeutung werden wir im Abschnitt \refImpl{operatoren}{Java} behandeln:
	\begin{table}[H]
		\centering
		\begin{tabular}{l | l}
			Kategorie & Ausprägungen \\
			\hline
			Arithmetische Verknüpfungen & \texttt{*}, \texttt{/}, \texttt{\%}, \texttt{+}, \texttt{-} \\
			Unäre Arithmetik            & \texttt{expr++}, \texttt{expr--}, \texttt{++expr}, \texttt{--expr}, \texttt{+expr}, \texttt{-expr} \\
			Logik                       & \texttt{!}, \texttt{\&\&}, \texttt{||}, \texttt{\textasciicircum}, \texttt{?:} \\
			Bitweise Logik              & \texttt{\textasciitilde}, \texttt{\&}, \texttt{|}, \texttt{\textasciicircum} \\
			Verschiebung (Shift)        & \texttt{<{}<}, \texttt{>{}>}, \texttt{>{}>{}>} \\
			Vergleiche                  & \texttt{<}, \texttt{>}, \texttt{<=}, \texttt{>=}, \texttt{==}, \texttt{!=}, \texttt{instanceof} \\
			Zuweisungen                 & \texttt{=}, \texttt{+=}, \texttt{-=}, \texttt{*=}, \texttt{/=}, \texttt{\%=}, \texttt{\textasciicircum=}, \texttt{|=}, \texttt{<{}<=}, \texttt{>{}>=}, \texttt{>{}>{}>=} \\
		\end{tabular}
		\caption{Java: Operatoren}
	\end{table}
	
	\todo{Schreiben}
% end

\subsection{Strukturierung des Codes, Packages und Imports}
	\implements{Paketen und logischen Dateistrukturen}{namespaces}{Java}
	
	\todo{Schreiben}
	
	% Klammerung, Kommentare, Whitespaces, Packages, Imports
% end

% end

\section{Anweisungen}
	Schauen wir uns nun an, wie man Dinge in Java tut, also wie wir Anweisungen und Ausdrücke formulieren können.

\subsection{Variablen}
	\implements{Variablen}{variablen}{Java}
	
	Die allgemeine Syntax zur Deklaration einer Variablen ist:
	\begin{figure}[H]
		\centering
		\lstinline|<modifier> <typ> <name>;|
	\end{figure}
	Dabei ist \texttt{<modifier>} eine Reihe von Schlüsselwörtern, welche das Verhalten der Variablen modifizieren (genannt \enquote{Modifier}). Diese werden wir uns weiter unten genau anschauen. \texttt{<typ>} ist der Datentyp der Variablen (dies kann ein primitiver Datentyp aber auch ein Referenztyp sein). Der Name der Variablen wird mit \texttt{<name>} festgelegt.
	
	\subsubsection{Modifier}
		Für eine lokale Variable (das heißt eine Variable innerhalb eines Codeblocks oder als Parameter) existiert ausschließlich folgender Modifier:
		\begin{description}
			\item[\texttt{final}] Sorgt dafür, dass die Variable nur einmal zugewiesen werden kann (zum Beispiel direkt nach oder noch während der Deklaration). Wenn möglich sollte eine Variable immer als \lstinline|final| markiert werden, um versehentliches Überschreiben des Wertes zu verhindern.
		\end{description}
		Handelt es sich bei der Variablen um eine Instanz- oder Klassenvariable, sind zusätzlich folgende Modifier verfügbar:
		\begin{description}
			\item[\texttt{volatile}] Bei der Zuweisung der Variablen geschieht die Zuweisung \textit{atomar}. Dieser Modifier kann nicht mit \lstinline|final| modifiziert werden.
			\item[\texttt{transient}] Bei der Serialisierung einer Instanzvariablen wird dieses Feld nicht serialisiert.
			\item[\(\bullet\)] Sämtliche Sichtbarkeitsmodifizierer (siehe \ref{sec:visibility}).
		\end{description}
		Alle Modifier können wir mit kleinen Einschränkungen beliebig kombinieren.
		
		Beispiel: Eine Definition einer privaten Klassenvariable \texttt{timestamp}, die atomar Zugewiesen werden soll und nicht mit serialisiert werden soll sieht so aus:
		\begin{figure}[H]
			\centering
			\lstinline|private static transient volatile long timestamp;|
		\end{figure}
	% end
	
	% TODO: Schreiben
	%\subsubsection{Lokale Variablen, Konstanten, Attribute, Arraykomponenten}
	%	\todo{Schreiben}
	%% end
	
	\subsubsection{Null- und Defaultwerte}
		Klassenvariablen, die nicht \lstinline|final| sind, werden bestimmte Default-Werte zugewiesen (sofern die Variable nicht während der Deklaration direkt zugewiesen wird):
		\begin{table}[H]
			\centering
			\begin{tabular}{l | l}
				\textbf{Typ} & \textbf{Default-Wert} \\ \hline
				\lstinline|byte| & \lstinline|0| \\
				\lstinline|short| & \lstinline|0| \\
				\lstinline|int| & \lstinline|0| \\
				\lstinline|long| & \lstinline|0| \\
				\lstinline|float| & \lstinline|0.0F| \\
				\lstinline|double| & \lstinline|0.0| \\
				\lstinline|boolean| & \lstinline|false| \\
				\lstinline|char| & \lstinline|'\000'| (Null-Byte) \\
				\lstinline|Object| und Unterklassen & \lstinline|null| \\
			\end{tabular}
			\caption{Java: Defaultwerte}
		\end{table}
	% end
% end

\subsection{Zuweisungen}
	\implements{Zuweisungen}{zuweisungen}{Java}
	
	Um eine Variable zuzuweisen, wird folgender Ausdruck verwendet:
	\begin{figure}[H]
		\centering
		\lstinline|<variable> = <ausdruck>;|
	\end{figure}
	Dabei ist der linke Teil \texttt{<variable>} der Name der Variablen, welcher der Wert des Ausdrucks \texttt{<ausdruck>} zugewiesen wird. Der Ausdruck kann dabei beliebig komplex sein.
	
	Wie können den Wert auch zeitgleich mit der Deklaration zuweisen, die Syntax ist dann wie folgt:
	\begin{figure}[H]
		\centering
		\lstinline|<modifier> <typ> <name> = <ausdruck>;|
	\end{figure}

	Eine Besonderheit ist hier, dass der Ausdruck einer normalen Zuweisung den Wert der Zuweisung zurück gibt (das heißt es gilt \texttt{(<variable> = <ausdruck>) == <ausdruck>}).
% end

\subsection{Methodenaufrufe}
	\implements{Methodenaufrufen}{methodenNutzung}{Java}
	
	Der allgemeine Ausdruck, um eine Methode in Java aufzurufen ist:
	\begin{figure}[H]
		\centering
		\lstinline|<objekt>.<methodenname>([parameter], [parameter], ...)|
	\end{figure}
	Der Methodenname muss immer gegeben sein, ebenso wie das Objekt (beziehungsweise bei einer statischen Methode die Klasse), welches/welche das Objekt enthält. Die Parameter müssen gegeben sein, wenn die aufgerufene Methode dies fordert, es gibt aber auch Methoden, die keine Parameter erfordern.
	
	Wir können die Rückgabe der Methode auch einer Variablen zuweisen, die Syntax ist dann wie folgt:
	\begin{figure}[H]
		\centering
		\lstinline|<variable> = <objekt>.<methodenname>([parameter], [parameter], ...)|
	\end{figure}
	Dies ist nur möglich, wenn die Methode einen Rückgabetyp hat, das heißt der Rückgabetyp nicht \lstinline|void| ist.
% end

\subsection{Operatoren}
	\implements{Operatoren}{operatoren}{Java}
	
	\subsubsection{Arithmetik-Operatoren}
		Es existieren die folgenden arithmetischen Operatoren, die allesamt alle primitiven und numerischen Datentypen (\lstinline|byte|, \lstinline|short|, \lstinline|int|, \lstinline|long|, \lstinline|float|, \lstinline|double|) annehmen:
		\begin{table}[H]
			\centering
			\begin{tabular}{c | l | l}
				\textbf{Operator} & \textbf{Syntax}                & \textbf{Beschreibung}                                    \\ \hline
				\texttt{++}       & \texttt{a++}, \texttt{++a}     & \texttt{a} wird um 1 \textit{inkrementiert}.             \\
				\texttt{-{}-}     & \texttt{a-{}-}, \texttt{a-{}-} & \texttt{a} wird um 1 \textit{dekrementiert}.             \\
				\texttt{*}        & \texttt{a * b}                 & \texttt{a} und \texttt{b} werden \textit{multipliziert}. \\
				\texttt{/}        & \texttt{a / b}                 & \texttt{a} wird durch \texttt{b} \textit{dividiert}.     \\
				\texttt{\%}       & \texttt{a \% b}                & Es wird \( \texttt{a} \textbf{ mod } \texttt{b} \) berechnet (d.h. \( \texttt{a} - \big\lfloor \frac{\texttt{a}}{\texttt{b}} \big\rfloor \texttt{b} \)) (\textit{Modulo}). \\
				\texttt{+}        & \texttt{a + b}                 & \texttt{a} und \texttt{b} werden \textit{addiert}.       \\
				\texttt{-}        & \texttt{a - b}                 & \texttt{b} wird von \texttt{a} \textit{subtrahiert}.     \\
				\texttt{-}        & \texttt{-a}                    & Negiert das Vorzeichen von \texttt{a}.
			\end{tabular}
		\end{table}
		Bei den Inkrementierungs-/Dekrementierungs-Operatoren ist der Unterschied zwischen den Syntaxen \texttt{a++} und \texttt{++a} (beziehungsweise \texttt{a-{}-} und \texttt{-{}-a}), dass das Ergebnis von ersterem Ausdruck den Wert von \texttt{a} vor der Inkrementierung/Dekrementierung und \texttt{++a}/\texttt{-{}-a} den Wert nach der Inkrementierung/Dekrementierung als Ergebnis liefert (Postfix vs. Prefix Operatoren). Das bedeutet, dass \texttt{a++ == a}, \texttt{a-{}- == a}, \texttt{++a == a + 1} und \texttt{-{}-a == a - 1} gelten.
		
		\paragraph{Kommazahlen und Division}
			Eine Division wird immer als \textit{Ganzzahldivision} durchgeführt, wenn nicht mindestens einer der Parameter eine Fließkommazahl ist. Das bedeutet, dass Nachkommastellen nur berechnet werden, wenn mindestens einer der Parameter ein \lstinline|float| oder \lstinline|double| ist.
			
			Eine Ganzzahldivision von \(a\) und \(b\) entspricht \( \big\lfloor \frac{a}{b} \big\rfloor \), dass heißt, die Nachkommastellen werden abgeschnitten.
		% end
	% end
	
	\subsubsection{Logik- und Vergleichs-Operatoren}
		Es existieren die folgenden logischen Operatoren und Vergleichsoperatoren, die alle als Ergebnis ein \lstinline|boolean| zurück geben.
		\begin{table}[H]
			\centering
			\begin{tabular}{c | l | l | l}
				\textbf{Operator} & \textbf{Syntax}   & \textbf{Parametertyp} & \textbf{Beschreibung}                                         \\ \hline
				   \texttt{<}     & \texttt{a < b}    & primitive Zahl        & Ist \texttt{a} kleiner \texttt{b}?                            \\
				   \texttt{>}     & \texttt{a > b}    & primitive Zahl        & Ist \texttt{a} größer \texttt{b}?                             \\
				   \texttt{<=}    & \texttt{a <= b}   & primitive Zahl        & Ist \texttt{a} kleiner-gleich \texttt{b}?                     \\
				   \texttt{>=}    & \texttt{a >= b}   & primitive Zahl        & Ist \texttt{a} größer-gleich \texttt{b}?                      \\
				   \texttt{==}    & \texttt{a == b}   & Beliebig              & Ist \texttt{a} identisch zu \texttt{b}?                       \\
				   \texttt{!=}    & \texttt{a != b}   & Beliebig              & Ist \texttt{a} nicht identisch zu \texttt{b}?                 \\
				  \texttt{\&\&}   & \texttt{a \&\& b} & Wahrheitswert         & Verknüpft \texttt{a} und \texttt{b} mit einem logischem UND.  \\
				   \texttt{||}    & \texttt{a || b}   & Wahrheitswert         & Verknüpft \texttt{a} und \texttt{b} mit einem logischem ODER. \\
				   \texttt{\^}    & \texttt{a \^{} b} & Wahrheitswert         & Verknüpft \texttt{a} und \texttt{b} mit einem logischem XOR.  \\
				   \texttt{!}     & \texttt{!a}       & Wahrheitswert         & Negiert den Wahrheitswert von \texttt{a}
			\end{tabular}
		\end{table}
		\textit{Identisch} bedeutet für Zahlen, dass diese bis auf die letzte Nachkommastelle gleich sind. Für Objekte bedeutet dies, dass es ein und das selbe Objekt sind (das heißt, dass die Speicheradresse identisch ist). Eine Änderung auf \texttt{a} ändert somit auch \texttt{b}, wenn \texttt{a == b} gilt (nur bei Objekten!). Aufgrund dessen ist es auch nicht möglich, Strings mit \texttt{==} zu vergleichen, da dies bei Benutzereingaben oder ähnlichem immer \lstinline|false| liefern würde, da die Objekte nur den gleichen Inhalt haben und nicht identisch sind (siehe auch \ref{sec:equals_identity}).
	% end
	
	\subsubsection{Bitlogik-Operatoren}
		Die bitlogischen Operatoren können auf primitive Ganzzahlen (\lstinline|byte|, \lstinline|short|, \lstinline|int|, \lstinline|long|) angewendet werden. Diese wenden die üblichen logischen Verknüpfungen auf Bit-Ebene an, dass heißt, die Zahl wird in Binärdarstellung überführt und die Verknüpfung der Reihe nach auf jedes Bit einzeln angewendet (bei ungleich großen Datentypen werden die fehlenden Stellen bei dem kleineren mit Nullen aufgefüllt). Der Rückgabetyp entspricht immer dem größeren Datentyp. Es existieren die folgenden Operatoren:
		\begin{table}[H]
			\centering
			\begin{tabular}{c | l | l}
				\textbf{Operator} & \textbf{Syntax}      & \textbf{Beschreibung}                                                                               \\ \hline
				  \texttt{<{}<}   & \texttt{a <{}< b}    & Verschiebt die Bits von \texttt{a} um \texttt{b} Stellen nach links.                                \\
				  \texttt{>{}>}   & \texttt{a >{}> b}    & Verschiebt die Bits von \texttt{a} um \texttt{b} Stellen nach rechts.                               \\
				\texttt{>{}>{}>}  & \texttt{a >{}>{}> b} & Verschiebt die Bits von \texttt{a} um \texttt{b} Stellen nach rechts und behält das Vorzeichen bei. \\
				   \texttt{\&}    & \texttt{a \& b}      & Verknüpft die Bits von \texttt{a} und \texttt{b} mit einer UND-Verknüpfung.                         \\
				   \texttt{\^}    & \texttt{a \^{} b}    & Verknüpft die Bits von \texttt{a} und \texttt{b} mit einer XOR-Verknüpfung.                         \\
				   \texttt{|}     & \texttt{a | b}       & Verknüpft die Bits von \texttt{a} und \texttt{b} mit einer ODER-Verknüpfung.                        \\
				\texttt{\(\sim\)} & \texttt{\(\sim\)a}   & Negiert die Bits von \texttt{a}.
			\end{tabular}
		\end{table}
	% end
	
	\subsubsection{Spezielle Operatoren}
		Zusätzlich zu den oben genannten Operatoren gibt es noch die Operatoren \lstinline|new|, \lstinline|instanceof| und der Ternäre Operator, die etwas anders funktionieren.
		\begin{description}
			\item[\texttt{\color{lstkeywords} new}] Mit diesem Operator können neue Instanzen (Objekte) einer Klasse erstellt werden und die allgemeine Syntax lautet \lstinline|new <klasse>([parameter], [parameter], ...)|; diesen Operator werden wird im Abschnitt \ref{sec:constructor} genauer betrachten.
			\item[\texttt{\color{lstkeywords} instanceof}] Mit diesem Operator kann geprüft werden, ob ein Objekt eine Instanz einer bestimmten Klasse darstellt, die allgemeine Syntax hierfür lautet \lstinline|<objekt> instanceof <klasse>|. Beispielsweise wäre für eine Variable \lstinline|Number x = 1.2| der Ausdruck \lstinline|x instanceof Double| wahr, der Ausdruck \lstinline|x instanceof Integer| jedoch falsch.
			\item[Ternärer Operator] Mit diesem Operator können, ähnlich wie bei einem If, Fallunterscheidungen vorgenommen werden. Die allgemeine Syntax lautet \lstinline|<test> ? <wahr-fall> : <sonst-fall>|. Dabei wird zuerst der Test ausgewertet, ist dieser Wahr, so wird das Ergebnis von dem wahr-Fall zurück gegeben, sonst das Ergebnis von dem sonst-Fall. Dabei muss der Test zu einem Wahrheitswert auswerten und die beiden Fälle zu dem gleichen Typ, beziehungsweise einem kompatiblen Typ für den äußeren Ausdruck.
		\end{description}
	% end
	
	\subsubsection{Bindungsstärke der Operatoren}
		Die Bindungsstärke der Operatoren in Java gliedert sich wie folgt, wobei die oberste Zeile die stärkste Bindungsstärke hat und mehrere Elemente auf einer Zeile die gleiche Bindungsstärke:
		\begin{enumerate}
			\item \texttt{expr++}, \texttt{expr--}
			\item \texttt{++expr}, \texttt{--expr}, \texttt{+expr}, \texttt{-expr}, \texttt{\(\sim\)}, \texttt{!}
			\item \texttt{*}, \texttt{/}, \texttt{\%}
			\item \texttt{+}, \texttt{-}
			\item \texttt{<{}<}, \texttt{>{}>}, \texttt{>{}>{}>}
			\item \texttt{<}, \texttt{>}, \texttt{<=}, \texttt{>=}, \texttt{instanceof}
			\item \texttt{==}, \texttt{!=}
			\item \texttt{\&}
			\item \texttt{\^}
			\item \texttt{|}
			\item \texttt{\&\&}
			\item \texttt{||}
			\item \texttt{? :}
			\item \texttt{=}, \texttt{+=}, \texttt{-=}, \texttt{*=}, \texttt{/=}, \texttt{\%=}, \texttt{\&=}, \texttt{\^{}=}, \texttt{|=}, \texttt{<{}<=}, \texttt{>{}>=}, \texttt{>{}>{}>=}
		\end{enumerate}
	% end
	
	\subsubsection{Klammerung}
		Um die Bindungsstärke von Operatoren zu beeinflussen, können Ausdrücke wie in der Mathematik geklammert werden, wobei die innerste Klammer immer zuerst ausgewertet wird. Hierfür dürfen ausschließlich runde Klammern (\texttt{(}, \texttt{)}) genutzt werden.
	% end
% end

\subsection{Implizite und Explizite Typenkonversion (Casts)}
	Schauen wir uns zuerst einmal an, was wir unter einer Typenkonversion verstehen: Wenn wir eine Variable \lstinline|int a = 41| haben, können wir diese Problemlos einer anderen Variable mit dem Datentyp \lstinline|long| zuweisen (\lstinline|long b = a|). Hier liegt uns eine \textit{implizite Typenkonversion} vor, bei der der Datentyp \lstinline|int| zu einem \lstinline|long| umgewandelt wird. Wir gehen nun getrennt auf primitive Typenkonversionen, Wrapper-Typen und Objektkonversionen ein.
	
	\subsubsection{Primitive Typen}
		Eine primitive Typenkonversion haben wir bereits gesehen. Eine implizite Typenkonversion ist immer dann möglich, wenn der neue Datentyp eine größere oder gleiche Datenmenge halten kann wie der alte Datentyp (das heißt es ist zum Beispiel nicht implizit möglich, eine Fließkommazahl in eine Ganzzahl zu konvertieren).
		
		\begin{figure}[H]
			\centering
			\begin{tikzpicture}[main/.style = { draw, rectangle, minimum height = 0.9cm, minimum width = 2cm }]
				\node [main] (byte) {\lstinline|byte|};
				\node [main, right = 2 of byte] (short) {\lstinline|short|};
				\node [main, right = 2 of short] (int) {\lstinline|int|};
				\node [main, right = 2 of int] (long) {\lstinline|long|};
				\node [main, below = 2 of int] (float) {\lstinline|float|};
				\node [main, right = 2 of float] (double) {\lstinline|double|};
				\node [main, above = 2 of short] (char) {\lstinline|char|};
				
				\draw [->] (char) -| (int);
				
				\draw [->] (byte) -- (short);
				\draw [->] (short) -- (int);
				\draw [->] (int) -- (long);
			
				\coordinate [below = 1 of long] (needle);
				\draw (long) -- (needle);
				\draw [->] (needle) -| (float);
				
				\draw [->] (float) -- (double);
			\end{tikzpicture}
		\end{figure}
		Der Pfeil \( A \rightarrow B \) bedeutet, dass \(A\) implizit in \(B\) konvertiert werden kann. Der Rückweg ist ausgeschlossen. Außerdem ist die Konvertierung transitiv, dass bedeutet, wenn \( A \rightarrow B \) und \( B \rightarrow C \), dann geht auch \( A \rightarrow C \).
		
		Eine explizite Konvertierung wird vorgenommen, indem der neue Typ in Klammern vor die Variable (oder den Ausdruck) des alten Typs gesetzt wird:
		\begin{figure}[H]
			\centering
			\lstinline|(<neuer-typ>) <ausdruck>|
		\end{figure}
		Beispielsweise Wertet der Ausdruck \( 1 / 2.0 \) zu einem \lstinline|double| aus und das Ergebnis muss explizit in ein \lstinline|int| konvertiert werden: \lstinline|(int) (1 / 2.0)|. Das Ergebnis wäre in diesem Falle \lstinline|0|, da bei einer Typenkonvertierung von einer Fließkommazahl in eine Ganzzahl die Nachkommastellen abgeschnitten werden.
	% end
	
	\subsubsection{Wrappertypen}
		Wie wir im Abschnitt zu Generics (\ref{sec:generics}) sehen werden, sind primitive Typen nicht immer hilfreich. Manchmal möchten wir auch Zahlen oder ähnliches in Objekten speichern können. Hier kommen die sogenannten \textit{Wrappertypen} ins Spiel, die ebenso wir Strings immutable, das heißt nicht veränderlich, sind.
		
		Wrappertypen sind Klassen, die eine primitive Variable speichern und diese bei Bedarf zur Verfügung stellt. Die Verwendung dieser Wrapper Typen erfolgt durch \textit{Autoboxing} transparent, das heißt, eine Variable wird automatisch in einem Wrappertyp gespeichert und gelesen.
		
		Die Namen der Wrappertypen entsprechen zu großen Teilen dem Namen des primitiven Typs mit einem großem Anfangsbuchstaben (die Klassen liegen allesamt in dem Package \lstinline|java.lang|):
		\begin{table}[H]
			\centering
			\begin{tabular}{l | l}
				\textbf{Primitiver Typ} & \textbf{Wrappertyp}   \\ \hline
				\lstinline|byte|        & \lstinline|Byte|      \\
				\lstinline|short|       & \lstinline|Short|     \\
				\lstinline|int|         & \lstinline|Integer|   \\
				\lstinline|long|        & \lstinline|Long|      \\
				\lstinline|float|       & \lstinline|Float|     \\
				\lstinline|double|      & \lstinline|Double|    \\
				\lstinline|char|        & \lstinline|Character| \\
				\lstinline|boolean|     & \lstinline|Boolean|
			\end{tabular}
		\end{table}
	
		\paragraph{Autoboxing}
			Weisen wir einer Variable \lstinline|Object obj| einen primitiven Wert (zum Beispiel \lstinline|1.2|) zu, so wird dieser primitive Typ automatisch in den entsprechenden Wrappertyp konvertiert und der Variable zugewiesen. Ebenfalls wird an Stellen, an denen primitive Typen gebraucht werden (zum Beispiel in arithmetischen Operationen oder Vergleichen) der Wrappertyp zurück in einen primitiven Wert gewandelt.
			
			Beispiel:
			\begin{figure}[H]
				\centering
				\begin{lstlisting}
double primitive = 1.2;
int wholeNumber = (int) x;
Double wrapper = primitive;   // Autoboxing.
if (wrapper > wholeNumber) {  // Autounboxing.
	...
}
\end{lstlisting}
			\end{figure}
		
			\warning{Im Gegensatz zu primitiven Typen können Variablen von Wrappertypen \lstinline|null| sein. Wird versucht, Autounboxing auf \lstinline|null|-Werten anzuwenden, so wird eine \lstinline|NullPointerException| geworfen.}
		% end
	% end
	
	\subsubsection{Objekte (\enquote{Downcast})}
		\label{sec:downcast}
	
		Auch bei Objekten müssen wir manchmal eine Typenkonvertierung vornehmen. Implizite Typenkonvertierungen sind hier genau dann möglich, wenn der neue Typ eine Oberklasse des alten Typs ist. Eine explizite Typenkonvertierung wird benötigt, wenn in der Klassenhierarchie \enquote{nach unten} gegangen werden soll (dies wird \textit{Downcast} genannt). Eine explizite Typenkonvertierung findet wie bei primitiven Typen statt indem der neue Typ in Klammern vor den Ausdruck geschrieben wird.
		
		Schauen wir uns dies am Beispiel eines Strings an:
		\begin{figure}[H]
			\centering
			\begin{tikzpicture}
				\umlemptyclass{Object}
				\umlemptyclass[below = 1 of Object]{CharSequence}
				\umlemptyclass[below = 1 of CharSequence]{String}
				
				\umlinherit{String}{CharSequence}
				\umlinherit{CharSequence}{Object}
			\end{tikzpicture}
		\end{figure}
		Eine implizite Typenkonvertierung ist nun immer nach oben in der Hierarchie möglich (also \( \texttt{String} \rightarrow \texttt{CharSequence} \rightarrow \texttt{Object} \)).
		
		Beispiel:
		\begin{figure}[H]
			\centering
			\begin{lstlisting}
String s = "Hello, World!";
Object o = s;                // Implizite Typenkonvertierung.
String casted = (String) o;  // Explizite Typenkonvertierung (Downcast).
\end{lstlisting}
		\end{figure}
	% end
% end

% TODO: Schreiben
%\subsection{Links-/Rechtsausdrücke}
%	\todo{Schreiben}
%% end

% TODO: Schreiben
%\subsection{Seiteneffekte}
%	\todo{Schreiben}
%% end

% end

\section{Kontrollstrukturen}
	\introduces{von Kontrollstrukturen}{kontrollstrukturen}

Bisher haben wir nur Konzepte betrachtet, die es uns ermöglichen, einen linearen und immer gleichen Ablauf des Programms zu bewerkstelligen. Aber spätestens, wenn unser Programm an irgendeiner Stelle \enquote{denken} (wir betrachten hier keine Konzepte des Machine Learnings o.ä.) soll, müssen wir anfangen, darüber nachzudenken, wie wir Entscheidungen implementieren.

Hier kommen Kontrollstrukturen ins Spiel, welche uns erlauben
\begin{enumerate}[label = \alph*)]
	\item Entscheidungen zu treffen und
	\item Codeblöcke zu wiederholen.
\end{enumerate}

\subsection{Verzweigungen} \functionalMark \imperativeMark \oopMark
	\introduces{von Verzweigungen}{verzweigungen}

	Wir beschäftigen uns nun mit Teil a) von obiger Liste: Verzweigungen, das wichtigste überhaupt, wenn es darum geht, in einem Programm Entscheidungen zu treffen. Wir werden nun den Haupttyp Typen von Verzweigungen kennen lernen: ein \textit{if}.

	\paragraph{If}
		Ein \textit{if} ist eine einfache Verzweigung der Form \enquote{Wenn \dots gilt, dann tue \dots. Sonst tue \dots.}. In den meisten Programmiersprachen wird dies gesprochen als \enquote{\textbf{if} \dots \textbf{then} \dots \textbf{else} \dots}, was einer einfachen Übersetzung des deutschen \enquote{wenn, dann, ansonsten} entspricht. Der sogenannte \textit{else-Block} kann bei den meisten Sprachen auch fallengelassen werden, wird die Bedingung dann zu \textit{Falsch} ausgewertet, so geschieht einfach nichts.
		
		In vielen Fällen muss man allerdings mehr als einen Fall betrachten, wodurch sich \textit{elseif-Blöcke} ergeben, die ungefähr die Form \enquote{\textbf{if} \dots \textbf{then} \dots \textbf{elseif} \dots \textbf{then} \dots \textbf{else} \dots} haben. Umgangssprachlich kann man ein else-if also als \enquote{wenn, dann, ansonsten wenn, $ \cdots $, ansonsten} ausdrücken. Es kann in den meisten Fällen beliebig viele else-if-Blöcke geben.
	% end
% end

\subsection{Schleifen} \imperativeMark \oopMark
	\introduces{von Schleifen}{schleifen}

	Nun beschäftigen wir uns mit Teil b) aus obiger Liste: Schleifen. Spätestens, wenn wir unseren Code mehrmals Ausführen wollen und ihn zu diesem Zweck nicht einfach untereinander kopieren können (beispielsweise wenn die Ausführung von einem Parameter abhängt, dessen Wert wir während dem Schreiben noch nicht kennen), müssen wir unseren Code dynamisch beliebig oft ausführen. Dies ist zum Beispiel der Fall, wenn wir die Fakultät einer Zahl berechnen wollen: \[ n! \coloneqq \prod _ { i = 1 } ^ n i = 1 \cdot 2 \cdot\,\cdots\,\cdot (n - 1) \cdot n \]
	
	Als Grundlage für alle Schleifen dient die \textit{while-Schleife}, bei der ein Codeblock so lange ausgeführt wird, bis eine bestimmte Bedingung zu \textit{Falsch} auswertet. Der Code kann somit als \enquote{Solange \dots tue \dots} verstanden werden und sieht in den meisten Sprachen auch ähnlich aus: \enquote{\textbf{while} \dots \textbf{do}}. Als Anlehnung an die while-Schleife gibt es selten auch die until-Schleife, die genau entgegengesetzt funktioniert: Der Codeblock wird so lange ausgeführt, bis eine bestimmte Bedingung zu \textit{Wahr} auswertet.
	
	Damit können wir nun das obige Problem wie folgt in unserer imaginären Sprache lösen, wobei wir hier in der Variable \texttt{n} das \( n \) von oben speichern und in der Variable \texttt{x} das Ergebnis (es soll also nach der Ausführung $ \texttt{x} = n! $ gelten).
	\begin{figure}[H]
		\centering
		\begin{lstlisting}
x = 1
while n > 0
	x = x * n

	n = n - 1
done
		\end{lstlisting}
		\caption{Beispiel: While-Schleife}
	\end{figure}
	
	Der Code \texttt{x = x * n}, \texttt{n = n - 1} wird nun immer ausgeführt, solange \( \texttt{n} > 0 \) gilt. Eine Ausführung des Blocks wird \enquote{Schleifendurchlauf} oder \enquote{Iteration} genannt.
	
	Da es manche Fälle gibt, in denen die gesamte Ausführung innerhalb der Schleife abgebrochen werden soll oder die aktuelle Iteration abgebrochen werden soll und mit der nächsten begonnen werden soll, gibt es meist noch die Ausdrücke \enquote{break} und \enquote{continue}, welche ihrem Namen treu bleiben und die folgenden Funktionen erfüllen:
	\begin{description}
		\item[break] Hält die gesamt Schleifenausführung an und fährt mit der ersten Zeile nach der Schleife fort.
		\item[continue] Hält den aktuellen Schleifendurchlauf an und fährt mit der nächsten Iteration fort. Gibt es keine weitere Iteration, so wird die Schleife beendet.
	\end{description}
% end

% end

\section{Methoden}
	\implements{Methoden}{methoden}{Java}

\textit{In Java sind Methoden immer an ein Objekt oder eine Klasse gebunden. Die Unterschiede hierzwischen werden wir uns später im Abschnitt \refImpl{oop}{Java} zu objektorientierter Programmierung in Java anschauen. In diesem Kapitel werden wir annehmen, dass sich alle Methoden in einer Klasse befinden, eine Instanz der Klasse vorliegt und die Methoden an diese Instanz gebunden sind.}

Betrachten wir zur Einführung die folgende Methode:
\begin{figure}[H]
	\centering
	\begin{lstlisting}
int add(int a, int b) {
	return a + b;
}
	\end{lstlisting}
\end{figure}
die die Summe der Zahlen \texttt{a} und \texttt{b} berechnet.

Dabei entspricht \textit{int add(int a, int b)} dem Kopf der Methode und alles in den geschweiften Klammern (also \texttt{return a + b;}) dem Körper der Methode.

\subsubsection{Der Methodenkopf}
	Ein Methodenkopf hat folgenden allgemeinen Aufbau:
	\begin{center}
		\texttt{[MODIFIER] [GENERICS] <RÜCKGABETYP> <METHODENNAME>([PARAMETER])}
	\end{center}
	dabei ist die Angabe von Modifizierern (\textit{Modifier}), Generics und Parametern optional, wobei beliebig viele Parameter angegeben werden können. Die Klammern hinter dem Methodennamen müssen dennoch vorhanden sein, auch wenn keine Parameter angegeben werden.
	
	\paragraph{Modifizierer}
		Die Modifizierer, die an einer Methode angegeben werden können, werden wir uns im Kapitel über objektorientierte Programmierung genauer anschauen, da die in Java vorhanden Modifizierer nur in diesem Kontext Sinn ergeben.
		
		\textbf{Erweitertes Wissen:} Eine Ausnahme stellt der Modifizierer \texttt{strictfp} dar, der der JVM aufträgt, arithmetische Operationen exakt wie in der Spezifikation der JVM vorzunehmen und nicht zu optimieren.
	% end
	
	\paragraph{Generics}
		Siehe \ref{sec:generics}.
	% end
	
	\paragraph{Rückgabetyp}
		Hiermit geben wir den Typ an, den das Ergebnis unserer Methode hat. Dies kann ein primitiver Datentyp oder eine Klasse sein. Wird nichts zurückgegeben, muss \texttt{void} angegeben werden, was so viel wie \enquote{nichts} heißt.
	% end
	
	\paragraph{Methodenname}
		Dies ist der Name der Methode, mit dem wir selbige referenzieren können. Der Name muss sich an die Grundregeln von Bezeichnern in Java halten (siehe \refImpl{identifier}{Java}).
	% end
	
	\paragraph{Parameter}
		Die ist eine Komma-separierte Liste von Parametern, die unsere Funktion erwartet.
		
		Einer dieser Parameter ist dabei aufgebaut wir eine normale Variablendeklaration, das heißt \texttt{[final] <DATENTYP> <NAME>}. Ein hier verwendeter Name kann im Körper der Methode nicht erneut für Variablen genutzt werden, der Zugriff auf den Wert des Parameters erfolgt, als wäre dieser eine ganz normale Variable. Ein in der Parameterliste angegebenes \texttt{final} verhält sich entsprechend.
		
		\subparagraph{Varargs}
			Varargs sind eine spezielle Form der Parameter, die dem Aufrufer erlauben, beliebig viele Parameter zu übergeben.
			
			Betrachten wir hierzu folgendes Beispiel, um beliebig viele Zahlen zu addieren:
			\begin{figure}[H]
				\centering
				\begin{lstlisting}
int add(int[] numbers) {
	int result = 0;
	for (int x : numbers) {
		result += x;
	}
	return result;
}
				\end{lstlisting}
			\end{figure}
			Ein Aufrufer müsste die Funktion zum Beispiel so aufrufen: \lstinline|add(new int[] { 1, 2, 3, 4, 5 })|, das heißt der müsste erst ein Array erstellen und dieses der Funktion übergeben. Dies stellt einen erheblichen Schreibaufwand dar.
			
			Hätten wir stattdessen unsere Funktion wie folgt mit Varargs gestaltet, vereinfacht sich der Aufruf, wie wir gleich sehen werden:
			\begin{figure}[H]
				\centering
				\begin{lstlisting}
int add(int... numbers) {
	int result = 0;
	for (int x : numbers) {
		result += x;
	}
	return result;
}
				\end{lstlisting}
			\end{figure}
			Nun vereinfacht sich der funktional identische Aufruf zu \lstinline|add(1, 2, 3, 4, 5)|.
			
			Wir sehen auch, dass sich an dem Körper unserer Funktion nichts geändert hat, einzig und allein die manuelle Erstellung des Arrays verschwindet. Konkret heißt dies, dass Java uns die Arbeit abnimmt, das Array manuell zu erstellen, sondern dies im Hintergrund erledigt. Wenn der Aufrufer unbedingt will, kann er dennoch ein einfaches Array übergeben.
			
			\warning{Es ist nicht möglich, nach einem Vararg-Parameter noch weitere Parameter anzugeben, da Java sonst nicht wüsste, welche Parameter noch zum Varargs gehören und welche nicht. Vor einem Vararg-Parameter ist dies problemlos möglich.}
		% end
	% end
% end

\subsubsection{Signatur}
	Die Signatur einer Methode muss innerhalb einer Klasse eindeutig sein. Zu der Signatur einer Methoden gehören
	\begin{itemize}
		\item der Methodenname und
		\item die Typen der Parameter.
	\end{itemize}

	Somit sind bei den folgenden Methoden:
	\begin{enumerate}
		\item \lstinline|int add(int a, int b)|
		\item \lstinline|float add(int a, int b)|
		\item \lstinline|float add(float a, float b)|
	\end{enumerate}
	die Methoden 1 und 2 der Signatur nach identisch, die 3. Methode hingegen verschieden. Somit dürften in einer Klasse nur folgende Kombinationen vorkommen:
	\begin{equation*}
		\emptyset \,,\quad \{ 1 \} \,,\quad \{ 1, 3 \} \,,\quad \{ 2 \} \,,\quad \{ 2, 3 \}
	\end{equation*}
% end

\subsubsection{Formale Parameter vs. Aktualparameter}
	Damit es bei Unterhaltungen über Methoden nicht zu Verwirrungen kommt, schauen wir uns noch die Begriffe \textit{Formale Parameter} und \textit{Aktualparameter} an:
	\begin{description}
		\item[Formale Parameter] Formale Parameter sind jene, welche bei der Definition einer Methode angegeben werden.
		\item[Aktualparameter] Aktualparameter sind die Parameter, welche einer Methode bei einem Aufruf übergeben werden.
	\end{description}

	\textbf{Beispiel:} \\
	\begin{lstlisting}
void doIt(int a, int b) { /* ... */ }

int g = 9.81;
doIt(1, g)
	\end{lstlisting}
	In diesem Beispiel sind \texttt{a} und \texttt{b} die formalen Parameter und \texttt{1} und \texttt{g} die Aktualparameter.
% end

\subsubsection{Veträge in Form von Javadoc}
	Verträge sind Teil der Dokumentation, siehe \refImpl{doku}{Java}.
% end

\subsubsection{Überladen}
	Wie wir bereits im Abschnitt über Signaturen gesehen haben, muss ausschließlich die Signatur einer Methode in einer Klasse eindeutig sein, nicht der Name der Methode.
	
	Wird ein Methodenname mehrmals in einer Klasse verwendet, so wird diese Methode \textit{überladen}. Dies ist sinnvoll, wenn eine Methode beispielsweise unterschiedliche Parametertypen annehmen kann, um mit ihnen zu arbeiten, ein jeweils eigener Methodenname aber keinen Sinn ergibt.
	
	\textbf{Beispiel:} \\
	Wir haben eine Klasse mit folgenden Methoden:
	\begin{itemize}
		\item \lstinline|String valueOf(int a)|
		\item \lstinline|String valueOf(float a)|
	\end{itemize}
	dann ist die Methode \texttt{valueOf(\dots)} \textit{überladen}. Bei einem Aufruf \lstinline|valueOf(42)| wird die erste, bei einem Aufruf \lstinline|valueOf(4.2F)| die zweite Methode verwendet.
	
	\paragraph{Problematiken}
		Manchmal kann der Compiler nicht entscheiden, welche der überladenen Methoden aufgerufen werden soll. In diesem Fall muss der Typ eines Parameters genauer spezifiziert werden, damit der Code kompiliert.
		
		\textbf{Beispiel:} \\
		Wir haben eine Klasse mit folgenden Methoden:
		\begin{itemize}
			\item \lstinline|String valueOf(Integer a)|
			\item \lstinline|String valueOf(Long a)|
		\end{itemize}
		Bei einem Aufruf \lstinline|valueOf(null)| könnten beide Methoden aufgerufen werden, der Typ passt bei beiden. Somit kann der Compiler den Code so nicht kompilieren und bricht ab. Zur Lösung müssen wir den Typ genauer spezifizieren, zum Beispiel durch einen Downcast: \lstinline|valueOf((Integer) null)|. Nun wird die erwartete Methode aufgerufen.
	% end
% end

\subsection{Rückgabe von Werten}
	Wie wir bereits gesehen haben, wird bei der Definition einer Methode ebenfalls ein Rückgabetyp definiert. Definieren wir einen Rückgabetyp, so müssen wir auch einen entsprechenden Wert zurück gegen. Dies wird mit dem Schlüsselwort \lstinline|return| vollführt. Die allgemeine Syntax ist hierbei \lstinline|return [AUSDRUCK];|, wobei der Ausdruck durch einen solchen ersetzt werden muss, der zu dem Rückgabetyp auswertet. Der Ausdruck \lstinline|return;| kann auch verwendet werden, wenn der Rückgabetyp \lstinline|void| ist. Hiermit kann die Besonderheit ausgenutzt werden, dass eine Methode sofort zurück geht, sobald ein Return ausgeführt wurde. Damit ist es zum Beispiel möglich, am Anfang der Methode einige Einschränkungen der Parameter zu prüfen und erst fortzufahren, sobald die Bedingungen erfüllt sind:
	\begin{figure}[H]
		\centering
		\begin{lstlisting}
void doIt(Integer a, Integer b) {
	if (a == null || b == null) {
	return;  // Vorzeitiges Return.
	}
	if (a < 0 || b < 0) {
		return;  // Vorzeitiges Return.
	}
	
	...
}
\end{lstlisting}
	\end{figure}
	
	\paragraph{Sonderfall \texttt{finally}}
		Wird ein Return innerhalb eines Try-Finally verwendet, so wird nach der Ausführung des Returns zuerst noch der \lstinline|finally|-Block ausgeführt und erst danach zurück zum Aufrufer gesprungen. Befindet sich innerhalb des Finallys wiederum ein Return, so wird der Wert des letzten Returns zurück gegeben.
		
		\textbf{Beispiel:}
		\begin{figure}[H]
			\centering
			\begin{lstlisting}
int add(int a, int b) {
	try {
		return a + b;
	} finally {
		return a - b;
	}
}
\end{lstlisting}
		\end{figure}
		Wird die Methode mit \lstinline|add(2, 1)| aufgerufen, so werden die Zeilen in der Reihenfolge \( (3, 5) \) ausgeführt und der Wert \lstinline|1| zurück gegeben.
	% end
% end

%\subsubsection{Abarbeitung von Methodenaufrufen}
%	\todo{Schreiben}
%% end

% end

\section{Scoping}
	\implements{Scoping und Scopes}{scoping}{Java}

In Java wird immer dann ein neuer Scope geöffnet, wenn eine geschweifte Klammer auf geht und ein Scope geschlossen, wenn die geschweifte Klammer zu geht. Das ist zum Beispiel bei Methoden und Schleifen der Fall.

Das bedeutet: Eine Variable, die wir innerhalb einer Methode definieren (inklusive der Parameter) ist von außerhalb nicht zugreifbar. Das gleiche gilt für Variablen, die innerhalb eines If-Blocks oder dem Körper einer Schleife definiert wurden.

\textbf{Beispiel:} \\
\begin{lstlisting}
int multiply(int a, int b) {
	int result = 0;

	int bAbs = Math.abs(b);
	for (int i = 0; i < Math.abs(b); i++) {
		result += a;
	}

	if (b < 0) {
		return -result;
	} else {
		return result;
	}
}
\end{lstlisting}
Auf die Variablen \texttt{a}, \texttt{b} und \texttt{result} kann von außerhalb der Methode nicht zugegriffen werden. Ebenfalls kann nicht auf \texttt{i} zugegriffen werden, dies ist aber schon außerhalb der Schleife (also in Zeile 1 bis 4 und 8 bis 14) nicht möglich.

Der Wert der Variable \texttt{result} wird zurück gegeben, womit der Aufrufer Zugriff auf den Wert der Variable hat, aber ohne Kenntnis darüber, wie das Ergebnis zustande gekommen ist.
% end

\section{Generische Programmierung}
	\implements{generischer Programmierung}{generischeProgrammierung}{Java}

Wir werden in diesem Kontext nur Typparameter (Generics) betrachten, welche in Java eine große Rolle spielen, wenn es um generische Programmierung geht.

\subsection{Generics}
	\label{sec:generics}

	In diesem Abschnitt betrachtet wir Generics, ein Konzept zur generischen Programmierung und Typsicherheit, welches mit der Java-Version 1.5 eingeführt wurde. Wie wir bereits wissen, ist Java eine statisch typisierte Programmiersprache, das heißt der korrekte Typ muss schon zur Compilezeit feststehen.
	
	\paragraph{Motivation}
		Betrachtet wir zur Motivation eine Klasse \texttt{List}, welche die folgenden Methoden aufweist:
		\begin{description}
			\item[\texttt{add(element: Object)}: void] Fügt ein Element zu der Liste hinzu.
			\item[\texttt{get(index: int)}: Object] Gibt das Element an der Position \texttt{index} zurück. Existiert kein solches Element, wird \texttt{null} zurück gegeben.
			\item[\texttt{size(): int}] Gibt die Anzahl der Element der Liste zurück.
		\end{description}
		Die Klasse selbst stellt eine Auflistung von Elementen dar und kann keine \texttt{null}-Elemente enthalten.
		
		Auffällig ist, dass sämtliche Element als Object-Referenz vorliegen. Es gibt für eine Methode, welche eine solche Liste annimmt, gibt es keine Möglichkeit, den Typ der Parameter einzuschränken. Schauen wir uns hierzu eine Methode an, welche alle Elemente aufsummieren soll. Natürlich ergibt eine solche Operation nur auf Zahlen Sinn, weshalb davon ausgegangen wird, dass sämtliche Elemente der Liste vom Typ \texttt{Integer} sind:
		\begin{figure}[H]
			\centering
			\begin{lstlisting}
public Integer sum(List list) {
	Integer result = 0;
	for (int i = 0; i < list.size(); i++) {
		result += @@!(Integer) list.get(i)@@@;
	}
	return result;
}
			\end{lstlisting}
			\caption{Java: Generics Motivation}
			\label{fig:generics_motivation}
		\end{figure}
		Hier tut sich folgendes Problem auf: Wenn der Aufrufer falsche Elemente in die Liste getan hat, fliegt uns der Algorithmus an der markierten Stelle um die Ohren: Es wird eine \texttt{ClassCastException}.
		
		Nun haben wir zwei Möglichkeiten:
		\begin{enumerate}
			\item Den korrekten Aufruf dem Aufrufer überlassen und annehmen, dass wir korrekte Daten bekommen.
			\item Die Korrektheit des Aufrufs vor der eigentlichen Ausführung überprüfen.
			\item Die Typprüfung des Compilers nutzen, damit wir erst gar keine falschen Typen bekommen können.
		\end{enumerate}
		Möglichkeit 1 mag bei solch kleinen Beispielen noch funktionieren, aber spätestens, wenn wir die Liste an andere Methoden weitergeben und nicht selbst für den Fehler sorgen, wird die Fehlersuche schrecklich. Ebenso ergeht es Möglichkeit 2, denn bei tief verschachtelten Klassenstrukturen bleibt das Prüfen der Korrektheit nicht so trivial wie in unserem Beispiel.
		
		\info{Generell gilt: Fail Fast, am besten noch zur Compilezeit. Dies vereinfacht die Fehlersuche erheblich. Fehler, die erst in den Tiefen eines Programmes passieren, sind schwer zu finden!}
		
		Somit bleibt uns als einzig gute Möglichkeit die Typprüfung des Compilers zu nutzen, welche auch genau dazu dient. Es wäre also eine Möglichkeit, eine Unterklasse \texttt{IntegerList} von \texttt{List} zu erstellen, die nur \texttt{Integer}s akzeptiert. Das dies eine schlechte Idee ist, würden wir spätestens nach der zehnten, bis auf den Typ identischen, Klasse merken. Wäre es nicht toll, alles nur einmal schreiben zu müssen und den Typ nachher setzen zu können? Hier helfen uns die zuvor genannten Generics: Diese ermöglichen genau das.
		
		Klassen, welche Generics nutzen, sind für viele Typen geschrieben und ermöglichen die Angabe des expliziten Typs in Spitzen Klammern hinter dem Klassennamen. Des weiteren ist es auch auf Methodenebene möglich, für viele Typen generisch zu Programmieren und bei der Nutzung der Methode den eigentlichen Typ festzulegen. Wir werden uns diese beiden Fälle in den folgenden beiden Abschnitten zu generischen Klassen und Methoden anschauen. Um einen ersten Eindruck zu kriegen, was uns Generics ermöglichen, hier ein Beispiel, welche die obige Methode ersetzen kann und eine größere Typsicherheit ermöglicht:
		\begin{figure}[H]
			\centering
			\begin{lstlisting}
public Integer sum(List@@?<Integer>@@@ list) {
	Integer result = 0;
	for (int i = 0; i < list.size(); i++) {
		result += @@?list.get(i)@@@;
	}
	return result;
}
			\end{lstlisting}
			\caption{Java: Generics Motivation: Nutzung von Generics}
			\label{fig:java_generics_motivation_gen}
		\end{figure}
		
		Wir haben hier nun die Typen in den Spitzen Klammern hinter \texttt{List} festgelegt, wodurch uns die Methode \texttt{get(\dots)} direkt Elemente des Typs \texttt{Integer} gibt. Wir können somit einfach die Zahlen aufsummieren. Somit wird ein möglicher Aufrufer nun schon beim Kompilieren merken, dass der Aufruf fehlschlägt. Da wir \texttt{null}-Elemente ausgeschlossen haben, wird der Code nun immer funktionieren, sofern dieser erfolgreich Kompiliert. Weshalb wir hier \texttt{Integer} und nicht \texttt{int} verwenden, werden wir im Abschnitt \ref{sec:generics_primitive_typen} betrachten.
	% end

	\subsubsection{Generische Klassen}
		Da wir in dem vorherigen Abschnitt gesehen haben, wofür Generics an sich gut sind, werden wir uns in diesem Abschnitt anschauen, wie wir selber generische Klassen erstellen können und uns anschauen, wie wir generische Klassen nutzen können.
		
		Wir schauen uns hierzu eine mögliche Implementierung der oben beschriebenen Liste an, welche noch keine Generics nutzt (die Implementierung leitet einfach alle Aufrufe an eine andere, nicht von uns implementierte List weiter):
		\begin{figure}[H]
			\centering
			\begin{lstlisting}
public class List {
	private DelegateList delegateList = new DelegateList();

	public void add(Object element) {
		delegateList.add(element);
	}

	public Object get(int index) {
		return delegateList.get(index);
	}

	public int size() {
		return delegateList.size();
	}
}
			\end{lstlisting}
			\caption{Java: Implementation von \texttt{List}}
		\end{figure}
		
		Diese Klasse können wir nur nutzen, wie es in Codebeispiel \ref{fig:generics_motivation} gelistet ist. Um die Klasse nun generisch Nutzbar zu machen, müssen wir sogenannte \textit{Typparameter} einführen. Diese werden in spitzen Klammern (\texttt{<}, \texttt{>}) hinter den Klassennamen geschrieben (ähnlich wie bei der Nutzung, nur werden die keine expliziten Klassen genannt, sondern Platzhalter verwendet). Als Name für Typparameter wird meist ein einzelner großer Buchstabe verwendet, um auf den ersten Blick ersichtlich zu machen, dass es sich um einen generischen Typ und nicht um eine existierende Klasse handelt.
		
		Die Klasse sieht, unter Nutzung von Typparametern, nun folgendermaßen aus (der Einfachheit halber nehmen wir an, die Klasse \texttt{DelegateList} sei auch generisch):
		\begin{figure}[H]
			\centering
			\begin{lstlisting}
public class List@@?<T>@@@ {
	private DelegateList@@?<T>@@@ delegateList = new DelegateList@@?<T>@@@();
	
	public void add(@@?T@@@ element) {
		delegateList.add(element);
	}
	
	public @@?T@@@ get(int index) {
		return delegateList.get(index);
	}
	
	public int size() {
		return delegateList.size();
	}
}
			\end{lstlisting}
			\caption{Java: Implementation von \texttt{List<T>}}
		\end{figure}
		
		Die Klasse können wir nun deutlich einfacher nutzen, wie es bereits in Codebeispiel \ref{fig:java_generics_motivation_gen} vorgestellt wurden. Es ist nun möglich, der Klasse beliebige Typen zu übergeben. Wie wir dies weiter Einschränken können, werden wir im Abschnitt \ref{sec:generics_restriction} über die Einschränkung der Typparameter behandeln.
		
		Den Typparameter \texttt{T} können wir nun überall in der Klasse verwenden, Instanzen des Typs verhalten sich wie Instanzen von \texttt{Object}, das heißt es können annähernd keine Operationen auf den Instanzen durchgeführt werden.
		
		\warning{Die Typparameter einer Klasse können nicht in dem statischen Kontext einer Klasse verwendet werden! Das heißt, die Typparameter können nicht im \texttt{static}-Block oder in statischen Methoden verwendet werden.}
		
		\info{Typische Namen für die Typparameter sind:
			\begin{description}
				\item[\texttt{T}] für beliebige Typen.
				\item[\texttt{K}] für Typen, die einen Key darstellen.
				\item[\texttt{V}] für Typen, die einen Wert (Value) darstellen.
			\end{description}
		}
	% end
	
	\subsubsection{Generische Methoden}
		In vielen Fällen ist es nicht nötig, die gesamte Klasse zu Parametrisieren, sondern es ist möglich oder auch nötig, nur einzelne Methoden generisch zu Halten. Ein gutes Beispiel hierfür sind Methoden in Utility-Klassen, bei denen gar keine Instanz der Klasse erstellt wird und damit die Klassentypparameter nicht nutzbar sind.
		
		Betrachten wir hierzu eine Methode, welche die Textrepräsentation eines jeden Objektes in einer Liste aneinanderhängt.
		
		Wir betrachten zuerst die nicht-generische Methode, welche nur Strings unterstützt:
		\begin{figure}[H]
			\centering
			\begin{lstlisting}
public static String concatenate(List<String> list) {
	String result = "";
	for (int i = 0; i < list.size(); i++) {
		result += list.get(i);
	}
	return result;
}
			\end{lstlisting}
			\caption{Java: Generics an Methoden: Motivation}
		\end{figure}
		
		Nun sollen auch andere Typen genutzt werden können, weshalb wir hier Typparameter einführen. Der Typparameter wird wie vorher auch in Spitze klammern gefasst und direkt vor dem Rückgabetyp plaziert. Hierdurch ist es auch möglich, den Typparameter bereits im Rückgabetyp zu verwenden, wie wir später sehen werden. Des weiteren gilt wie bei Klassen, dass für Typparameter ein einzelner, großer Buchstabe verwendet werden sollte.
		
		Schauen wir uns nun also die obige Methode unter Verwendung von Typparametern an:
		\begin{figure}[H]
			\centering
			\begin{lstlisting}
public static @@?<T>@@@ String concatenate(List<@@?T@@@> list) {
	String result = "";
	for (int i = 0; i < list.size(); i++) {
		// Da wir nun nicht mehr direkt auf Strings arbeiten, muessen wir uns erst die
		// Textrepraesentation der Elemente holen.
		result += list.get(i)@@?.toString()@@@;
	}
	return result;
}
			\end{lstlisting}
			\caption{Java: Generics an Methoden}
			\label{fig:generics_methoden}
		\end{figure}
		
		Nun ist es also möglich, die obige Methode mit jedem Typ aufzurufen.
	% end
	
	\subsubsection{Primitive Typen und Generics}
		\label{sec:generics_primitive_typen}
	
		Da Generics nur Klassentypen entgegen nehmen können, ist es somit nicht möglich, primitive Datentypen in Generics zu verwenden (also \texttt{int}, \texttt{float}, \dots). Zur Abhilfe dieses Problems gibt es sogenannte Wrapper-Klassen, welche den Wert einer primitiven Variable halten und mit Generics genutzt werden können. Namentlich sind diese:
		\begin{table}[H]
			\centering
			\begin{tabular}{l | l}
				Wrapper-Klasse     & Primitiver Typ   \\ \hline
				\texttt{Byte}      & \texttt{byte}    \\
				\texttt{Short}     & \texttt{short}   \\
				\texttt{Integer}   & \texttt{int}     \\
				\texttt{Long}      & \texttt{long}    \\
				\texttt{Float}     & \texttt{float}   \\
				\texttt{Double}    & \texttt{double}  \\
				\texttt{Character} & \texttt{char}    \\
				\texttt{Boolean}   & \texttt{boolean}
			\end{tabular}
			\caption{Java: Wrapper-Klassen (im Package \texttt{java.lang})}
		\end{table}
		Alle Wrapper-Klassen haben unter anderem die folgenden Methoden:
		\begin{description}
			\item[\texttt{valueOf(\dots)}] Erstellt eine Instanz der Wrapper-Klasse mit dem übergebenen primitiven Wert.
			\item[\texttt{xxxValue()}]     Gibt den gespeicherten primitiven Wert zurück. \enquote{\texttt{xxx}} muss dabei durch den konkreten primitiven Typ ersetzt werden.
		\end{description}
		
		Schauen wir uns unter diesen Voraussetzungen folgenden Code an, der den Durchschnitt zweier benachbarten Zahlen (gleitender Durchschnitt) berechnet:
		\begin{figure}[H]
			\centering
			\begin{lstlisting}
public List<Double> floatingAverage(List<Integer> list) {
	List<Double> result = new List<Double>();
	for (int i = 1; i < list.size(); i++) {
		int previous = list.get(i - 1)@@?.intValue()@@@;
		int current = list.get(i)@@?.intValue()@@@;
		double average = (previous + current) / 2.0;
		result.add(@@?Double.valueOf(@@@average@@?)@@@);
	}
	return result;
}
			\end{lstlisting}
			\caption{Java: Generics und primitive Datentypen}
			\label{fig:java_generics_motivation_gen}
		\end{figure}
		
		Wir sehen, dass sich ein extremer Overhead an Code ergibt, welcher nur zum Konvertieren der Wrapper-Klassen in primitive dient. Da dies sehr nervig sein kann, wurde in Java 5 das sogenannte Autoboxing eingeführt, welches wir nun betrachten.
	
		\paragraph{Autoboxing}
			Der Name \enquote{Autoboxing} bezeichnet das automatische Konvertieren von primitiven in Wrapper-Klassen und anders herum (\enquote{Unboxing}). Der primitive Wert wird in eine Wrapper-Klasse \enquote{geboxed}.
			
			Die Konvertierung findet genau dann statt, wenn es benötigt wird. Führen wie also eine Rechnung mit Wrapper-Klassen aus, so wird im Hintergrund der primitive Wert geunboxed und mit diesem gerechnet. Weisen wir das Ergebnis einer Wrapper-Klasse zu, so wird das Ergebnis wieder geboxed.
			
			\warning{Autoboxing kann zu unerwarteten Fehlern führen! Wird eine \texttt{null}-Instanz einer Wrapper-Klasse geunboxed, so führt dies zu einer \texttt{NullPointerException}!}
			
			Schauen wir uns also obigen Beispiel erneut an, diesmal aber mit Autoboxing:
			\begin{figure}[H]
				\centering
				\begin{lstlisting}
public List<Double> floatingAverage(List<Integer> list) {
	List<Double> result = new List<Double>();
	for (int i = 1; i < list.size(); i++) {
		int previous = list.get(i - 1);              // Autounboxing
		int current = list.get(i);                   // Autounboxing
		double average = (previous + current) / 2.0;
		result.add(average);                         // Autoboxing
	}
	return result;
}
				\end{lstlisting}
				\caption{Java: Autoboxing}
				\label{fig:java_generics_motivation_gen}
			\end{figure}
		% end
	% end
	
	\subsubsection{Vererbung}
		In Codebeispiel \ref{fig:java_generics_motivation_gen} verwenden wir \texttt{T}, um Listen jedes Typs annehmen zu können. Wieso können wir hier nicht einfach wie üblich \texttt{Object} nehmen? Der Grund ist: Generics unterstützen keine Vererbung, das heißt es wäre nicht möglich, Listen vom Typ \texttt{String} zu übergeben, wenn die Methode einen Parameter \texttt{List<Object>} erwartet.
		
		Es ist dennoch möglich, ein solches Verhalten hervorzubringen, wie wir im Abschnitt \ref{sec:generics_restriction} sehen werden.
	% end
	
	\subsubsection{Einschränkung der Typparameter und Wildcards}
		\label{sec:generics_restriction}
	
		Wie bereits erwähnt ist es nicht möglich, einer Methode mit Parameter \texttt{List<Object>} eine Instanz einer Liste \texttt{List<String>} zu übergeben.
		
		\paragraph{Wildcard}
			Eine Möglichkeit ist es wie in \ref{fig:generics_methoden} einen uneingeschränkten Typparameter zu definieren. Eine andere Möglichkeit ist, die Wildcard \texttt{?} zu verwenden, welche alle Typen akzeptiert. Somit kann die Methode in \ref{fig:generics_methoden} zu folgender, äquivalenter, Methode umgeschrieben werden:
			\begin{figure}[H]
				\centering
				\begin{lstlisting}
public static String concatenate(List<@@??@@@> list) {
	String result = "";
	for (int i = 0; i < list.size(); i++) {
		result += list.get(i).toString();
	}
	return result;
}
				\end{lstlisting}
				\caption{Java: Generics Wildcard}
			\end{figure}
			Das \enquote{\texttt{?}} nimmt dann alle Typen an.
		% end
		
		\paragraph{Einschränkungen der Typparameter}
			Oftmals ist es jedoch nötig, nicht einfach alle Typen zu akzeptieren, sondern die Parameter weiter einzuschränken. Hierzu gibt es folgende essentielle Operatoren:
			\begin{description}
				\item[\texttt{extends}] Akzeptiert alle Typen, die eine bestimmte Klasse sind oder eine Unterklasse von dieser sind.
				\item[\texttt{super}] Akzeptiert alle Typen, die eine bestimmte Klasse sind oder eine Oberklasse von dieser sind.
			\end{description}
			Wir werden gleich noch einige Beispiele betrachten, vorher sei jedoch noch gesagt, dass sich die Einschränkungen sowohl auf Typparameter selbst (also auf die Definition von \texttt{T}, \texttt{K}, \dots) anwenden lassen als auch direkt auf Wildcards (also \texttt{?}).
			
			\subparagraph{\texttt{extends}}
				Erweitern wir unsere Methoden zum Aufsummieren von Zahlen so, dass die Methode alle Zahlentypen akzeptiert und Wildcards nutzt (Achtung: In dieser Implementierung werden einfach alle Nachkommastellen abgeschnitten, sollte eine Fließkommazahl übergeben werden!):
				\begin{figure}[H]
					\centering
					\begin{lstlisting}
public long sum(List@@?<? extends Number>@@@ list) {
	long result = 0;
	for (int i = 0; i < list.size(); i++) {
		result += list.get(i).longValue();
	}
	return result;
}

// Erfolgreiche Aufrufe.
List<Integer> integerList = new List<Integer>();
sum(integerList);

List<Long>    longList    = new List<Long>();
sum(longList);

List<Float>   floatList   = new List<Float>();
sum(floatList);


// Fehlschlagende Aufrufe.
List<String> stringList = new List<String>();
@@!sum(stringList)@@@;
					\end{lstlisting}
					\caption{Java: Generic \texttt{extends}}
				\end{figure}
				Die Methode kann nun mit allen Unterklassen von \texttt{Number} aufgerufen werden. Eine Auswahl dieser ist auch im Codebeispiel gegeben.
			% end
			
			\subparagraph{\texttt{super}}
				Zur Nutzung von \texttt{super} schauen wir uns folgendes Beispiel an, welches die Elemente einer Liste \texttt{src} in eine Liste \texttt{dest} kopiert und dabei Typsicher verfährt:
				\begin{figure}[H]
					\centering
					\begin{lstlisting}
public static <T> copy(List<? extends T> src, List<? super T> dest) {
	for (int i = 0; i < src.size(); i++) {
		dest.add(a.get(i));
	}
}

// Erfolgreiche Aufrufe.
List<String>            src  = new List<String>();
List<CharacterSequence> dest = new List<CharacterSequence>();
copy(src, dest);


// Fehlschlagende Aufrufe.
List<CharacterSequence> src  = new List<CharacterSequence>();
List<String>            dest = new List<String>();
@@!copy(src, dest)@@@;
					\end{lstlisting}
					\caption{Java: Generics \texttt{super}}
				\end{figure}
				Der zweite Aufruf schlägt hierbei fehl, denn weder ist \texttt{CharacterSequence} eine Unterklasse von \texttt{String}, noch ist \texttt{String} eine Oberklasse von \texttt{CharacterSequence}. Das Gegenteil ist hingegen der Fall, womit der erste Aufruf erfolgreich ist.
			% end
			
			\subparagraph{Zusammenfassung \texttt{extends}/\texttt{super}}
				Für die obige Definition von \texttt{get} und \texttt{add} (welche die übliche ist bei solchen Listen), gilt folgende Tabelle, was welche Methode zurück gibt oder annimmt:
				\begin{table}[H]
					\centering
					\todo{Tabelle Generics extends/super}
					\caption{Java: Generics Tabelle \texttt{extends}/\texttt{super}}
				\end{table}
			% end
		% end
	% end
	
	\subsubsection{Typlöschung (Type Erasure)}
		In diesem Abschnitt werden wir uns mit der Implementierung von Generics im Compiler beschäftigen, welche zu einigen interessanten Eigenschaften von Generics führt.
		
		Implementiert werden Generics durch Typlöschung, das heißt, die Generics sind zur Laufzeit nicht mehr vorhanden. Stattdessen werden die Aufrufe, bei denen die Typen durch Generics gesteuert werden, durch den höchstmöglichen Typ (meist \texttt{Object} wenn kein \texttt{extends} genutzt wurde) und Downcasts ersetzt, das heißt, der Code wird zu einem äquivalenten Code umgebaut.
		
		\paragraph{Beispiel}
			Betrachten wir folgende Methode, welche das maximale Element einer List findet:
			\begin{figure}[H]
				\centering
				\begin{lstlisting}
public <T extends Comparable<? super T>> T max(List<T> list) {
	T max = null;
	for (int i = 0; i < list.size(); i++) {
		if (max == null || list.get(i).compareTo(max) > 0) {
			max = list.get(i);
		}
	}
	return max;
}

// Aufruf.
List<Integer> list = new List<Integer>();
list.add(1); list.add(2); list.add(3);
Integer max = max(list);                  // max == 3
				\end{lstlisting}
				\caption{Java: Generics vor Typlöschung}
			\end{figure}
			von dem Compiler durch folgenden Code ersetzt (das Autoboxing wird hier außer Acht gelassen):
			\begin{figure}[H]
				\centering
				\begin{lstlisting}
public Comparable max(List list) {
	Comparable max = null;
	for (int i = 0; i < list.size(); i++) {
		if (max == null || ((Comparable) list.get(i)).compareTo(max) > 0) {
			max = (Comparable) list.get(i);
		}
	}
	return max;
}

// Aufruf.
List list = new List();
list.add(1); list.add(2); list.add(3);
Integer max = (Integer) max(list);     // max == 3
				\end{lstlisting}
				\caption{Java: Generics vor Typlöschung}
			\end{figure}
		% end
	% end
	
	\subsubsection{Limitierungen}
		Primär durch die Typlöschung verursacht gibt es einige unerwartete Effekte, wenn man mit Generics arbeitet. Wir werden nun die zwei wichtigsten betrachten:
		\begin{itemize}
			\item Es ist nicht (mit vertretbarem Aufwand) möglich, ein Array eines Typparameters erstellen. Dies lässt sich durch die Typlöschung einfach begründen, wenn Code der Form \texttt{new T[10]} würde zu \texttt{new Object[10]} umgebaut werden, was nicht zu \texttt{String[]} oder welchen Typ auch immer gecastet werden kann.
			\item Es ist außerdem nicht möglich, neue Instanzen eines Typparameters zu erstellen. Dies ist lässt sich auf zwei Faktoren zurück führen:
				\begin{enumerate}
					\item Es ist zur Compilezeit nicht bekannt, welche Parameter der Konstruktor hat.
					\item Durch die Typlöschung würde Code der Form \texttt{new T()} zu \texttt{new Object()} umgebaut werden, was absolut nutzlos ist und auch nicht weiter gecastet werden kann.
				\end{enumerate}
		\end{itemize}
	% end
% end

% end

\section{Datenstrukturen}
	\label{sec:java_datenstrukturen}

\todo{Schreiben}
% end

\section{Fehlerbehandlung}
	\todo{Schreiben}

\subsection{Exceptions}
	\todo{Schreiben}
	
	\subsubsection{Fangen von Exceptions}
		% `try`-`catch`-`finally`, `try` with resources
		\todo{Schreiben}
	% end
	
	\subsubsection{Erstellen von Exceptions und Exception-Hierarchie}
		\todo{Schreiben}
	% end
	
	\subsubsection{Werfen von Exceptions}
		\todo{Schreiben}
	% end
	
	\subsubsection{Weiterleiten von Exceptions}
		% throws
		\todo{Schreiben}
	% end
	
	\subsubsection{Spezielle Exceptions}
		\todo{Schreiben}
	% end
	
	\subsubsection{Exceptions in Lambdas}
		\todo{Schreiben}
	% end
% end

\subsection{Result Code}
	\todo{Schreiben}
% end

% end

\section{Dokumentation mit JavaDoc}
	\todo{Schreiben}

% Klassen
% Methoden
% Konstruktoren
% etc.

% end

    %% end
    
    \chapter{Java}
    	\label{c:java}
    	
    	Dieses Kapitel wird in den nächsten Wochen folgen.
    % end
    
    %\chapter{Abstraktion}
	%    \label{c:abstraktion}
    %
	%    \todo{Schreiben}

\section{Funktionale Abstraktion}
	\input{parts/abstraktion/functional}
% end

\section{Objektorientierte Abstraktion}
	\todo{Schreiben}

\subsection{Konzept}
	\todo{Schreiben}
% end

\subsection{Klassen, Objekte und Methoden}
	\todo{Schreiben}
% end

\subsection{Vererbung}
	\todo{Schreiben}
% end

\subsection{Abstrakte Klassen}
	\todo{Schreiben}
% end

\subsection{Interfaces}
	\todo{Schreiben}
% end

\subsection{Polymorphie und späte Bindung}
	\todo{Schreiben}
% end

% end
    %% end
    
    \chapter{Abstraktion}
    	\label{c:abstraktion}
    	
    	Dieses Kapitel wird in den nächsten Wochen folgen.
    % end
    
    %\chapter{Komplexität und Landau-Symbolik}
	%    \label{c:komplexitaet}
    %
	%    Wir haben in den vorhergehenden Kapiteln viel über Programmiersprachen und deren Eigenarten gehört und widmen uns nun einem etwas theoretischeren Thema als die Programmierung an sich: Der Komplexitätstheorie, genauer der Landau-Symbolik.

Betrachten wir hier zum Einstieg folgende Algorithmen, die zum einen die Fakultät einer Ganzzahl berechnen und zum anderen die Fakultät-Fakultät einer Ganzzahl berechnen (d.h. das Produkt der Fakultäten)
\begin{figure}[H]
	\centering
	\caption{Komplexität: Berechnung der Fakultät}
\end{figure}

\todo{Stopped here.}
    %% end

    %\chapter{Glossar}
	%    \label{c:glossar}
    %
    %    \todo{Skript zur Generierung des Glossars.}
    %% end
    
    \bibliography{../cite}
\end{document}
