\documentclass[a4paper, 11pt, accentcolor = tud3b]{tudreport}

% Core packages.
\usepackage[T1]{fontenc}
\usepackage[utf8]{inputenc}
\usepackage[ngerman]{babel}
% Other packages.
\usepackage[linesnumbered, ruled]{algorithm2e}
\usepackage{cite}
\usepackage{enumitem}
\usepackage{tabto}
\usepackage[mathcal]{euscript} % Get readable mathcal font.
\usepackage{hyperref}
\usepackage{listings}
\usepackage{mathtools}
\usepackage{xspace}
\usepackage[disable]{todonotes}
\usepackage{tcolorbox}
\usepackage[german = quotes]{csquotes}
\usepackage{tikz}
\usepackage{wrapfig}
\usepackage{stmaryrd}
\usepackage{float}
\usepackage{multicol}
\usepackage{csvsimple}
\usepackage{wrapfig}
\tcbuselibrary{skins}
\usetikzlibrary{arrows.meta, shapes, backgrounds, calc}

% Basic information.
\title{Funktionale und objektorientierte Programmierkonzepte \\ Version 1.1}
\subtitle{Fabian Damken (\href{mailto:fabian.damken@stud.tu-darmstadt.de?subject=[FoP-Skript]}{fabian.damken@stud.tu-darmstadt.de})}
\author{Fabian Damken}
\date{\today}

% Description-list styling.
\SetLabelAlign{parright}{\parbox[t]{\labelwidth}{\raggedleft#1}}
\setlist[description]{style = multiline, leftmargin = 4cm, align = parright}

\colorlet{lstcomments}{tud4c}
\colorlet{lstkeywords}{tud9d}
\colorlet{lstlinenumbers}{tud0c}
\colorlet{lststrings}{tud2c}
\colorlet{functionalmarker}{tud2b}
\colorlet{imperativemarker}{tud4b}
\colorlet{oopmarker}{tud6b}
\colorlet{errorred}{tud9b}
\colorlet{changedpurple}{tud11a}

\tikzset{> = { Latex[length = 2mm] }}

\lstdefinelanguage{Racket}{
	morekeywords = {
		list,
		cons,
		first,
		second,
		third,
		fourth,
		fifth,
		sixth,
		seventh,
		eighth,
		rest,
		empty,
		number,
		real,
		rational,
		integer,
		natural,
		string,
		true,
		false,
		\#t,
		\#f,
		\#true,
		\#false,
		define,
		remainder,
		if,
		cond,
		error,
		define-struct,
		print,
		lambda,
		else
	},
	sensitive = true,
	morecomment=[l]{;},
	morestring=[b]",
	keepspaces = true
}
\lstdefinestyle{base}{
	moredelim = **[is][\color{errorred}]{@@!}{@@@},
	moredelim = **[is][\color{changedpurple}]{@@?}{@@@}
}
\lstset{
	backgroundcolor = \color{white},
	basicstyle = \ttfamily\normalsize\color{black},
	breakatwhitespace = true,
	breaklines = true,
	breakautoindent = true,
	commentstyle = \color{lstcomments},
	escapeinside = {{*@}{@*}},
	keywordstyle = \color{lstkeywords},
	numbers = left,
	numberstyle = \tiny\color{lstlinenumbers},
	showstringspaces = false,
	stringstyle = \color{lststrings},
	tabsize = 4
}

% New commands.
\providecommand{\textbox}[1]{
	\begin{figure}[H]
		\centering
		\fbox{\parbox[c]{0.75\textwidth}{#1}}
	\end{figure}
}
\providecommand{\info}[1]{\textbox{\textbf{Info:} #1}}
\providecommand{\warning}[1]{\textbox{\textbf{Warnung:} #1}}
\providecommand{\funfact}[1]{\textbox{\textbf{Fun Fact:} #1}}
\providecommand{\memo}[1]{\textbox{\textbf{Merksatz:} #1}}
\providecommand{\forwhich}{\ensuremath{{\,\vert\,}}}
\providecommand{\abs}[1]{\ensuremath{{\lvert #1 \rvert}}}

\providecommand{\qed}{{\hfill q.e.d.}}

\providecommand{\subsubparagraph}[1]{\hspace{1cm} \textbf{#1:}}

\providecommand{\definition}[2]{\subparagraph{Definition (#1)} #2}
\providecommand{\notation}[2]{\subparagraph{Notation (#1)} #2}
\providecommand{\theorem}[1]{\subparagraph{Theorem} #1}
\providecommand{\intuition}[1]{\subsubparagraph{Intuition} #1}

\providecommand{\functional}{%
	\begin{tcolorbox}[standard jigsaw, colframe = functionalmarker, colback = tud2a, width = 2.3cm, height = 0.5cm, top = -2pt, left = 5pt, boxrule = 3pt, opacityback = 0.8]
		\footnotesize Funktional
	\end{tcolorbox}
}
\providecommand{\imperative}{%
	\begin{tcolorbox}[standard jigsaw, colframe = imperativemarker, colback = tud4a, width = 2.25cm, height = 0.5cm, top = -2pt, left = 7pt, boxrule = 3pt, opacityback = 0.8]
		\footnotesize Imperativ
	\end{tcolorbox}
}
\providecommand{\oop}{%
	\begin{tcolorbox}[standard jigsaw, colframe = oopmarker, colback = tud6a, width = 3.1cm, height = 0.5cm, top = -2pt, left = 6pt, boxrule = 3pt, opacityback = 0.8]
		\footnotesize Objektorientiert
	\end{tcolorbox}
}
\providecommand{\functionalMark}{%
	\raggedleft
	\functional
	\raggedright
}
\providecommand{\imperativeMark}{%
	\raggedleft
	\imperative
	\raggedright
}
\providecommand{\oopMark}{%
	\raggedleft
	\oop
	\raggedright
}
%\providecommand{\functionalMark}{\flushright \functional \flushleft}
%\providecommand{\imperativeMark}{\flushright \imperative \flushleft}
%\providecommand{\oopMark}{\flushright \oop \flushleft}

\providecommand{\introduces}[2]{\label{def:#2} Dieser Abschnitt führt das Konzept #1 ein.}
\providecommand{\implements}[3]{\label{imp:#2_#3} Dieser Abschnitt beschreibt die Implementierung von #1 in #3. Für die abstrakte Definition siehe \ref{def:#2}.}
\providecommand{\refIntr}[1]{\ref{def:#1}}
\providecommand{\refImpl}[2]{\ref{imp:#1_#2}}

\providecommand{\HREF}[1]{\href{#1}{#1}}

\begin{document}
	\bibliographystyle{alpha}

    \maketitle
    \tableofcontents
    \listoftodos

    \chapter{Einführung}
	    \label{c:einfuehrung}
    
        \todo{Schreiben}

\section{Aufbau}
	Dieses Skript ist in folgende Kapitel gegliedert:
	\begin{itemize}
		\item[\ref{c:abstrakte_konzepte}] Abstrakte Konzepte \\ In diesem Kapitel werden abstrakte Konzepte der Programmierung eingeführt, d.h. es wird über keine Programmiersprache an sich gesprochen.
		\item[\ref{c:racket}] \racket \\ Dieses Kapitel führt in die funktionale (\ref{sec:paradigma_imperativ}) Programmiersprache \racket ein, indem die im vorherigen Kapitel eingeführten Konzepte auf die Sprache angewendet werden.
		\item[\ref{c:java}] Java \\ Ebenso wir im Kapitel über \racket, nur werden hier die Konzepte auf die objektorientierte (\refIntr{oop}) und imperative (\ref{sec:paradigma_imperativ}) Programmiersprache Java angewendet.
		\item[\ref{c:abstraktion}] Abstraktion \\ Dieses Kapitel behandelt unterschiedliche Methodiken der Abstraktion, wie sie in funktionaler und objektorientierter Programmierung eingesetzt werden.
		%\item[\ref{c:komplexitaet}] Komplexität und Landau-Symbolik \\ Dieses Kapitel führt in die Komplexität von Algorithmen und die Landau-Symbolik ein.
		%\item[\ref{c:glossar}] Glossar \\ Ein Glossar, welches synchron zu dem Glossar im jeweiligen Moodle-Kurs ist.
	\end{itemize}
% end
    % end

	\chapter{Abstrakte Konzepte}
		\label{c:abstrakte_konzepte}
	
		Dieses Kapitel führt ein in die abstrakten Konzepte, welche hinter eine Programmiersprache stehen.

Da sich nicht alle Konzepte auf alle Programmierparadigmen \footnote{Siehe \ref{sec:paradigmen}} anwenden lassen, ist jeder Abschnitt mit
\begin{itemize}
	\item[] \functional
	\item[] \imperative
	\item[] \oop
\end{itemize}
gekennzeichnet, je nachdem, auf welches Paradigma sich das vorgestellte Konzept anwenden lässt. Die Markierungen werden am rechten Rand angebracht sein, sodass diese leicht zu finden sind.

\section{Programmierparadigmen}
	\label{sec:paradigmen}
	
	\subsection{Deklarativ}
	\introduces{von deklarativen Programmiersprachen}{deklarativ}

    Der der Deklarativen Programmierung steht die Beschreibung des Problems im Vordergrund, die Lösung wird hier meist automatisiert gefunden.
    
    Es steht somit im Vordergrund, \textit{welches} Problem gelöst werden soll und nicht \textit{wie} ein Problem gelöst werden soll. Hierdurch ist eine genaue Trennung von Problem und Implementierung möglich, was bei imperativen Programmiersprachen (\ref{sec:paradigma_imperativ}) gar nicht oder zumindest nicht trivial möglich ist.
    
    Das Paradigma der deklarativen Programmierung kann in weitere unterteilt werden, beispielsweise in funktionale (\refIntr{funktional}) und logische Sprachen. Logische Sprachen werden hier nicht weiter ausgeführt.
    
    \paragraph{Beispiele}
        \begin{itemize}
            \item SQL, Cypher (Abfragesprachen)
            \item Lisp, Racket, Haskell (Funktionale Sprachen)
            \item Prolog (Logische Sprache)
        \end{itemize}
    % end
% end

\subsection{Funktional}
	\introduces{von funktionalen Programmiersprachen}{funktional}

    Funktionale Programmiersprache sind Ausarbeitungen von deklarativen Sprachen (\refIntr{deklarativ}), bei denen ebenfalls die Beschreibung des Problems im Vordergrund steht. Sie werden oftmals zur Beschreibung von mathematischen Problem verwendet.
    
    In diesen Sprachen wir auf Konstrukte wie Schleifen (\refIntr{schleifen}) und Variablen (\refIntr{variablen}) verzichtet, wodurch Seiteneffekte (beispielsweise das Überschreiben von Zustandsvariablen) verhindert werden und die Implementierung zur Lösung eines Problems robuster wird.
    
    Zur Abgrenzung von funktionalen Sprachen zu imperativen Sprachen siehe \ref{sec:paradigma_abgrenzung_funktional_imperativ}.
    
    \paragraph{Beispiele}
        \begin{itemize}
            \item Lisp
            \item Racket
            \item Haskell
        \end{itemize}
    % end
% end

\subsection{Imperativ}
    \label{sec:paradigma_imperativ}

    Imperative Programmiersprachen sehen vor, dass der Entwickler beschreibt, \textit{wie} ein Problem zu lösen ist, wobei die Beschreibung des eigentlichen Problems (das \enquote{\textit{Was}}) fallen gelassen wird. Ein Programm besteht \enquote{aus einer Folge von Anweisungen [\dots], die vorgeben, in welcher Reihenfolge was vom Computer getan werden soll}. ~\cite{andreas2005grundkurs}
    
    Im Gegensatz zu deklarativen und funktionalen Sprachen ist die Korrektheit eines Algorithmus weniger offensichtlich und es werden Kontrollstrukturen wie Schleifen (\refIntr{schleifen}) und Variablen (\refIntr{variablen}) eingeführt.
    
    Zur Abgrenzung von imperativen und funktionalen Sprachen siehe \ref{sec:paradigma_abgrenzung_funktional_imperativ}.
    
    \paragraph{Beispiele}
        \begin{itemize}
            \item Java
            \item C/C++
            \item Assembler
        \end{itemize}
    % end
% end

\subsection{Objektorientiert}
	\introduces{von objektorientierten Programmiersprachen}{oop}
    
    Bei der objektorientierten Programmierung (OOP) werden reale Strukturen, sogenannte Objekte in der Software abgebildet. Die Architektur der Software wird somit an bestehenden Systemen der Wirklichkeit abgebildet und erlaubt den meisten Entwickler*innen einen einfachen Zugang zu der Software, da die Wirklichkeit repräsentiert wird. Ein Programm besteht besteht aus Anweisungen, welche vorgeben, was der Computer abarbeiten soll und in welcher Reihenfolge. Somit sind (die meisten) objektorientierten Programmiersprachen ebenfalls imperativ (\ref{sec:paradigma_imperativ}).
    
    Durch die Vermischung mit imperativen und funktionalen Paradigmen lassen sich objektorientierte Sprachen nicht hinreichend von ersteren Abgrenzen, da letztere meistens auch Teile der imperativen und funktionalen Paradigmen beinhalten.
    
    \paragraph{Beispiele}
        \begin{itemize}
            \item Java
            \item C++
            \item Python
        \end{itemize}
    % end
% end

\subsection{Abgrenzung Funktional $ \leftrightarrow $ Imperativ}
	\label{sec:paradigma_abgrenzung_funktional_imperativ}

    Die Abgrenzung von funktionalen und imperativen Sprachen lässt sich am besten anhand eines Beispiels erläutern:
    
    Gegeben sei das mathematische Problem, die Fakultät einer beliebigen natürlichen Zahl $ n \in \mathbb{N} _ 0 $ zu bestimmen. Mathematisch wird das Problem wie folgt rekursiv definiert:
    \begin{equation*}
        n! = f(n) = \begin{cases*}
            1 & \text{ falls } n = 0 \\
            n \cdot f(n - 1) & \text{ falls } n > 0 \\
        \end{cases*}
    \end{equation*}
    
    In einer (fiktionalen) funktionalen Sprache kann das Problem folgendermaßen implementiert werden:
    \begin{figure}[H]
        \centering
        \begin{lstlisting}
f(0) := 1
f(n) := n * f(n - 1)
        \end{lstlisting}
        \caption{Funktionale Implementierung der Fakultät}
    \end{figure}
    
    In einer (ebenfalls fiktionalen) imperativen Sprache kann das Problem folgendermaßen implementiert werden:
    \begin{figure}[H]
        \centering
        \begin{lstlisting}
function f(n)
    num = 1
    for i in 1..n
        num = num * i
    endloop
endfunction
        \end{lstlisting}
        \caption{Imperative Implementierung der Fakultät}
    \end{figure}
% end

% end

\section{Lexikalische Bestandteile} \functionalMark \imperativeMark \oopMark
	\subsection{Datentypen}
	\implements{Datentypen}{datentypen}{Java}
	
	Java ist eine statisch Typisierte Sprache, das heißt der Typ einer Variable muss jederzeit angegeben werden und dem Compiler bekannt sein. Es ist nicht möglich, eine Variable nacheinander für zum Beispiel Zahlen und Zeichenketten zu verwenden.
	
	In Java existieren viele Datentypen, die in zwei Kategorien unterteilt werden können:
	\begin{itemize}
		\item Primitive Datentypen
		\item Objektreferenzen
	\end{itemize}
	
	\paragraph{Primitive Datentypen}
		Einer der Unterschiede zwischen primitiven Datentypen und Objektreferenzen ist, dass Daten, welche in primitiven Datentypen gespeichert sind, mit Pass-by-Value weitergegeben werden. Das bedeutet, die Daten werden bei einer Übergabe an die Methode kopiert und Änderungen an den Daten an einer Stelle wirken sich nicht auf andere Stelle aus. Außerdem sind ist die Anzahl an primitiven Datentypen begrenzt und die Datentypen sind von vornherein festgelegt. Ferner gibt es große Unterschiede bei der Behandlung von Konstanten, die wir später betrachten werden. \todo{Primitive vs. Objekte: Konstanten} Als ersten Anhaltspunkt eignet sich, dass primitive Datentypen mit einem kleinen Buchstaben und Objektreferenztypen mit einem großen Buchstaben beginnen.
		
		Es existieren folgende primitive Datentypen:
		\begin{table}[H]
			\centering
			\begin{tabular}{l | l | l | l}
				Schlüsselwort       & Typ            & Beschreibung               & Wertebereich                                                      \\ \hline
				\lstinline|byte|    & Ganzzahl       & Vorzeichenbehaftet, 8 Bit  & \( -(2 ^ { 7}) \) bis \( 2 ^ { 7} - 1 \)                          \\
				\lstinline|short|   & Ganzzahl       & Vorzeichenbehaftet, 16 Bit & \( -(2 ^ {15}) \) bis \( 2 ^ {15} - 1 \)                          \\
				\lstinline|int|     & Ganzzahl       & Vorzeichenbehaftet, 32 Bit & \( -(2 ^ {31}) \) bis \( 2 ^ {31} - 1 \)                          \\
				\lstinline|long|    & Ganzzahl       & Vorzeichenbehaftet, 64 Bit & \( -(2 ^ {63}) \) bis \( 2 ^ {63} - 1 \)                          \\
				\lstinline|float|   & Fließkommazahl & einfache Genauigkeit       & \( 1,4 \cdot 10 ^ {-45} \) bis \( \approx 3,4 \cdot 10 ^ {38} \)  \\
				\lstinline|double|  & Fließkommazahl & doppelte Genauigkeit       & \( 4,9 \cdot 10 ^ {324} \) bis \( \approx 1,8 \cdot 10 ^ {308} \) \\
				\lstinline|char|    & Charakter      & Unicode-Code, 16 Bit       & \( 0 \) bis \( 2 ^ {16} - 1 \)                                    \\
				\lstinline|boolean| & Wahrheitswert  &                            & \texttt{true}/\texttt{false}
			\end{tabular}
			\caption{Liste der primitiven Datentypen in Java}
		\end{table}
		
		Hierbei fällt auf, dass es in Java keinen eingebauten Datentyp für vorzeichenfreie Zahlen (\enquote{unsigned}) gibt. Dies kann bei der Verarbeitung von Binärdaten (beispielsweise bei Netzwerkkommunikation) zu Fehlern führen.
	% end
	
	\paragraph{Objektreferenzen}
		Neben primitiven Datentypen gibt es noch die Objektreferenzen. Diese verhalten sich anders als primitive Datentypen, wobei uns vor allem die folgenden Unterschiede auffallen:
		\begin{itemize}
			\item Es gibt als einziges \enquote{echtes} Literal den Wert \lstinline|null|, der aussagt, dass die Objektreferenz kein Objekt referenziert. Jegliche Methodenaufrufe auf dieser Referenz brechen mit einer \lstinline|NullPointerException| ab. Deshalb muss vor jedem Zugriff auf eine solche Referenz geprüft werden, ob sie ungleich \lstinline|null| ist.
			\item Werden Objektreferenzen als Parameter übergeben, so wird hierbei ausschließlich die Referenz übergeben und auf das gleiche Objekt referenziert (Pass-by-Reference). Das bedeutet, eine Änderung an dem Objekt an einer Stelle kann sich an beliebig vielen anderen Stellen auswirken.
		\end{itemize}
		Wir werden uns Objektreferenzen im Abschnitt \refImpl{oop}{Java} über objektorientierte Programmierung in Java nochmals genauer anschauen.
	% end
	
	\paragraph{Sonderfall \lstinline|String|}
		Ein \lstinline|String| ist eine Objektreferenz, kann allerdings in manchen Bereichen als ein primitiver Datentyp angesehen werden. Beispielsweise existieren, wir wir weiter unten noch sehen werden, Literale für diesen Datentyp, welche implizit ein Objekt erzeugen. Auch scheint ein String mit Pass-by-Value übergeben zu werden, da ein String nicht veränderbar ist.
		
		Insgesamt existieren folgende Unterschiede:
		\begin{itemize}
			\item Ein String ist \textit{immutable}, das heißt nicht veränderlich.
			\item Dadurch scheint es, als wird ein String Pass-by-Value übergeben.
			\item Bei Konstanten wird ein String als primitiver Typ angesehen und gleich behandelt.
			\item Es gibt eine syntaktische Form, String-Literale auszudrücken.
			\item Trotz allem ist es möglich, einem String den Wert \lstinline|null| zuzuweisen. Dies ist gleichermaßen praktisch wie nervig.
		\end{itemize}
	% end
% end

\subsection{Literale}
	\implements{Literalen}{literale}{Java}
	
	In Java gibt es Schreibweisen für Literale für alle Datentypen, wobei die Erstellung von Objekten einen Sonderfall darstellt und nicht vollständig als Literal bezeichnet werden kann (es können zwar alle Argumente fest im Code stehen, das Objekt selbst wird allerdings erst zur Laufzeit erstellt).
	
	In der folgenden Tabelle sind sämtliche syntaktische Methoden zu Definition von Literalen gelistet:
	\begin{table}[H]
		\centering
		\begin{tabular}{l | l}
			\textbf{Datentyp} & \textbf{Schreibweise}                          \\ \hline
			\texttt{byte}     & \texttt{123}, \texttt{-123}                    \\
			\texttt{short}    & \texttt{1234}, \texttt{-1234}                   \\
			\texttt{int}      & \texttt{12345}, \texttt{-12345}                 \\
			\texttt{long}     & \texttt{123456L}, \texttt{123456L}               \\
			\texttt{float}    & \texttt{12.34F}, \texttt{0.34F}/\texttt{.34F}  \\
			\texttt{double}   & \texttt{123.456}, \texttt{0.456}/\texttt{.456} \\
			\texttt{char}     & \texttt{'a'}                                   \\
			\texttt{boolean}  & \texttt{true}, \texttt{false}                  \\
			\texttt{String}   & \texttt{"Hello, World!"}                       \\
			\texttt{Object}   & \texttt{null}
		\end{tabular}
	\end{table}
	\begin{itemize}
		\item \texttt{String} ist hier kein primitiver Datentyp, das heißt mit einem String-Literal wird auch immer ein neues Objekt erzeugt.
		\item Das Literal \texttt{null} für \texttt{Object} ist allgemein anwendbar, wenn mit Objekten gearbeitet wird. Allerdings kann dies zu unerwarteten \textit{\texttt{NullPointerExceptions}} führen, welche wir später noch eingehend betrachten werden.
	\end{itemize}
	
	Bei Literalen von Zahlen gibt es außerdem folgende Besonderheiten:
	\begin{itemize}
		\item Bei einem \texttt{float}-Literal muss ein \enquote{\texttt{F}} am Ende des Literals angehängt werden, damit das Literal als \texttt{float} und nicht als \texttt{double} interpretiert wird. Die Groß-/Kleinschreibung ist irrelevant.
		\item Bei einem \texttt{long}-Literal kann ein \enquote{\texttt{L}} am Ende des Literals angehängt werden, damit das Literal als \texttt{long} und nicht als \texttt{int} interpretiert wird. Die Groß-/Kleinschreibung ist irrelevant, aufgrund der Ähnlichkeit von \enquote{\texttt{l}} und \enquote{\texttt{1}} wir allerdings ein großes \enquote{\texttt{L}} empfohlen.
		\item Bei allen Ganzzahlen (\texttt{byte}, \texttt{short}, \texttt{int}, \texttt{long}) können die Zahlen mit den Zahlensystemen Binär, Oktal, Dezimal und Hexadezimal eingegeben werden, wobei Dezimal sinnvollerweise der Standard ist. Zur Nutzung hiervon müssen den Werten bestimmte Zeichenketten vorangestellt werden. Dies sind \texttt{0b} für Binär, \texttt{0} für Oktal, nichts für Dezimal und \texttt{0x} für Hexadezimal. \\ Das heißt, die folgenden Literale sind äquivalent:
			\begin{itemize}
				\item \texttt{0b101010}
				\item \texttt{052}
				\item \texttt{42}
				\item \texttt{0x2A}
			\end{itemize}
			Wobei auch hier die Groß-/Kleinschreibung irrelevant ist, für den Prefix allerdings die Kleinschreibung und für die Zahl die Großschreibung empfohlen wird.
	\end{itemize}
	
	\warning{Wird bei Zahlen eine \texttt{0} vorangestellt, wird die Zahl Oktal interpretiert! Das heißt es gilt \texttt{010 \(\neq\) 10}.}
	
	\paragraph{Escape-Sequenzen}
		Escape-Sequenzen werden innerhalb eines Strings mit einem Backslash (\textbackslash) eingeleitet und bestehen in den meisten Fällen auf einem Zeichen.
		
		In Java sind folgende Escape-Sequenzen verfügbar:
		\begin{table}[H]
			\centering
			\begin{tabular}{c | l}
				\textbf{Escape-Sequenz} & \textbf{Repräsentiertes Zeichen} \\ \hline
				\lstinline|\t|          & Tab.                             \\
				\lstinline|\b|          & Backspace.                       \\
				\lstinline|\n|          & New line.                        \\
				\lstinline|\r|          & Carriage Return.                 \\
				\lstinline|\f|          & Formfeed.                        \\
				\lstinline|\'|          & Single quote.                    \\
				\lstinline|\"|          & Double quote.                    \\
				\lstinline|\\|          & Backslash.
			\end{tabular}
			\caption{Java: Escape-Sequenzen}
		\end{table}
	% end
% end

\subsection{Schlüsselwörter}
	\implements{Schlüsselwörtern}{keywords}{Java}
	
	In Java existieren folgende Schlüsselwörter (kursiv geschriebene Themen werden wir nicht ausführlicher betrachten):
	\begin{description}
        \item[\texttt{abstract}] Markiert eine\dots
	        \begin{description}
	        	\item[Klasse] das heißt, diese kann abstrakte Methoden enthalten.
	        	\item[Methode] die von Unterklassen implementiert werden muss.
	        \end{description}
        \item[\texttt{continue}] Fährt in einer Schleife mit dem nächsten Element fort.
        \item[\texttt{for}] Leitet eine for-Schleife ein.
        \item[\texttt{new}] Operator zur Erstellung eines neuen Objektes einer Klasse.
        \item[\texttt{switch}] Leitet eine switch-Anweisung ein.
        \item[\texttt{assert}] Legt bestimmte Bedingungen fest, die für Parameter gelten müssen. Gelten diese nicht, wird ein Fehler ausgelöst.
        \item[\texttt{default}]
	        \begin{itemize}
	        	\item Default-Fall in einer switch-Anweisung.
	        	\item Definition einer Default-Methode innerhalb eines Interfaces.
	        	\item \textit{Definition des Default-Wertes einer Methode in einer Annotation}
	        \end{itemize}
        \item[\texttt{if}] Leitet eine if-Verzweigung ein.
        \item[\texttt{package}] Definition des Packages einer Klasse.
        \item[\texttt{synchronized}] Markiert eine Methode oder einen Codeblock als synchron, das heißt es kann maximal ein Thread zur gleichen Zeit die Methode \enquote{betreten}.
        \item[\texttt{boolean}] Datentyp.
        \item[\texttt{do}] Leitet eine do-while-Schleife ein.
        \item[\texttt{goto}] Reserviert. Löst ausschließlich einen Compilefehler aus.
        \item[\texttt{private}] Markiert eine Klasse, einen Konstruktor, eine Methode oder ein Attribut als privat.
        \item[\texttt{this}] Referenz auf die Instanz des aktuellen Objektes.
        \item[\texttt{break}] Bricht die Ausführung einer Schleife ab.
        \item[\texttt{double}] Datentyp.
        \item[\texttt{implements}] Implementiert ein Interface.
        \item[\texttt{protected}] Markiert eine Klasse, einen Konstruktor, eine Methode oder ein Attribut als protected.
        \item[\texttt{throw}] Wirft eine Instanz einer Exception.
        \item[\texttt{byte}] Datentyp.
        \item[\texttt{else}] Leitet einen else-Block ein.
        \item[\texttt{import}] Importiert eine Klasse/Methode aus einem anderen Paket.
        \item[\texttt{public}]  Markiert eine Klasse, einen Konstruktor, eine Methode oder ein Attribut als public.
        \item[\texttt{throws}] Definiert, dass ein Konstruktor/eine Methode eine bestimmte Exception werfen kann.
        \item[\texttt{case}] Leitet einen Fall eines switch-Ausdruckes ein.
        \item[\texttt{enum}] Leitet die Definition eines Enums ein.
        \item[\texttt{instanceof}] Operator zum Prüfen, ob eine Instanz eine Instanz einer anderen Klasse ist.
        \item[\texttt{return}] Gibt einen Wert zurück und bricht die Ausführung der Methode/des Konstruktors ab.
        \item[\textit{\texttt{transient}}] \textit{Definiert, dass ein bestimmte Attribut einer Instanz nicht mit serialisiert wird.}
        \item[\texttt{catch}] Leitet einen catch-Block ein.
        \item[\texttt{extends}]
	        \begin{description}
	        	\item[Klasse] Erweitert eine bestehende (möglicherweise abstrakte) Klasse.
	        	\item[Interface] Erweitert ein bestehendes Interface.
	        \end{description}
        \item[\texttt{int}] Datentyp.
        \item[\texttt{short}] Datentyp.
        \item[\texttt{try}] Leitet einen try-Block ein.
        \item[\texttt{char}] Datentyp.
        \item[\texttt{final}]
	        \begin{description}
	        	\item[Klasse] Die Klasse ist nicht vererbbar.
	        	\item[Methode] Die Methode ist nicht überschreibbar.
	        	\item[Variable] Die Variable ist nur einmal zuweisbar.
	        \end{description}
        \item[\texttt{interface}] Leitet die Definition eines Interfaces ein.
        \item[\texttt{static}] Markiert eine innere Klasse, eine Methode oder ein Attribut als statisch.
        \item[\texttt{void}] \enquote{Datentyp} als Platzhalter für \enquote{Nichts}.
        \item[\texttt{class}] Leitet die Definition einer Klasse ein.
        \item[\texttt{finally}] Leitet einen finally-Block ein.
        \item[\texttt{long}] Datentyp.
        \item[\textit{\texttt{strictfp}}] \textit{Legt fest, dass innerhalb einer Methode/einer Klasse nur strikte mathematische Operationen verwendet werden, sodass diese nicht optimiert werden sollen (es wird sich strikt an den Standard gehalten).}
        \item[\textit{\texttt{volatile}}] \textit{Markiert ein Attribut, sodass Modifikationen an diesem atomar durchgeführt werden und für andere Threads direkt sichtbar sind.}
        \item[\texttt{const}] Reserviert. Löst ausschließlich einen Compilefehler aus.
        \item[\texttt{float}] Datentyp.
        \item[\textit{\texttt{native}}] \textit{Markiert die Implementierung einer Methode als nativ, das heißt, die Implementierung liegt in nativem Code (C/C++) vor. Siehe JNI (Java Native Interface).}
        \item[\texttt{super}] Referenz auf die Instanz der Oberklasse des aktuellen Objektes.
        \item[\texttt{while}] Leitet eine while-Schleife ein.
	\end{description}
	
	Die genaue Bedeutung der obigen Schlüsselwörter werden wir in den jeweiligen Kapiteln genauer betrachten.
% end

\subsection{Bezeichner und Konventionen}
	\implements{Bezeichnern und Namenskonventionen}{identifier}{Java}
	
	In Java können Zeichenketten als Bezeichner gelten, wenn sie folgenden Bedingungen genügen:
	\begin{itemize}
		\item Sie bestehen nur aus \texttt{a} bis \texttt{z}, \texttt{0} bis \texttt{9}, \texttt{\_} oder \texttt{\$}.
		\item Sie beginnen nur mit \texttt{a} bis \texttt{z}, \texttt{\_} oder \texttt{\$}.
	\end{itemize}
	
	Zur Benennung sind außerdem folgende Konventionen zu empfehlen:
	\begin{itemize}
		\item Namen von Klassen beginnen mit einem Großbuchstaben.
		\item Namen von Methoden/Parametern/Variablen/etc. beginnen mit einem Kleinbuchstaben.
		\item Namen von Klassen sollen Subjekte und Objekte sein. Beispiel: \enquote{\texttt{User}}
		\item Namen von Methoden sollen mit einem Verb beginnen. Beispiel: \enquote{\texttt{generateAccessToken}}
	\end{itemize}
	
	\info{Die oben Bedingungen, wann eine Zeichenkette als Bezeichner dienen kann, stellen Vereinfachungen dar. Streng genommen können alle Zeichen verwendet werden, für die die Methoden \texttt{Character.isJavaIdentifierStart(char)} bzw. \texttt{Character.isJavaIdentifierPart(char)} den Wert \texttt{true} ergeben. Damit \textit{wären} auch Bezeichner wie \enquote{\texttt{\(\Delta\Psi\)}} möglich.}
% end

\subsection{Operatoren}
	\implements{Operatoren}{lexOperatoren}{Java}
	
	In Java existieren die folgenden Operatoren, die genaue Bedeutung werden wir im Abschnitt \refImpl{operatoren}{Java} behandeln:
	\begin{table}[H]
		\centering
		\begin{tabular}{l | l}
			Kategorie & Ausprägungen \\
			\hline
			Arithmetische Verknüpfungen & \texttt{*}, \texttt{/}, \texttt{\%}, \texttt{+}, \texttt{-} \\
			Unäre Arithmetik            & \texttt{expr++}, \texttt{expr--}, \texttt{++expr}, \texttt{--expr}, \texttt{+expr}, \texttt{-expr} \\
			Logik                       & \texttt{!}, \texttt{\&\&}, \texttt{||}, \texttt{\textasciicircum}, \texttt{?:} \\
			Bitweise Logik              & \texttt{\textasciitilde}, \texttt{\&}, \texttt{|}, \texttt{\textasciicircum} \\
			Verschiebung (Shift)        & \texttt{<{}<}, \texttt{>{}>}, \texttt{>{}>{}>} \\
			Vergleiche                  & \texttt{<}, \texttt{>}, \texttt{<=}, \texttt{>=}, \texttt{==}, \texttt{!=}, \texttt{instanceof} \\
			Zuweisungen                 & \texttt{=}, \texttt{+=}, \texttt{-=}, \texttt{*=}, \texttt{/=}, \texttt{\%=}, \texttt{\textasciicircum=}, \texttt{|=}, \texttt{<{}<=}, \texttt{>{}>=}, \texttt{>{}>{}>=} \\
		\end{tabular}
		\caption{Java: Operatoren}
	\end{table}
% end

\subsection{Strukturierung des Codes, Packages und Imports}
	\implements{Paketen und Code-Strukturierung}{namespaces}{Java}
	
	\subsubsection{Kommentare}
		Es gibt drei verschiedene Arten von Kommentaren in Java:
		\begin{itemize}
			\item Einzeilige Kommentare \\ Der Kommentar ist nur in der aktuellen Zeile gültig.
			\item Blockkommentare \\ Der Kommentar ist gültig, bis der Kommentar explizit beendet wird (auch über Zeilenumbrüche hinweg).
			\item Javadoc \\ Dies ist eine besondere Form eines Blockkommentars, der so nur vor einer Methode, einem Feld oder einem Typ stehen kann (und, um es ganz genau zu nehmen, vor der \lstinline|package|-Deklaration in einer Datei \texttt{package-info.java}). Diese besondere Form von Kommentaren wird genutzt, um den Code zu Dokumentieren und anschließend eine HTML-Dokumentation daraus zu generieren. Wir werden dies im Abschnitt \refImpl{doku}{Java} über Javadoc intensiver anschauen.
		\end{itemize}
	% end
	
	\subsubsection{Whitespaces}
		Jegliche Whitespaces (Tab, Zeilenumbruch, Leerzeichen) sind in Java optional und dienen nur der Strukturierung des Codes. Trotz dass dies überflüssig ist, empfiehlt es sich, wenn wir unseren Code einrücken, gezielt Zeilenumbrüche setzen und so unseren Code lesbar machen.
	% end
	
	\subsubsection{Klammerung}
		In Java werden jegliche verfügbaren Klammern genutzt ((), {}, [], <>). Sie haben die folgenden Zwecke:
		\begin{description}
			\item[\texttt{()}] Klammerung von Ausdrücken (um die Operatorenpräzedenz festzulegen), Aufrufen von Methoden/Konstruktoren, Notwendig bei If-Ausdrücken, Switch-Case und Schleifen.
			\item[\texttt{\{\}}] Kennzeichnung von Codeblöcken (Klassen-/Methodendefinition, If-Ausdrücke, Schleifen, Switch-Case, \dots).
			\item[\texttt{[]}] Zugriff auf die Elemente eines Arrays und Erstellung von Arrays.
			\item[\texttt{<>}] Vergleichsoperatoren und Generics.
		\end{description}
	% end
	
	\subsubsection{Packages und Imports}
		Als übergeordnete Strukturierung unserer Klassen existiert das Konstrukt von \textit{Packages}, mit denen Klassen gruppiert und somit logisch zusammengefasst werden können. Beispielsweise können wir alle Klassen, die mit dem Zugriff auf eine Datenbank zu tun haben in ein Package \texttt{database} legen, und die Klassen, die mit dem User Interface zu tun haben in ein Package \texttt|ui|. Der Name eines Packages muss ein gültiger Bezeichner sein. Um eine Klasse in einem Package abzulegen, muss die Klasse zum einen mit der Zeile \lstinline|package /* Packagename */;| starten und außerdem im korrekten Ordner liegen (das heißt eine Klasse im Package \texttt{ui} muss in einem Ordner \texttt{ui} liegen).
		
		Damit entsteht eine logische Trennung und das Projekt wird übersichtlicher. Der \textit{voll-qualifizierte Klassen} ist dann der Name der Klasse mit dem vorangestellten Package-Namen. Hierdurch werden Kollisionen in der Klassenbenennung vermieden. Beispiel: Unsere Klasse \texttt{Connection} liegt im Package \texttt{database}. Dann ist der voll-qualifizierte Name dieser Klasse \texttt{database.Connection}. Nutzen wir innerhalb einer Klasse eine andere Klasse, die nicht im selben Package und nicht im Package \texttt{java.lang} liegt, so müssen wir entweder den voll-qualifizierten Klassennamen bei jeder Verwendung angeben oder die Klasse mit dem Ausdruck \lstinline|import /* Voll-qualifizierter Klassenname */;| importieren. Dann können wir die Klasse überall nutzen, als wäre sie im gleichen Package. Standardmäßig sind alle Klassen aus \lstinline{java.lang} importiert.
		
		Um Packages von Unterpackages zu trennen, können wir Punkte innerhalb des Package-Namen nutzen. Somit können wir beispielsweise Klassen, die mit der Datenbank und konkret mit MySQL zu tun haben, in ein Package \lstinline{database.mysql} legen.
		
		\paragraph{Konvention}
			Zur Vermeidung von Kollisionen ist es üblich, allen Packages innerhalb eines Projektes den umgekehrten Namen der Domain voranzustellen, für die das Projekt entwickelt wird (Bindestriche oder andere nicht-Java-konforme Zeichen werden dabei durch einen Unterstrich ersetzt, siehe §3.8 Java-Spezifikation). Danach folgt der Projektname.
			
			Bei der Entwicklung von Nabla für das Fachgebiet Algorithmik am Fachbereich Informatik an der TU Darmstadt sollte also der Packagename \lstinline|de.tu_darmstadt.informatik.algo.nabla| vorangestellt werden.
		% end
	% end
% end

% end

\section{Anweisungen} \functionalMark \imperativeMark \oopMark
	Schauen wir uns nun an, wie man Dinge in Java tut, also wie wir Anweisungen und Ausdrücke formulieren können.

\subsection{Variablen}
	\implements{Variablen}{variablen}{Java}
	
	Die allgemeine Syntax zur Deklaration einer Variablen ist:
	\begin{figure}[H]
		\centering
		\lstinline|<modifier> <typ> <name>;|
	\end{figure}
	Dabei ist \texttt{<modifier>} eine Reihe von Schlüsselwörtern, welche das Verhalten der Variablen modifizieren (genannt \enquote{Modifier}). Diese werden wir uns weiter unten genau anschauen. \texttt{<typ>} ist der Datentyp der Variablen (dies kann ein primitiver Datentyp aber auch ein Referenztyp sein). Der Name der Variablen wird mit \texttt{<name>} festgelegt.
	
	\subsubsection{Modifier}
		Für eine lokale Variable (das heißt eine Variable innerhalb eines Codeblocks oder als Parameter) existiert ausschließlich folgender Modifier:
		\begin{description}
			\item[\texttt{final}] Sorgt dafür, dass die Variable nur einmal zugewiesen werden kann (zum Beispiel direkt nach oder noch während der Deklaration). Wenn möglich sollte eine Variable immer als \lstinline|final| markiert werden, um versehentliches Überschreiben des Wertes zu verhindern.
		\end{description}
		Handelt es sich bei der Variablen um eine Instanz- oder Klassenvariable, sind zusätzlich folgende Modifier verfügbar:
		\begin{description}
			\item[\texttt{volatile}] Bei der Zuweisung der Variablen geschieht die Zuweisung \textit{atomar}. Dieser Modifier kann nicht mit \lstinline|final| modifiziert werden.
			\item[\texttt{transient}] Bei der Serialisierung einer Instanzvariablen wird dieses Feld nicht serialisiert.
			\item[\(\bullet\)] Sämtliche Sichtbarkeitsmodifizierer (siehe \ref{sec:visibility}).
		\end{description}
		Alle Modifier können wir mit kleinen Einschränkungen beliebig kombinieren.
		
		Beispiel: Eine Definition einer privaten Klassenvariable \texttt{timestamp}, die atomar Zugewiesen werden soll und nicht mit serialisiert werden soll sieht so aus:
		\begin{figure}[H]
			\centering
			\lstinline|private static transient volatile long timestamp;|
		\end{figure}
	% end
	
	% TODO: Schreiben
	%\subsubsection{Lokale Variablen, Konstanten, Attribute, Arraykomponenten}
	%	\todo{Schreiben}
	%% end
	
	\subsubsection{Null- und Defaultwerte}
		Klassenvariablen, die nicht \lstinline|final| sind, werden bestimmte Default-Werte zugewiesen (sofern die Variable nicht während der Deklaration direkt zugewiesen wird):
		\begin{table}[H]
			\centering
			\begin{tabular}{l | l}
				\textbf{Typ} & \textbf{Default-Wert} \\ \hline
				\lstinline|byte| & \lstinline|0| \\
				\lstinline|short| & \lstinline|0| \\
				\lstinline|int| & \lstinline|0| \\
				\lstinline|long| & \lstinline|0| \\
				\lstinline|float| & \lstinline|0.0F| \\
				\lstinline|double| & \lstinline|0.0| \\
				\lstinline|boolean| & \lstinline|false| \\
				\lstinline|char| & \lstinline|'\000'| (Null-Byte) \\
				\lstinline|Object| und Unterklassen & \lstinline|null| \\
			\end{tabular}
			\caption{Java: Defaultwerte}
		\end{table}
	% end
% end

\subsection{Zuweisungen}
	\implements{Zuweisungen}{zuweisungen}{Java}
	
	Um eine Variable zuzuweisen, wird folgender Ausdruck verwendet:
	\begin{figure}[H]
		\centering
		\lstinline|<variable> = <ausdruck>;|
	\end{figure}
	Dabei ist der linke Teil \texttt{<variable>} der Name der Variablen, welcher der Wert des Ausdrucks \texttt{<ausdruck>} zugewiesen wird. Der Ausdruck kann dabei beliebig komplex sein.
	
	Wie können den Wert auch zeitgleich mit der Deklaration zuweisen, die Syntax ist dann wie folgt:
	\begin{figure}[H]
		\centering
		\lstinline|<modifier> <typ> <name> = <ausdruck>;|
	\end{figure}

	Eine Besonderheit ist hier, dass der Ausdruck einer normalen Zuweisung den Wert der Zuweisung zurück gibt (das heißt es gilt \texttt{(<variable> = <ausdruck>) == <ausdruck>}).
% end

\subsection{Methodenaufrufe}
	\implements{Methodenaufrufen}{methodenNutzung}{Java}
	
	Der allgemeine Ausdruck, um eine Methode in Java aufzurufen ist:
	\begin{figure}[H]
		\centering
		\lstinline|<objekt>.<methodenname>([parameter], [parameter], ...)|
	\end{figure}
	Der Methodenname muss immer gegeben sein, ebenso wie das Objekt (beziehungsweise bei einer statischen Methode die Klasse), welches/welche das Objekt enthält. Die Parameter müssen gegeben sein, wenn die aufgerufene Methode dies fordert, es gibt aber auch Methoden, die keine Parameter erfordern.
	
	Wir können die Rückgabe der Methode auch einer Variablen zuweisen, die Syntax ist dann wie folgt:
	\begin{figure}[H]
		\centering
		\lstinline|<variable> = <objekt>.<methodenname>([parameter], [parameter], ...)|
	\end{figure}
	Dies ist nur möglich, wenn die Methode einen Rückgabetyp hat, das heißt der Rückgabetyp nicht \lstinline|void| ist.
% end

\subsection{Operatoren}
	\implements{Operatoren}{operatoren}{Java}
	
	\subsubsection{Arithmetik-Operatoren}
		Es existieren die folgenden arithmetischen Operatoren, die allesamt alle primitiven und numerischen Datentypen (\lstinline|byte|, \lstinline|short|, \lstinline|int|, \lstinline|long|, \lstinline|float|, \lstinline|double|) annehmen:
		\begin{table}[H]
			\centering
			\begin{tabular}{c | l | l}
				\textbf{Operator} & \textbf{Syntax}                & \textbf{Beschreibung}                                    \\ \hline
				\texttt{++}       & \texttt{a++}, \texttt{++a}     & \texttt{a} wird um 1 \textit{inkrementiert}.             \\
				\texttt{-{}-}     & \texttt{a-{}-}, \texttt{a-{}-} & \texttt{a} wird um 1 \textit{dekrementiert}.             \\
				\texttt{*}        & \texttt{a * b}                 & \texttt{a} und \texttt{b} werden \textit{multipliziert}. \\
				\texttt{/}        & \texttt{a / b}                 & \texttt{a} wird durch \texttt{b} \textit{dividiert}.     \\
				\texttt{\%}       & \texttt{a \% b}                & Es wird \( \texttt{a} \textbf{ mod } \texttt{b} \) berechnet (d.h. \( \texttt{a} - \big\lfloor \frac{\texttt{a}}{\texttt{b}} \big\rfloor \texttt{b} \)) (\textit{Modulo}). \\
				\texttt{+}        & \texttt{a + b}                 & \texttt{a} und \texttt{b} werden \textit{addiert}.       \\
				\texttt{-}        & \texttt{a - b}                 & \texttt{b} wird von \texttt{a} \textit{subtrahiert}.     \\
				\texttt{-}        & \texttt{-a}                    & Negiert das Vorzeichen von \texttt{a}.
			\end{tabular}
		\end{table}
		Bei den Inkrementierungs-/Dekrementierungs-Operatoren ist der Unterschied zwischen den Syntaxen \texttt{a++} und \texttt{++a} (beziehungsweise \texttt{a-{}-} und \texttt{-{}-a}), dass das Ergebnis von ersterem Ausdruck den Wert von \texttt{a} vor der Inkrementierung/Dekrementierung und \texttt{++a}/\texttt{-{}-a} den Wert nach der Inkrementierung/Dekrementierung als Ergebnis liefert (Postfix vs. Prefix Operatoren). Das bedeutet, dass \texttt{a++ == a}, \texttt{a-{}- == a}, \texttt{++a == a + 1} und \texttt{-{}-a == a - 1} gelten.
		
		\paragraph{Kommazahlen und Division}
			Eine Division wird immer als \textit{Ganzzahldivision} durchgeführt, wenn nicht mindestens einer der Parameter eine Fließkommazahl ist. Das bedeutet, dass Nachkommastellen nur berechnet werden, wenn mindestens einer der Parameter ein \lstinline|float| oder \lstinline|double| ist.
			
			Eine Ganzzahldivision von \(a\) und \(b\) entspricht \( \big\lfloor \frac{a}{b} \big\rfloor \), dass heißt, die Nachkommastellen werden abgeschnitten.
		% end
	% end
	
	\subsubsection{Logik- und Vergleichs-Operatoren}
		Es existieren die folgenden logischen Operatoren und Vergleichsoperatoren, die alle als Ergebnis ein \lstinline|boolean| zurück geben.
		\begin{table}[H]
			\centering
			\begin{tabular}{c | l | l | l}
				\textbf{Operator} & \textbf{Syntax}   & \textbf{Parametertyp} & \textbf{Beschreibung}                                         \\ \hline
				   \texttt{<}     & \texttt{a < b}    & primitive Zahl        & Ist \texttt{a} kleiner \texttt{b}?                            \\
				   \texttt{>}     & \texttt{a > b}    & primitive Zahl        & Ist \texttt{a} größer \texttt{b}?                             \\
				   \texttt{<=}    & \texttt{a <= b}   & primitive Zahl        & Ist \texttt{a} kleiner-gleich \texttt{b}?                     \\
				   \texttt{>=}    & \texttt{a >= b}   & primitive Zahl        & Ist \texttt{a} größer-gleich \texttt{b}?                      \\
				   \texttt{==}    & \texttt{a == b}   & Beliebig              & Ist \texttt{a} identisch zu \texttt{b}?                       \\
				   \texttt{!=}    & \texttt{a != b}   & Beliebig              & Ist \texttt{a} nicht identisch zu \texttt{b}?                 \\
				  \texttt{\&\&}   & \texttt{a \&\& b} & Wahrheitswert         & Verknüpft \texttt{a} und \texttt{b} mit einem logischem UND.  \\
				   \texttt{||}    & \texttt{a || b}   & Wahrheitswert         & Verknüpft \texttt{a} und \texttt{b} mit einem logischem ODER. \\
				   \texttt{\^}    & \texttt{a \^{} b} & Wahrheitswert         & Verknüpft \texttt{a} und \texttt{b} mit einem logischem XOR.  \\
				   \texttt{!}     & \texttt{!a}       & Wahrheitswert         & Negiert den Wahrheitswert von \texttt{a}
			\end{tabular}
		\end{table}
		\textit{Identisch} bedeutet für Zahlen, dass diese bis auf die letzte Nachkommastelle gleich sind. Für Objekte bedeutet dies, dass es ein und das selbe Objekt sind (das heißt, dass die Speicheradresse identisch ist). Eine Änderung auf \texttt{a} ändert somit auch \texttt{b}, wenn \texttt{a == b} gilt (nur bei Objekten!). Aufgrund dessen ist es auch nicht möglich, Strings mit \texttt{==} zu vergleichen, da dies bei Benutzereingaben oder ähnlichem immer \lstinline|false| liefern würde, da die Objekte nur den gleichen Inhalt haben und nicht identisch sind (siehe auch \ref{sec:equals_identity}).
	% end
	
	\subsubsection{Bitlogik-Operatoren}
		Die bitlogischen Operatoren können auf primitive Ganzzahlen (\lstinline|byte|, \lstinline|short|, \lstinline|int|, \lstinline|long|) angewendet werden. Diese wenden die üblichen logischen Verknüpfungen auf Bit-Ebene an, dass heißt, die Zahl wird in Binärdarstellung überführt und die Verknüpfung der Reihe nach auf jedes Bit einzeln angewendet (bei ungleich großen Datentypen werden die fehlenden Stellen bei dem kleineren mit Nullen aufgefüllt). Der Rückgabetyp entspricht immer dem größeren Datentyp. Es existieren die folgenden Operatoren:
		\begin{table}[H]
			\centering
			\begin{tabular}{c | l | l}
				\textbf{Operator} & \textbf{Syntax}      & \textbf{Beschreibung}                                                                               \\ \hline
				  \texttt{<{}<}   & \texttt{a <{}< b}    & Verschiebt die Bits von \texttt{a} um \texttt{b} Stellen nach links.                                \\
				  \texttt{>{}>}   & \texttt{a >{}> b}    & Verschiebt die Bits von \texttt{a} um \texttt{b} Stellen nach rechts.                               \\
				\texttt{>{}>{}>}  & \texttt{a >{}>{}> b} & Verschiebt die Bits von \texttt{a} um \texttt{b} Stellen nach rechts und behält das Vorzeichen bei. \\
				   \texttt{\&}    & \texttt{a \& b}      & Verknüpft die Bits von \texttt{a} und \texttt{b} mit einer UND-Verknüpfung.                         \\
				   \texttt{\^}    & \texttt{a \^{} b}    & Verknüpft die Bits von \texttt{a} und \texttt{b} mit einer XOR-Verknüpfung.                         \\
				   \texttt{|}     & \texttt{a | b}       & Verknüpft die Bits von \texttt{a} und \texttt{b} mit einer ODER-Verknüpfung.                        \\
				\texttt{\(\sim\)} & \texttt{\(\sim\)a}   & Negiert die Bits von \texttt{a}.
			\end{tabular}
		\end{table}
	% end
	
	\subsubsection{Spezielle Operatoren}
		Zusätzlich zu den oben genannten Operatoren gibt es noch die Operatoren \lstinline|new|, \lstinline|instanceof| und der Ternäre Operator, die etwas anders funktionieren.
		\begin{description}
			\item[\texttt{\color{lstkeywords} new}] Mit diesem Operator können neue Instanzen (Objekte) einer Klasse erstellt werden und die allgemeine Syntax lautet \lstinline|new <klasse>([parameter], [parameter], ...)|; diesen Operator werden wird im Abschnitt \ref{sec:constructor} genauer betrachten.
			\item[\texttt{\color{lstkeywords} instanceof}] Mit diesem Operator kann geprüft werden, ob ein Objekt eine Instanz einer bestimmten Klasse darstellt, die allgemeine Syntax hierfür lautet \lstinline|<objekt> instanceof <klasse>|. Beispielsweise wäre für eine Variable \lstinline|Number x = 1.2| der Ausdruck \lstinline|x instanceof Double| wahr, der Ausdruck \lstinline|x instanceof Integer| jedoch falsch.
			\item[Ternärer Operator] Mit diesem Operator können, ähnlich wie bei einem If, Fallunterscheidungen vorgenommen werden. Die allgemeine Syntax lautet \lstinline|<test> ? <wahr-fall> : <sonst-fall>|. Dabei wird zuerst der Test ausgewertet, ist dieser Wahr, so wird das Ergebnis von dem wahr-Fall zurück gegeben, sonst das Ergebnis von dem sonst-Fall. Dabei muss der Test zu einem Wahrheitswert auswerten und die beiden Fälle zu dem gleichen Typ, beziehungsweise einem kompatiblen Typ für den äußeren Ausdruck.
		\end{description}
	% end
	
	\subsubsection{Bindungsstärke der Operatoren}
		Die Bindungsstärke der Operatoren in Java gliedert sich wie folgt, wobei die oberste Zeile die stärkste Bindungsstärke hat und mehrere Elemente auf einer Zeile die gleiche Bindungsstärke:
		\begin{enumerate}
			\item \texttt{expr++}, \texttt{expr--}
			\item \texttt{++expr}, \texttt{--expr}, \texttt{+expr}, \texttt{-expr}, \texttt{\(\sim\)}, \texttt{!}
			\item \texttt{*}, \texttt{/}, \texttt{\%}
			\item \texttt{+}, \texttt{-}
			\item \texttt{<{}<}, \texttt{>{}>}, \texttt{>{}>{}>}
			\item \texttt{<}, \texttt{>}, \texttt{<=}, \texttt{>=}, \texttt{instanceof}
			\item \texttt{==}, \texttt{!=}
			\item \texttt{\&}
			\item \texttt{\^}
			\item \texttt{|}
			\item \texttt{\&\&}
			\item \texttt{||}
			\item \texttt{? :}
			\item \texttt{=}, \texttt{+=}, \texttt{-=}, \texttt{*=}, \texttt{/=}, \texttt{\%=}, \texttt{\&=}, \texttt{\^{}=}, \texttt{|=}, \texttt{<{}<=}, \texttt{>{}>=}, \texttt{>{}>{}>=}
		\end{enumerate}
	% end
	
	\subsubsection{Klammerung}
		Um die Bindungsstärke von Operatoren zu beeinflussen, können Ausdrücke wie in der Mathematik geklammert werden, wobei die innerste Klammer immer zuerst ausgewertet wird. Hierfür dürfen ausschließlich runde Klammern (\texttt{(}, \texttt{)}) genutzt werden.
	% end
% end

\subsection{Implizite und Explizite Typenkonversion (Casts)}
	Schauen wir uns zuerst einmal an, was wir unter einer Typenkonversion verstehen: Wenn wir eine Variable \lstinline|int a = 41| haben, können wir diese Problemlos einer anderen Variable mit dem Datentyp \lstinline|long| zuweisen (\lstinline|long b = a|). Hier liegt uns eine \textit{implizite Typenkonversion} vor, bei der der Datentyp \lstinline|int| zu einem \lstinline|long| umgewandelt wird. Wir gehen nun getrennt auf primitive Typenkonversionen, Wrapper-Typen und Objektkonversionen ein.
	
	\subsubsection{Primitive Typen}
		Eine primitive Typenkonversion haben wir bereits gesehen. Eine implizite Typenkonversion ist immer dann möglich, wenn der neue Datentyp eine größere oder gleiche Datenmenge halten kann wie der alte Datentyp (das heißt es ist zum Beispiel nicht implizit möglich, eine Fließkommazahl in eine Ganzzahl zu konvertieren).
		
		\begin{figure}[H]
			\centering
			\begin{tikzpicture}[main/.style = { draw, rectangle, minimum height = 0.9cm, minimum width = 2cm }]
				\node [main] (byte) {\lstinline|byte|};
				\node [main, right = 2 of byte] (short) {\lstinline|short|};
				\node [main, right = 2 of short] (int) {\lstinline|int|};
				\node [main, right = 2 of int] (long) {\lstinline|long|};
				\node [main, below = 2 of int] (float) {\lstinline|float|};
				\node [main, right = 2 of float] (double) {\lstinline|double|};
				\node [main, above = 2 of short] (char) {\lstinline|char|};
				
				\draw [->] (char) -| (int);
				
				\draw [->] (byte) -- (short);
				\draw [->] (short) -- (int);
				\draw [->] (int) -- (long);
			
				\coordinate [below = 1 of long] (needle);
				\draw (long) -- (needle);
				\draw [->] (needle) -| (float);
				
				\draw [->] (float) -- (double);
			\end{tikzpicture}
		\end{figure}
		Der Pfeil \( A \rightarrow B \) bedeutet, dass \(A\) implizit in \(B\) konvertiert werden kann. Der Rückweg ist ausgeschlossen. Außerdem ist die Konvertierung transitiv, dass bedeutet, wenn \( A \rightarrow B \) und \( B \rightarrow C \), dann geht auch \( A \rightarrow C \).
		
		Eine explizite Konvertierung wird vorgenommen, indem der neue Typ in Klammern vor die Variable (oder den Ausdruck) des alten Typs gesetzt wird:
		\begin{figure}[H]
			\centering
			\lstinline|(<neuer-typ>) <ausdruck>|
		\end{figure}
		Beispielsweise Wertet der Ausdruck \( 1 / 2.0 \) zu einem \lstinline|double| aus und das Ergebnis muss explizit in ein \lstinline|int| konvertiert werden: \lstinline|(int) (1 / 2.0)|. Das Ergebnis wäre in diesem Falle \lstinline|0|, da bei einer Typenkonvertierung von einer Fließkommazahl in eine Ganzzahl die Nachkommastellen abgeschnitten werden.
	% end
	
	\subsubsection{Wrappertypen}
		Wie wir im Abschnitt zu Generics (\ref{sec:generics}) sehen werden, sind primitive Typen nicht immer hilfreich. Manchmal möchten wir auch Zahlen oder ähnliches in Objekten speichern können. Hier kommen die sogenannten \textit{Wrappertypen} ins Spiel, die ebenso wir Strings immutable, das heißt nicht veränderlich, sind.
		
		Wrappertypen sind Klassen, die eine primitive Variable speichern und diese bei Bedarf zur Verfügung stellt. Die Verwendung dieser Wrapper Typen erfolgt durch \textit{Autoboxing} transparent, das heißt, eine Variable wird automatisch in einem Wrappertyp gespeichert und gelesen.
		
		Die Namen der Wrappertypen entsprechen zu großen Teilen dem Namen des primitiven Typs mit einem großem Anfangsbuchstaben (die Klassen liegen allesamt in dem Package \lstinline|java.lang|):
		\begin{table}[H]
			\centering
			\begin{tabular}{l | l}
				\textbf{Primitiver Typ} & \textbf{Wrappertyp}   \\ \hline
				\lstinline|byte|        & \lstinline|Byte|      \\
				\lstinline|short|       & \lstinline|Short|     \\
				\lstinline|int|         & \lstinline|Integer|   \\
				\lstinline|long|        & \lstinline|Long|      \\
				\lstinline|float|       & \lstinline|Float|     \\
				\lstinline|double|      & \lstinline|Double|    \\
				\lstinline|char|        & \lstinline|Character| \\
				\lstinline|boolean|     & \lstinline|Boolean|
			\end{tabular}
		\end{table}
	
		\paragraph{Autoboxing}
			Weisen wir einer Variable \lstinline|Object obj| einen primitiven Wert (zum Beispiel \lstinline|1.2|) zu, so wird dieser primitive Typ automatisch in den entsprechenden Wrappertyp konvertiert und der Variable zugewiesen. Ebenfalls wird an Stellen, an denen primitive Typen gebraucht werden (zum Beispiel in arithmetischen Operationen oder Vergleichen) der Wrappertyp zurück in einen primitiven Wert gewandelt.
			
			Beispiel:
			\begin{figure}[H]
				\centering
				\begin{lstlisting}
double primitive = 1.2;
int wholeNumber = (int) x;
Double wrapper = primitive;   // Autoboxing.
if (wrapper > wholeNumber) {  // Autounboxing.
	...
}
\end{lstlisting}
			\end{figure}
		
			\warning{Im Gegensatz zu primitiven Typen können Variablen von Wrappertypen \lstinline|null| sein. Wird versucht, Autounboxing auf \lstinline|null|-Werten anzuwenden, so wird eine \lstinline|NullPointerException| geworfen.}
		% end
	% end
	
	\subsubsection{Objekte (\enquote{Downcast})}
		\label{sec:downcast}
	
		Auch bei Objekten müssen wir manchmal eine Typenkonvertierung vornehmen. Implizite Typenkonvertierungen sind hier genau dann möglich, wenn der neue Typ eine Oberklasse des alten Typs ist. Eine explizite Typenkonvertierung wird benötigt, wenn in der Klassenhierarchie \enquote{nach unten} gegangen werden soll (dies wird \textit{Downcast} genannt). Eine explizite Typenkonvertierung findet wie bei primitiven Typen statt indem der neue Typ in Klammern vor den Ausdruck geschrieben wird.
		
		Schauen wir uns dies am Beispiel eines Strings an:
		\begin{figure}[H]
			\centering
			\begin{tikzpicture}
				\umlemptyclass{Object}
				\umlemptyclass[below = 1 of Object]{CharSequence}
				\umlemptyclass[below = 1 of CharSequence]{String}
				
				\umlinherit{String}{CharSequence}
				\umlinherit{CharSequence}{Object}
			\end{tikzpicture}
		\end{figure}
		Eine implizite Typenkonvertierung ist nun immer nach oben in der Hierarchie möglich (also \( \texttt{String} \rightarrow \texttt{CharSequence} \rightarrow \texttt{Object} \)).
		
		Beispiel:
		\begin{figure}[H]
			\centering
			\begin{lstlisting}
String s = "Hello, World!";
Object o = s;                // Implizite Typenkonvertierung.
String casted = (String) o;  // Explizite Typenkonvertierung (Downcast).
\end{lstlisting}
		\end{figure}
	% end
% end

% TODO: Schreiben
%\subsection{Links-/Rechtsausdrücke}
%	\todo{Schreiben}
%% end

% TODO: Schreiben
%\subsection{Seiteneffekte}
%	\todo{Schreiben}
%% end

% end

\section{Kontrollstrukturen} \functionalMark \imperativeMark \oopMark
	\subsection{Verzweigungen}
	\implements{Verzweigungen}{verzweigungen}{Java}

	In Java gibt es als grundlegende Art der Verzweigung nur das einfache If. Ein Switch, welches wir uns später anschauen werden, baut sehr direkt auf einem If auf reduziert größtenteils die Tipparbeit.

	\subsubsection{If}
		In \texttt{if} hat in Java die folgende Form:
		\begin{figure}[H]
			\centering
			\begin{lstlisting}
if (/* Test */) {
	/* then-Fall */
} else if (/* else-Test */) {
	/* else-then-Fall */
} else {
	/* else-Fall */
}
			\end{lstlisting}
			\caption{Java: \texttt{if}-Verzweigung}
		\end{figure}
		
		Dabei kann es beliebig viele Else-Ifs geben, oder diese können ganz weg gelassen werden. Ebenfalls kann der Else-Fall weggelassen werden, einzig und allein der Then-Fall ist nötig (dieser kann theoretisch auch leer sein, dies ergibt aber in den meisten Fällen keinen Sinn).
		
		Somit ist die einfache Form des \texttt{if}s:
		\begin{figure}[H]
			\centering
			\begin{lstlisting}
if (/* Test */) {
	/* then-Fall */
}
			\end{lstlisting}
			\caption{Java: Einfache \texttt{if}-Verzweigung}
		\end{figure}
		
		Alle Tests (die Bedingungen für das If) \textit{müssen} zu einem \texttt{boolean} auswerten. Alle anderen Datentypen werden nicht akzeptiert und der Code wird nicht kompilieren.
	% end
	
	\subsubsection{Switch}
		Ein \texttt{switch} stellt eine Vereinfachung von vielen If-Else-Verzweigungen dar, welche alle die gleiche Operation (beispielsweise das Vergleichen von zwei Objekten) ausführen.
		
		Schauen wir uns als Motivation den folgenden Code an, welcher ausgibt, wie viele Primzahlen \texttt{p} mit \texttt{p <= x <= 10} existieren.
		\begin{figure}[H]
			\centering
			\begin{lstlisting}
if (x == 1) {
	prime = 0;
} else if (x == 2) {
	prime = 1;
} else if (x == 3) {
	prime = 2;
} else if (x == 4) {
	prime = 2;
} else if (x == 5) {
	prime = 3;
} else if (x == 6) {
	prime = 3;
} else if (x == 7) {
	prime = 4;
} else if (x == 8) {
	prime = 4;
} else if (x == 9) {
	prime = 4;
} else if (x == 10) {
	prime = 4;
} else {
	// Error.
}
			\end{lstlisting}
			\caption{Java: \texttt{switch} Motivation}
		\end{figure}
		
		Mit Hilfe eines Switches können wir den Code nun äquivalent umformen:
		\begin{figure}[H]
			\centering
			\begin{lstlisting}
switch (x) {
case 1:
	prime = 0;
	break;
case 2:
	prime = 1;
	break;
case 3:
	prime = 2;
	break;
case 4:
	prime = 2;
	break;
case 5:
	prime = 3;
	break;
case 6:
	prime = 3;
	break;
case 7:
	prime = 4;
	break;
case 8:
	prime = 4;
	break;
case 9:
	prime = 4;
	break;
case 10:
	prime = 4;
	break;
default:
	// Error.
	break;
}
			\end{lstlisting}
			\caption{Java: \texttt{switch}}
		\end{figure}
		
		Der Code ist äquivalent zu der If-Else-Kaskade und ist einfacher zu verstehen. Allerdings müssen wir in jedem Case (ein einzelner Fall in dem Switch-Konstrukt) ein \texttt{break} platzieren, welches die Ausführung des Switches abbricht. Wird das \texttt{break} weggelassen, so \textit{fällt die Ausführung durch}, das heißt es wird einfach mit dem nächsten Case fortgefahren.
		
		Um den Nutzen hiervon verstehen zu können müssen wir uns darüber im klaren sein, wie ein Switch ausgewertet wird:
		\begin{itemize}
			\item Im ersten Schritt wird die \textit{Vergleichsvariable} in den Klammern hinter dem Schlüsselwort \texttt{switch} ausgewertet. Diese Variable darf nur zu einem String, einem primitiven Wert oder dem Wert eines Enums auswerten.
			\item Anschließend wird der erhaltene Wert mit dem Wert eines jeden Cases verglichen. Hierfür ist es nötig, das hinter dem Schlüsselwort \texttt{case} ausschließlich Literale oder Konstanten mit dem gleichen Typ stehen. Es kann niemals zwei Cases mit dem gleichen Wert (auch genannt \textit{Label}) geben!
			\item Nun wird zur ersten Zeile des Cases gesprungen, dessen Wert gleich dem Vergleichswert ist. Existiert kein solcher Case, so wird zu dem Default-Case gesprungen, welcher aber nicht existieren muss. Wird kein Code zur Ausführung gefunden, so wird das gesamte Switch übersprungen.
			\item Sämtlicher nachfolgender Code wird nun ausgeführt, bis ein \texttt{break} gefunden wird. Dann wird aus dem Switch gesprungen un nach dem Switch fortgefahren. Außerdem beenden Returns und Exceptions wie üblich die Ausführung.
		\end{itemize}
		
		Das heißt: Beim Start des Cases wird zu einer Stelle im Code gesprungen und dieser so lange ausgeführt, bis die Ausführung \textit{explizit} beendet wird oder kein Code mehr im Switch existiert.
		
		Mit dieser Kenntnis kann obiger Code des Switches deutlich vereinfacht werden, indem wir einfach bei jeder Primzahl den Zähler um eins erhöhen:
		\begin{figure}[H]
			\centering
			\begin{lstlisting}
prime = 0;
switch (x) {
	case 10:
	case 9:
	case 8:
	case 7:
		prime++;
	case 6:
	case 5:
		prime++;
	case 4:
	case 3:
		prime++;
	case 2:
		prime++;
	case 1:
		break;
	default:
		// Error.
		break;
}
			\end{lstlisting}
			\caption{Java: \texttt{switch} mit Fall-Thru}
		\end{figure}
		
		\warning{Ein Case in einem Switch öffnet keinen neuen Scope! Somit können Variablen nur einmal genutzt werden oder es muss ein Block um den Case geschrieben werden.}
	% end
% end

\subsection{Schleifen}
	\implements{Schleifen}{schleifen}{Java}
	
	\subsubsection{While-Schleife}
		In Java sieht die zuvor vorgestellte While-Schleife wie folgt aus:
		\begin{figure}[H]
			\centering
			\begin{lstlisting}
while (/* Test */) {
	/* Code */
}
			\end{lstlisting}
			\caption{Java: \texttt{while}-Schleife}
		\end{figure}
		
		Wie auch schon beim If gesehen, darf auch hier der Test nur zu einem \texttt{boolean} und nicht zu anderen Datentypen ausgewertet werden. Der Test wird \textit{vor} jedem Schleifendurchlauf ausgeführt und bricht ab, sobald er zu \texttt{false} auswertet.
	% end
	
	\subsubsection{Do-While-Schleife}
		Als Spezialfall einer While-Schleife gibt es in Java die Do-While-Schleife, welche immer \textit{mindestens einmal} ausgeführt wird:
		\begin{figure}[H]
			\centering
			\begin{lstlisting}
do {
	/* Code */
} while (/* Test */);
			\end{lstlisting}
			\caption{Java: \texttt{do-while} Schleife}
		\end{figure}
		
		Wie bei allen Schleifen und Bedingungen darf auch hier der Test nur zu einem \texttt{boolean} auswerten. Im Gegensatz zur While-Schleife wird der Test hier allerdings \textit{nach} jedem Schleifendurchlauf ausgeführt, wodurch die Schleife immer mindestens einmal ausgeführt wird.
	% end
	
	\subsubsection{For-Schleife}
		Da oftmals über Elemente einer Liste oder eines Arrays iteriert wird und der Code hierfür immer gleich ist (eine Zählvariable wird in jedem Schritt hochgezählt):
		\begin{figure}[H]
			\centering
			\begin{lstlisting}
int i = 0;
while (i < array.length) {
	/* Code */

	i++;
}
			\end{lstlisting}
			\caption{Java: \texttt{for each}-Schleife Motivation}
		\end{figure}
		kann dieser Code zu folgendem, äquivalentem, Code umgewandelt werden:
		\begin{figure}[H]
			\centering
			\begin{lstlisting}
for (int i = 0; i < array.length; i++) {
	/* Code */
}
			\end{lstlisting}
			\caption{Java: \texttt{for each}-Schleife Motivation}
		\end{figure}
		womit Code eingespart wird und die Iteration deutlich übersichtlicher ist.
		
		Die einzelnen Bestandteile des Schleifenkopfes, welche mit Semikola getrennt werden müssen, haben folgende Namen und Funktionen:
		\begin{description}
			\item[\texttt{int i = 0}] \textit{Initialisierung} - Der Code an dieser Stelle wird \textit{einmalig vor} Durchlauf der Schleife ausgeführt.
			\item[\texttt{i < array.length}] \textit{Test} - Ein zu \texttt{boolean} auswertender Ausdruck, welcher \textit{vor jedem} Durchlauf ausgewertet wird. Wird die Bedingung zu \texttt{false} ausgewertet, wird die Schleife beendet.
			\item[\texttt{i++}] \textit{Schritt} - Der Code an dieser Stelle wird \texttt{nach jedem} Durchlauf ausgeführt.
		\end{description}
		
		Ferner können alle Bestandteile der For-Schleife ausgelassen werden (unter Beibehaltung der Semikola!), wobei Initialisierung und Schritt einfach nicht ausgeführt werden und die Bedingung immer zu \texttt{true} auswertet. Somit sind \enquote{\texttt{while (true) \{ \}}} und \enquote{\texttt{for (;;) \{ \}}} äquivalent.
	% end
	
	\subsubsection{\enquote{Erweiterte} For-Schleife}
		Statt wie üblich über Arrays und Listen zu iterieren:
		\begin{figure}[H]
			\centering
			\begin{lstlisting}
for (int i = 0; i < array.length; i++) {
	Object element = array[i];

	/* Code */
}
			\end{lstlisting}
			\caption{Java: Erweiterte For-Schleife Motivation}
		\end{figure}
		kann seit Java 5 auch die \textit{erweiterte For-Schleife} verwendet werden, um über Arrays oder Instanzen von \texttt{java.util.Iterable} iterieren (alle Standard-Klassen für Listen implementieren dieses Interface):
		\begin{figure}[H]
			\centering
			\begin{lstlisting}
for (Object element : array) {
	/* Code */
}
			\end{lstlisting}
			\caption{Java: Erweiterte For-Schleife}
		\end{figure}
		Dabei wird immer über den konkreten Typ, der im Array (oder der Liste) gespeichert ist, iteriert.
		
		Anders ausgedrückt: Der Code im Schleifenkörper wird für jedes Element des Arrays oder der Liste ausgeführt.
	% end
	
	\subsubsection{\texttt{break}, \texttt{continue}}
		Um eine Schleifenausführung vorzeitig auszuführen, gibt es die folgenden Schlüsselwörter:
		\begin{description}
			\item[\texttt{break}] Bricht die gesamte Schleifenausführung der innerstmöglichen Schleife ab.
			\item[\texttt{continue}] Fährt mit der nächsten Iteration der innerstmöglichen Schleife fort.
		\end{description}
		
		\paragraph{Beispiel}
			Schauen wir uns abschließend folgendes schwachsinniges Beispiel an, um die Funktionalität von \texttt{break} und \texttt{continue} zu verdeutlichen. Der folgende Code summiert die ungeraden Elemente eines Arrays, wobei keine weiteren Element aufsummiert werden, sobald die Summe einmal \texttt{10} überschritten hat.
			\begin{figure}[H]
				\centering
				\begin{lstlisting}
int sumOdd = 0;
for (int x : array) {
	if (x % 2 == 0) {
		// Element is even --> continue with next element.
		continue;
	}

	if (sumOdd > 10) {
		// Sum is over 10 --> stop loop.
		break;
	}

	sumOdd += x;
}
System.out.println(sumOdd);
				\end{lstlisting}
				\caption{Java: \texttt{break}, \texttt{continue} Beispiel}
			\end{figure}
			Mit den Werten \texttt{array = new int[] \{ 1, 2, 3, 4, 5, 7, 8, 9 \}} wird \texttt{16} ausgegeben, wobei der Code wie folgt ausgeführt wird:
			\begin{figure}[H]
				\centering
				\begin{lstlisting}
(Zeile =  1; sumOdd =  0)

(Zeile =  2; sumOdd =  0; x = 1)
(Zeile =  3; sumOdd =  0; x = 1; (x % 2 == 0) = false)
(Zeile =  8; sumOdd =  0; x = 1; (sumOdd > 10) = false)
(Zeile = 13; sumOdd =  1; x = 1)

(Zeile =  2; sumOdd =  1; x = 2)
(Zeile =  3; sumOdd =  1; x = 2; (x % 2 == 0) = true)

(Zeile =  2; sumOdd =  1; x = 3)
(Zeile =  3; sumOdd =  1; x = 3; (x % 2 == 0) = false)
(Zeile =  8; sumOdd =  1; x = 3; (sumOdd > 10) = false)
(Zeile = 13; sumOdd =  4; x = 3)

(Zeile =  2; sumOdd =  4; x = 4)
(Zeile =  3; sumOdd =  4; x = 4; (x % 2 == 0) = true)

(Zeile =  2; sumOdd =  4; x = 5)
(Zeile =  3; sumOdd =  4; x = 5; (x % 2 == 0) = false)
(Zeile =  8; sumOdd =  4; x = 5; (sumOdd > 10) = false)
(Zeile = 13; sumOdd =  9; x = 5)

(Zeile =  2; sumOdd =  9; x = 6)
(Zeile =  3; sumOdd =  9; x = 6; (x % 2 == 0) = true)

(Zeile =  2; sumOdd =  9; x = 7)
(Zeile =  3; sumOdd =  9; x = 7; (x % 2 == 0) = false)
(Zeile =  8; sumOdd =  9; x = 7; (sumOdd > 10) = false)
(Zeile = 13; sumOdd = 16; x = 7)

(Zeile =  2; sumOdd = 16; x = 8)
(Zeile =  3; sumOdd = 16; x = 8; (x % 2 == 0) = true)

(Zeile =  2; sumOdd = 16; x = 9)
(Zeile =  3; sumOdd = 16; x = 9; (x % 2 == 0) = false)
(Zeile =  8; sumOdd = 16; x = 9; (sumOdd > 10) = true)

(Zeile = 15; sumOdd = 16)
				\end{lstlisting}
				\caption{Java: \texttt{break}, \texttt{continue} Beispielausführung}
			\end{figure}
		% end
	% end
% end

% end

\section{Methoden/Funktionen} \functionalMark \imperativeMark \oopMark
	\todo{Schreiben}

% Mit steigender Komplexität der Programme werden wir sehen, dass sich oftmals viele Stellen im Code doppelt, dreifach oder noch öfter zu finden sind. Auch wird ersichtlich, dass ein Programm, welches aus $ >200 $ Zeilen besteht, nicht mehr übersichtlich ist.

%Hier können \textit{Methoden} helfen, welche den Code strukturieren.

% end

\section{Scoping} \functionalMark \imperativeMark \oopMark
	\implements{Scoping und Scopes}{scoping}{Java}

In Java wird immer dann ein neuer Scope geöffnet, wenn eine geschweifte Klammer auf geht und ein Scope geschlossen, wenn die geschweifte Klammer zu geht. Das ist zum Beispiel bei Methoden und Schleifen der Fall.

Das bedeutet: Eine Variable, die wir innerhalb einer Methode definieren (inklusive der Parameter) ist von außerhalb nicht zugreifbar. Das gleiche gilt für Variablen, die innerhalb eines If-Blocks oder dem Körper einer Schleife definiert wurden.

\textbf{Beispiel:} \\
\begin{lstlisting}
int multiply(int a, int b) {
	int result = 0;

	int bAbs = Math.abs(b);
	for (int i = 0; i < Math.abs(b); i++) {
		result += a;
	}

	if (b < 0) {
		return -result;
	} else {
		return result;
	}
}
\end{lstlisting}
Auf die Variablen \texttt{a}, \texttt{b} und \texttt{result} kann von außerhalb der Methode nicht zugegriffen werden. Ebenfalls kann nicht auf \texttt{i} zugegriffen werden, dies ist aber schon außerhalb der Schleife (also in Zeile 1 bis 4 und 8 bis 14) nicht möglich.

Der Wert der Variable \texttt{result} wird zurück gegeben, womit der Aufrufer Zugriff auf den Wert der Variable hat, aber ohne Kenntnis darüber, wie das Ergebnis zustande gekommen ist.
% end

\section{Klassen und objektorientierte Programmierung} \oopMark
	% Ein Konstruktor ist eine besondere Methode innerhalb einer Klasse, welche keinen expliziten Rückgabetyp besitzt und den gleichen Namen trägt wie die Klasse. In dieser Methode werden alle zur Initialisierung der Klasse nötigen Aktionen vollführt, zum Beispiel Objektvariablen belegen. Jede Klasse besitzt einen Default-Konstruktor, welcher keine Parameter annimmt

\subsection{Klassen, Objekte und Methoden}
	\todo{Schreiben}
	
	\subsubsection{Statische Methoden und Attribute}
		\todo{Schreiben}
	% end
	
	\subsubsection{Sichtbarkeit}
		\todo{Schreiben}
	% end
	
	\subsubsection{Abgrenzung: Objektvariable \(\leftrightarrow\) Objektkonstante \(\leftrightarrow\) Klassenvariable \(\leftrightarrow\) Klassenkonstante}
		\todo{Schreiben}
	% end
	
	\subsubsection{Abgrenzung: Objektmethode \(\leftrightarrow\) Klassenmethode}
		\todo{Schreiben}
	% end
	
	\subsubsection{Konstruktoren}
		\todo{Schreiben}
		
		\paragraph{Überladen von Konstruktoren}
			\todo{Schreiben}
		% end
		
		\paragraph{\texttt{this}}
			\todo{Schreiben}
		% end
		
		\paragraph{\texttt{super}}
			\todo{Schreiben}
		% end
	% end
	
	\subsubsection{Initializer-Block}
		\todo{Schreiben}
	% end
	
	\subsubsection{Static-Initializer-Block}
		\todo{Schreiben}
	% end
% end

\subsection{Referenzen}
	\todo{Schreiben}
	
	\subsubsection{Vergleich zu primitiven Daten}
		\todo{Schreiben}
	% end
	
	\subsubsection{Literal \texttt{null}}
		\todo{Schreiben}
	% end
	
	\subsubsection{Sonderfall \texttt{String}}
		\todo{Schreiben}
	% end
	
	\subsubsection{Zuweisen vs. Kopieren}
		\todo{Schreiben}
	% end
	
	\subsubsection{Test auf Gleichheit und Identität}
		\label{sec:equals_identity}
	
		\todo{Schreiben}
	% end
	
	\subsubsection{Downcasts}
		\todo{Schreiben}
	% end
% end

\subsection{Vererbung}
	\todo{Schreiben}
	
	\subsubsection{Methoden}
		\todo{Schreiben}
	% end
	
	\subsubsection{Variation der Sichtbarkeit}
		\todo{Schreiben}
	% end
	
	\subsubsection{Variation von Rückgabetyp und Exceptions}
		\todo{Schreiben}
	% end
	
	\subsubsection{Attribute}
		\todo{Schreiben}
	% end
	
	\subsubsection{Finale Klassen}
		\todo{Schreiben}
	% end
% end

\subsection{Abstrakte Klassen}
	\todo{Schreiben}
% end

\subsection{Interfaces}
	\todo{Schreiben}
	
	\subsubsection{Default-Methoden}
		\todo{Schreiben}
	% end
	
	\subsubsection{Funktionale Interfaces}
		\todo{Schreiben}
		
		\paragraph{Interfaces in \texttt{java.util.function}}
			\todo{Schreiben}
		% end
	% end
% end

\subsection{Polymorphie und späte Bindung}
	\todo{Schreiben}
	
	% statischer/dynamischer Typ
% end

\subsection{Verschachtelte Klassen}
	\todo{Schreiben}
	
	\subsubsection{Statische verschachtelte Klassen}
		\todo{Schreiben}
	% end
	
	\subsubsection{Innere Klassen}
		\todo{Schreiben}
	% end
	
	\subsubsection{Anonyme Innere Klassen}
		\todo{Schreiben}
	% end
% end

\subsection{Lambda-Ausdrücke}
	\todo{Schreiben}
	
	\subsubsection{Methoden-Referenzen}
		\todo{Schreiben}
	% end
% end

\subsection{Enumerations}
	\todo{Schreiben}
	
	\subsubsection{Klasse \texttt{java.lang.Enum}}
		\todo{Schreiben}
	% end
	
	\subsubsection{Vererbung}
		\todo{Schreiben}
	% end
% end

\subsection{Metadaten}
	\todo{Schreiben}
	
	% Zur Klasse
	% Zu den Attributen
	% Zu den Methoden
	% Methodentabelle
% end

\subsection{Speicherverwaltung}
	\todo{Schreiben}
% end

% end

\section{Fehlerbehandlung} \functionalMark \imperativeMark \oopMark
	\introduces{von Fehlerbehandlung}{fehlerbehandlung}

Während der Ausführung von Code kann es immer zu Fehler kommen, welche es zu behandeln gilt (beispielsweise bei einer Division durch \text{0}). Hierbei stellt sich die Frage, wie wir dem Nutzer (oder dem Aufrufer einer Methode) erkenntlich machen, dass es zu einem Fehler gekommen ist.

Dabei gibt es unterschiedliche Möglichkeiten, welche sich grob wie folgt einteilen lassen:
\begin{itemize}
	\item Exceptions und
	\item Result Codes.
\end{itemize}

Diese zwei Unterarten werden wir in den folgenden Abschnitten behandeln.

\subsection{Result Code}
	\introduces{von Result Codes}{resultcodes}

	Beschäftigen wir uns zuerst mit der einfachsten Methode, Fehler anzuzeigen: Wir sagen dem Aufrufer über den Rückgabewert der Methode Bescheid, ob alles korrekt abgelaufen ist.
	
	An einem konkreten Beispiel heißt dies:
	\begin{itemize}
		\item Szenario: Wir haben eine Methode \texttt{indexOf(val: String, el: char): int}, welche uns die Position des ersten Vorkommens von \texttt{el} in \texttt{val} zurück gibt. \\ Beispiel: \texttt{indexOf("asdfgas", 's')} gibt \texttt{1} zurück.
		\item Ist das Zeichen nicht in dem String vorhanden, stellt dies einen Fehler dar.
		\item Mit Fehlermeldung über Result Codes könnten wir nun beispielsweise \texttt{-1} zurück geben, da dies kein valider Index ist (welche immer \texttt{\(\geq\) 0} sein müssen).
		\item Damit sieht der Aufrufer, dass der Methodenaufruf schief gegangen ist und kann entsprechend reagieren.
	\end{itemize}
	
	\paragraph{Vorteile}
		\begin{itemize}
			\item Es werden keine expliziten Verfahren zum Melden von Fehlern benötigt.
			\item Die genutzt Technologie wird in vielen Sprachen eingesetzt.
		\end{itemize}
	% end
	
	\paragraph{Nachteile}
		\begin{itemize}
			\item Manchmal ist es nicht möglich, Fehler so anzuzeigen (beispielsweise wenn alle Werte gültig sind).
			\item Der Aufrufer muss extra prüfen und daran denken, ob und welche Codes zurück kommen könnten.
		\end{itemize}
	% end
% end

\subsection{Exceptions}
	\introduces{von Exceptions}{exceptions}

	Schauen wir uns nun \textit{Exception} an, ein sehr viel mächtigeres System als Result Codes.
	
	Die Grundidee einer Exception ist, dass die Ausführung des Codes an einer beliebigen Stelle abgebrochen wird und dem Aufrufer über einen weiteren Mechanismus (den Exceptions) aufgezeigt wird, dass es Fehler vorlag. Der Aufrufer kann den Fehler anschließend behandeln.
	
	Das System besteht aus den folgenden Teilen, welche wir in den nächsten Abschnitten näher betrachten:
	\begin{itemize}
		\item Werfen von Exceptions und
		\item Fangen von Exceptions.
	\end{itemize}
	
	\subsubsection{Werfen von Exceptions}
		Eine Methode, welche beispielsweise die Berechnung \(\frac{a}{b}\) durchführt, muss einen Fehler auslösen, wenn \(b = 0\) gilt.
		
		Im Kontext von Exceptions wird dieses Auslösen eines Fehler \textit{werfen} einer Exception genannt, das heißt der Code bricht ab, es wird kein Wert zurück gegeben und der Aufrufer \enquote{erhält} den Fehler, welcher eine genauere Beschreibung enthalten kann (beispielsweise die Nachricht \enquote{MathException: Cannot divide by 0.}).
		
		\paragraph{Beispiel}
			\begin{figure}[H]
				\centering
				\begin{lstlisting}[language = Java, style = base]
divide(int a, int b) {
	if (b == 0) {
		// Nach der folgenden Zeile wird zum Aufrufer zurueck gekehrt, dieser
		// erhaelt die Nachricht und die Ausfuehrung der Methode wird
		// abgebrochen.
		throw "MathException: Cannot divide by 0." // Werfen der Exception.
	}

	// Somit koennen wir uns nun sicher sein, dass 'b != 0' gilt und einfach mit
	// der Division forfahren.
	return /* Divisions-Algorithmus */
}
				\end{lstlisting}
				\caption{Exceptions Werfen: Beispiel}
				\label{fig:throw_exceptions}
			\end{figure}
		% end
	% end
	
	\subsubsection{Fangen von Exceptions}
		Rufen wir eine Methode auf, welche eine Exception werfen kann (dies wird je nach Sprache in der Signatur der Methode dokumentiert), so müssen wir diese \textit{fangen}. Das bedeutet, wir müssen die Exception empfangen und den Fehler behandeln (wie auch immer).
		
		Dies geschieht zumeist mit einem Try-Catch-Konstrukt, welcher in zwei Blöcke aufgeteilt ist:
		\begin{itemize}
			\item Der \textit{Try-Block} enthält den Code, der die Exception auslösen kann. Tritt irgendwo eine Exception auf, so bricht die Ausführung dieses Blocks ab.
			\item Der \textit{Catch-Block} fängt eine mögliche Exception und wird nur ausgeführt, wenn im Try-Block ein Fehler aufgetreten ist. Sofern der Catch-Block nicht für einen Abbruch der Ausführung sorgt, wird nach seiner Ausführung einfach mit der ersten Zeile nach dem Try-Catch fortgefahren.
			\item In den meisten Sprachen gibt es noch einen Finally-Block, diesen werden wir aber erst im Java-Abschnitt zu Exceptions behandeln.
		\end{itemize}
		
		\paragraph{Beispiel}
			Sei wieder die Methode aus Abbildung \ref{fig:throw_exceptions} gegeben, nur rufen wir diese diesmal auf.
			\begin{figure}[H]
				\centering
				\begin{lstlisting}[language = Java, style = base]
// Try-Catch-Konstrukt
try { // Try-Block

	// Dieser Aufruf geht noch gut, denn '2 != 0'.
	divide(4, -2)
	// Dieser Aufruf wird fehlschlagen und der Catch-Block wird ausgefuehrt.
	divide(5, 0)
	// Dieser Aufruf wird nicht mehr ausgefuehrt, da der vorherige Aufruf
	// fehlgeschlagen ist.
	divide(6, 2)
} catch (String exception) { // Catch-Block

	// Der String 'exception' enthaelt nun den Wert "MathException: Cannot divide
	// by 0.", welcher von der Methode divide(int, int) als Fehlermeldung
	// uebergeben wurde.
	// Wir geben den Fehler hier einfach aus und behandeln ihn nicht weiter.
	print(exception)
}
				\end{lstlisting}
				\caption{Exceptions Fangen: Beispiel}
			\end{figure}
		% end
	% end
	
	\subsubsection{Exception-Typen}
		Es wird im allgemeinen zwischen den folgenden Exception-Typen unterschieden:
		\begin{description}
			\item[Geprüft] Diese Exceptions müssen von dem Aufrufer gefangen und behandelt oder weitergeleitet werden. Das Ignorieren der Exception führt zu einem Compiler Fehler.
			\item[Nicht Geprüft] Diese Exceptions müssen nicht von dem Aufrufer gefangen werden und können ignoriert werden. Allerdings stürzt das Programm ab, sollte doch ein solcher Fehler auftreten.
		\end{description}
		
		\info{Die Diskussion, ob man nur geprüfte Exceptions, nur ungeprüfte Exceptions oder beides verwenden sollte, ist sehr langwierig und es gibt für beide Seiten gute Argumente. Meiner Meinung nach ist die Mischform der beste Weg, da dieser am meisten Flexibilität bietet.}
	% end
% end

% end

% TODO: Schreiben
%\section{Generische Programmierung} \functionalMark \imperativeMark \oopMark
%	\introduces{von generischer Programmierung}{generischeProgrammierung}

\todo{Schreiben}
%% end

\section{Datenstrukturen} \functionalMark \imperativeMark \oopMark
	\introduces{von Datenstrukturen}{datastruct}

Eine Datenstruktur ist ein Objekt zur Speicherung und Organisation von Daten, indem diese in einer bestimmten Art und Weise angeordnet sind und es klare Namen gibt, sodass unterschiedliche Entwickler sich über  Datenstrukturen unterhalten können, ohne an eine bestimmte Programmiersprache gebunden zu sein.

\info{In dieser Veranstaltung werden wir Datenstrukturen nur grob behandeln, genauer wird dies in der Veranstaltung \enquote{Algorithmen und Datenstrukturen} behandelt.}

Im allgemeinen unterscheidet man zwischen folgenden Typen von Datenstrukturen:
\begin{description}
	\item[Indexbasiert] Jedes Element innerhalb einer Datenstruktur 
	\item[Nicht Indexbasiert]
\end{description}

\subsection{Arrays, Listen, Mengen} \functionalMark \imperativeMark \oopMark
	Bevor wir uns einige Implementierungen von Arrays, Listen und Mengen anschauen, stellt sich zuerst die Frage, was das eigentlich alles ist?
	
	Alle drei stellen eine Auflistung von Elementen eines beliebigen Typs dar und dienen dazu, beliebig viele und noch nicht zur Compile-Zeit bekannte Elemente in einer Variablen zusammenzufassen. Dabei wird unterschieden zwischen \textit{indexbasierten} und \textit{nicht indexbasierten} Strukturen, wobei bei ersteren jedes Element mit einem Index (einer Zahl) identifiziert werden kann, bei letzteren nicht.
	
	\subsubsection{Array}
		\introduces{von Arrays}{datastructArray}
	
		Ein Array ist eine solche Auflistung, welche in den meisten Sprachen nicht zur Laufzeit vergrößert werden kann und in einigen Sprachen (beispielsweise C) sogar schon zur Laufzeit feststehen muss. Grob gesagt kann man sich ein Array als beschriftetes Kistensystem vorstellen, bei dem jede Kiste eine Zahl zugewiesen bekommt:
		\begin{figure}[H]
			\centering
			\begin{tikzpicture}[->, shorten >= 2pt, every node/.style = { minimum width = 1cm }, elem/.style = { draw, rectangle, minimum width = 2cm }]
				\node (i0) {\texttt{0}};
				\node [elem, right = of i0] (e0) {\enquote{One}};
				
				\node [right = 2 of e0] (i1) {\texttt{1}};
				\node [elem, right = of i1] (e1) {\enquote{Two}};
				
				\node [below = of i0] (i2) {\texttt{2}};
				\node [elem, right = of i2] (e2) {\enquote{Three}};
				
				\node [right = 2 of e2] (i3) {\texttt{3}};
				\node [elem, right = of i3] (e3) {\enquote{Four}};
				
				\draw (i0) -- (e0);
				\draw (i1) -- (e1);
				\draw (i2) -- (e2);
				\draw (i3) -- (e3);
			\end{tikzpicture}
			\caption{Datenstruktur: Array}
		\end{figure}
		
		\warning{In annähernd allen Programmiersprachen werden Arrays ab dem Index \texttt{0} indiziert! Methoden zum Anzeigen der Länge zeigen jedoch die Anzahl der Elemente an und nicht den letzten Index!}
	% end
	
	\subsubsection{Liste}
		\introduces{von Listen}{datastructList}
		
		Eine Liste ist einem Array sehr ähnlich, die Größe kann im Allgemeinen aber zur Laufzeit angepasst werden und es existieren viele verschiedene Implementierungen (unter anderem Implementierungen zur Abbildung auf Arrays, sogenannte Array-Listen). Oftmals ist auch eine Liste indexbasiert, dies muss aber nicht immer der Fall sein (beispielsweise bei gelinkten Listen, die wir später behandeln werden).
	% end
	
	\subsubsection{Menge}
		\introduces{von Sets/Mengen}{datastructSet}
	
		Eine Menge ist, lapidar gesagt, eine Liste ohne Duplikate. Ansonsten gelten genau die gleichen Fakten wie bei einer Liste: Es kann indexbasiert sein, muss es aber nicht, \dots.
	% end
	
	\subsubsection{Linked List (Gelinkte Liste)}
		Eine \textit{gelinkte Liste} ist eine indexlose Liste, in der die Datenspeicherung wie folgt abgebildet wird:
		\begin{itemize}
			\item Jedes Element der Liste enthält die Referenz auf:
				\begin{itemize}
					\item die eigentlichen Nutzdaten (data),
					\item den Nachfolger des Elementes (next) und
					\item im Fall von einer doppelt gelinkten Liste, eine Referenz auf das vorherige Element (previous).
				\end{itemize}
			\item Existiert kein nachfolgendes/vorheriges Element, so wir nichts in das Feld eingetragen.
			\item Manchmal gibt es noch eine Schnittstelle, die eine Referenz auf das erste Element enthält und einige Methoden zur Verfügung stellt (\texttt{first}, \texttt{second}, \texttt{third}, \dots). Diese ist aber nicht vorgeschrieben.
		\end{itemize}
		
		Visualisiert sieht eine einfach gelinkte Liste wie folgt aus:
		\begin{figure}[H]
			\centering
			\begin{tikzpicture}[every node/.style = { align = center }, elem/.style = { draw, rectangle, minimum width = 3cm }]
				\node [elem] (el0) {\texttt{data: } \enquote{One} \\ \texttt{next}};
				\node [elem, right = 1.5 of el0] (el1) {\texttt{data: } \enquote{Two} \\ \texttt{next}};
				\node [elem, right = 1.5 of el1] (el2) {\texttt{data: } \enquote{Three} \\ \texttt{next}};
				\node [elem, right = 1.5 of el2] (el3) {\texttt{data: } \enquote{Four} \\ \texttt{next}};
				
				\draw (el0.west) -- (el0.east);
				\draw (el1.west) -- (el1.east);
				\draw (el2.west) -- (el2.east);
				\draw (el3.west) -- (el3.east);
				
				\coordinate (el0n) at ($(el0.east)!0.5!(el0.south east)$);
				\coordinate (el0a) at ($(el0.north)!0.5!(el1.north)+(0,0.5)$);
				\draw (el0n) -| (el0a);
				\draw [->] (el0a) -| (el1.north);
				
				\coordinate (el1n) at ($(el1.east)!0.5!(el1.south east)$);
				\coordinate (el1a) at ($(el1.north)!0.5!(el2.north)+(0,0.5)$);
				\draw (el1n) -| (el1a);
				\draw [->] (el1a) -| (el2.north);
				
				\coordinate (el2n) at ($(el2.east)!0.5!(el2.south east)$);
				\coordinate (el2a) at ($(el2.north)!0.5!(el3.north)+(0,0.5)$);
				\draw (el2n) -| (el2a);
				\draw [->] (el2a) -| (el3.north);
			\end{tikzpicture}
			\caption{Datenstruktur: Einfach gelinkte Liste}
		\end{figure}
		
		Und das gleiche als doppelt gelinkte Liste:
		\begin{figure}[H]
			\centering
			\begin{tikzpicture}[every node/.style = { align = center }, elem/.style = { draw, rectangle, minimum width = 3cm }]
				\node [elem] (el0) {\texttt{data: } \enquote{One} \\ \texttt{next} \\ \texttt{previous}};
				\node [elem, right = 1.5 of el0] (el1) {\texttt{data: } \enquote{Two} \\ \texttt{next} \\ \texttt{previous}};
				\node [elem, right = 1.5 of el1] (el2) {\texttt{data: } \enquote{Three} \\ \texttt{next} \\ \texttt{previous}};
				\node [elem, right = 1.5 of el2] (el3) {\texttt{data: } \enquote{Four} \\ \texttt{next} \\ \texttt{previous}};
				
				\coordinate (el0tl) at ($(el0.north west)!1/3!(el0.south west)$);
				\coordinate (el0bl) at ($(el0.north west)!2/3!(el0.south west)$);
				\coordinate (el0tr) at ($(el0.north east)!1/3!(el0.south east)$);
				\coordinate (el0br) at ($(el0.north east)!2/3!(el0.south east)$);
				\draw (el0tl) -- (el0tr);
				\draw (el0bl) -- (el0br);
				
                \coordinate (el1tl) at ($(el1.north west)!1/3!(el1.south west)$);
                \coordinate (el1bl) at ($(el1.north west)!2/3!(el1.south west)$);
                \coordinate (el1tr) at ($(el1.north east)!1/3!(el1.south east)$);
                \coordinate (el1br) at ($(el1.north east)!2/3!(el1.south east)$);
				\draw (el1tl) -- (el1tr);
				\draw (el1bl) -- (el1br);
				
                \coordinate (el2tl) at ($(el2.north west)!1/3!(el2.south west)$);
                \coordinate (el2bl) at ($(el2.north west)!2/3!(el2.south west)$);
                \coordinate (el2tr) at ($(el2.north east)!1/3!(el2.south east)$);
                \coordinate (el2br) at ($(el2.north east)!2/3!(el2.south east)$);
				\draw (el2tl) -- (el2tr);
				\draw (el2bl) -- (el2br);
				
                \coordinate (el3tl) at ($(el3.north west)!1/3!(el3.south west)$);
                \coordinate (el3bl) at ($(el3.north west)!2/3!(el3.south west)$);
                \coordinate (el3tr) at ($(el3.north east)!1/3!(el3.south east)$);
                \coordinate (el3br) at ($(el3.north east)!2/3!(el3.south east)$);
				\draw (el3tl) -- (el3tr);
				\draw (el3bl) -- (el3br);
				
				
				\coordinate (el0n) at ($(el0tr)!0.5!(el0br)$);
				\coordinate (el0a) at ($(el0.north)!0.5!(el1.north)+(0,0.5)$);
				\draw (el0n) -| (el0a);
				\draw [->] (el0a) -| (el1.north);
				
				\coordinate (el1n) at ($(el1tr)!0.5!(el1br)$);
				\coordinate (el1a) at ($(el1.north)!0.5!(el2.north)+(0,0.5)$);
				\draw (el1n) -| (el1a);
				\draw [->] (el1a) -| (el2.north);
				
				\coordinate (el2n) at ($(el2tr)!0.5!(el2br)$);
				\coordinate (el2a) at ($(el2.north)!0.5!(el3.north)+(0,0.5)$);
				\draw (el2n) -| (el2a);
				\draw [->] (el2a) -| (el3.north);
				
				
				\coordinate (el1o) at ($(el1bl)!0.5!(el1.south west)$);
				\coordinate (el1b) at ($(el0.south)!0.5!(el1.south)-(0,0.5)$);
				\draw (el1o) -| (el1b);
				\draw [->] (el1b) -| (el0.south);
				
				\coordinate (el2o) at ($(el2bl)!0.5!(el2.south west)$);
				\coordinate (el2b) at ($(el1.south)!0.5!(el2.south)-(0,0.5)$);
				\draw (el2o) -| (el2b);
				\draw [->] (el2b) -| (el1.south);
				
				\coordinate (el3o) at ($(el3bl)!0.5!(el3.south west)$);
				\coordinate (el3b) at ($(el2.south)!0.5!(el3.south)-(0,0.5)$);
				\draw (el3o) -| (el3b);
				\draw [->] (el3b) -| (el2.south);
			\end{tikzpicture}
			\caption{Datenstruktur: Doppelt gelinkte Liste}
		\end{figure}
	% end
% end

\subsection{Map}
	\introduces{von Maps/Dictionaries}{datastructMap}

	Eine \textit{Map}, oder auf \textit{Dictionary} genannt, ist eine Art Liste, welche als Indizes aber jeden beliebigen Typ haben kann (beispielsweise Strings). Damit ist beispielsweise eine Zuordnung von Namen zu Objekten möglich.
	
	\begin{figure}[H]
		\centering
			\begin{tikzpicture}[->, shorten >= 2pt, every node/.style = { minimum width = 2cm }, elem/.style = { draw, rectangle, minimum width = 1cm }]
				\node (i0) {\texttt{"{}One{}"}};
				\node [elem, right = of i0] (e0) {\enquote{0}};
				
				\node [right = 2 of e0] (i1) {\texttt{"{}Two{}"}};
				\node [elem, right = of i1] (e1) {\enquote{1}};
				
				\node [below = of i0] (i2) {\texttt{"{}Three{}"}};
				\node [elem, right = of i2] (e2) {\enquote{2}};
				
				\node [right = 2 of e2] (i3) {\texttt{"{}Four{}"}};
				\node [elem, right = of i3] (e3) {\enquote{3}};
				
				\draw (i0) -- (e0);
				\draw (i1) -- (e1);
				\draw (i2) -- (e2);
				\draw (i3) -- (e3);
			\end{tikzpicture}
		\caption{Datenstruktur: Map/Dictionary}
	\end{figure}
% end

% end

\section{I/O (Input/Output)} \functionalMark \imperativeMark \oopMark
	\introduces{von I/O}{io}

Mit I/O (Input/Output, Eingabe/Ausgabe) wird im Allgemeinen das Lesen von Daten (Input, Eingabe) und das Schreiben von Daten (Output, Ausgabe) bezeichnet.

Dies kann das Erstellen von Ordnern im Dateisystem sein, das Lesen von Dateien oder das Schreiben selbiger.

Jede Sprache kann unterschiedlich viele unterschiedliche I/O-Operationen durchführen, die meisten unterstützen aber mindestens das Lesen und Schreiben von Dateien.

\subsection{Allgemeiner Aufbau}
	\warning{In diesem Abschnitt schauen wir uns den allgemeinen Aufbau von Lese-/Schreiboperationen an, wie er in den meisten Sprachen zu finden ist. Es gibt aber auch Sprachen, bei denen dies grundlegend anders funktioniert.}
	
	\paragraph{Lesen}
		\begin{enumerate}
			\item Die Datei wird \textit{geöffnet} (\textit{open}). Hierbei wird überprüft, ob die Datei überhaupt existiert, ob das Programm die Datei Lesen darf, u.v.m..
			\item Die Daten werden (auf irgendeine Weise) gelesen\dots
			\item Die Datei wird \textit{geschlossen} (\textit{close}). Dabei werden die Ressourcen wieder freigegeben, sodass ein anderes Programm die Datei lesen kann, die Datei gelöscht werden kann, etc..
		\end{enumerate}
		
		\warning{Der Letzte Schritt (close) ist mit Abstand am wichtigsten, da hiermit sichergestellt wird, dass das umliegende System intakt bleibt. Eine Datei sollte \textit{immer} geschlossen werden, unabhängig ob bei dem Lesen ein Fehler aufgetreten ist.}
	% end
	
	\paragraph{Schreiben}
		Der Prozess, um eine Datei zu Schreiben ist dem Prozess zum Lesen sehr ähnlich, wie wir im folgenden sehen werden.
	
		\begin{enumerate}
			\item Die Datei wird \textit{geöffnet} (\textit{open}). Hierbei wird überprüft, ob die Datei überhaupt existiert, ob das Programm die Datei Schreiben darf, u.v.m..
			\item Die Daten werden (auf irgendeine Weise) schreiben\dots
			\item Die Datei wird \textit{geschlossen} (\textit{close}). Dabei werden die Ressourcen wieder freigegeben, sodass ein anderes Programm die Datei lesen (oder schreiben) kann, die Datei gelöscht werden kann, etc..
		\end{enumerate}
		
		\warning{Der Letzte Schritt (close) ist mit Abstand am wichtigsten, da hiermit sichergestellt wird, dass das umliegende System intakt bleibt. Eine Datei sollte \textit{immer} geschlossen werden, unabhängig ob bei dem Schreiben ein Fehler aufgetreten ist.}
	% end
% end
% end

\section{Multithreading und parallele Verarbeitung} \functionalMark \imperativeMark \oopMark
	\introduces{von Multithreading und Parallelisierung}{multithreading}

Bisher sind uns nur Programme geläufig, welche sequentiell (das heißt nacheinander) ablaufen. In der Praxis ist dies in vielen Fällen nützlich, es gibt aber ebenso viele Fälle, in denen mehrere Dinge parallel ablaufen müssen (beispielsweise möchte man nicht immer die Musik pausieren müssen, nur um einen Absatz im Skript zu lesen).

Vor ähnliche Problematiken werden wir gestellt, wenn unser Programm irgendetwas im Hintergrund abarbeiten muss (beispielsweise eine große Rechenaufgabe wir Wurzelziehen). Hierbei unterstützt uns Multithreading, was von vielen Sprachen in sehr unterschiedlichen Wegen implementiert wird. Damit ist es möglich, Dinge auszulagern und im Hintergrund laufen zu lassen.

\subsection{Thread}
	Ein \textit{Thread} bezeichnet einen Ausführungsstrang unserer Anwendung, welcher \textit{parallel}, also zeitgleich \footnote{\enquote{zeitgleich} ist hier tatsächlich etwas hoch gegriffen, das Betriebssystem und die CPU simulieren dies nur sehr gut.}, zu anderen Ausführungssträngen ausgeführt wird.
	
	Gegenüber dem Betriebssystem tritt unsere Anwendung dennoch als ein Prozess (eine Applikation) auf, weshalb man bei Threads auch von \textit{leichtgewichtigen Prozessen} spricht.
	
	Ein Thread kann gestartet werden und läuft ab diesem Moment asynchron (zeitlich unabhängig) zu anderen Threads. Zum stoppen eines Threads kann dieser wieder mit anderen Threads synchronisiert werden (join) oder auch einfach gestoppt (terminiert) werden.
% end

\subsection{Parallelisierung}
	\subsubsection{Echte Parallelität}
		Multithreading kann uns helfen, eine komplexe Rechenaufgabe drastisch zu beschleunigen, in dem wir mehrere Operationen zeitgleich durchführen. Hierbei ist zu beachten, dass uns dies nur etwas nützt, wenn unsere Threads \textit{echtparallel} ablaufen. Das bedeutet, dass die Operationen sogar auf der Hardware (der CPU) zeitgleich ausgeführt werden und die Parallelität nicht nur von dem Betriebssystem/der CPU simuliert wird. Konkret heißt das, wir dürfen maximal \( \text{Kernanzahl} - 1 \) Threads starten, damit noch eine Beschleunigung eintritt.
	% end
	
	\subsubsection{Simulierte Parallelität (Scheduling)}
		Läuft unser Programm auf einer Maschine mit einem Kern und es ist somit keine echte Parallelität möglich, hilft uns Multithreading nicht, um Rechenoperationen zu beschleunigen.
		
		Allerdings ist uns geholfen, wenn beispielsweise ein Thread die GUI aufbaut, ein anderer Thread die Verbindung zu einer Datenbank und ein dritter Thread die Benutzereingaben entgegen nimmt. \\ Auch kann dies sinnvoll sein, wenn ein Thread mit dem Nutzer interagiert und ein anderer auf Änderungen einer Datei wartet, um den Nutzer darüber zu informieren. \\ Diese Kette an Beispielen lässt sich ewig fortsetzen und wir sehen, dass Multithreading sehr viele unterschiedliche Anwendungsgebiete hat.
		
		Diese Form der Parallelität, beziehungsweise der Nutzung selbiger, ist die Häufigste in Anwenderprogrammen, die eben keine komplexen Berechnungen durchführen.
	% end
% end

\subsection{Beispiel: Window Manager}
	Ein Ort, an dem wir schon mit Multithreading in Kontakt gekommen sind, ist der Window Manager des Betriebssystems. Dieser verwaltet alle offenen Programme (mit GUI), Benutzereingaben, Bildschirme (einen oder mehrere), \dots.
	
	Auch hierbei ist Multithreading sehr wichtig, da wir nicht wollen, dass ein Programm pausiert, sobald wir mit der Maus den Fokus zu einem anderen Programm wechseln. Damit wäre es zum Beispiel nicht möglich, zeitgleich Musik zu hören und dieses Skript zu lesen.
	
	Dabei kann es sich je nach Implementierung sogar um echte Parallelität handeln, wenn die CPU des Computers mehrere Kerne hat. Meistens laufen jedoch so viele Programme parallel, dass die Kerne nicht ausreichen und die Parallelität somit simuliert wird (Scheduling).
% end

% end

\section{GUI (Graphical User Interface)} \functionalMark \imperativeMark \oopMark
	\todo{Schreiben}
% end

\section{Dokumentation} \functionalMark \imperativeMark \oopMark
	\introduces{von Dokumentation in der Software}{doku}

\todo{Schreiben}

\subsection{Verträge} \functionalMark \imperativeMark \oopMark
	\introduces{von Veträgen}{vertraege}

	Gerade in nicht-typisierten Programmiersprachen wie \racket ist es sehr sinnvoll, einen \textit{Vertrag} zwischen der Methode und dem Aufrufer zu schließen. In diesem Wird genau festgelegt, welcher Parameter von welchem Typ erwartet wird, wie dieser genau auszusehen hat und welchen Typ der Rückgabewert hat und wie dieser genau aussieht.
	
	Die Beschreibung der Funktionalität der Methode gehört nicht mit zu dem Vertrag!
	
	\paragraph{Beispiel}
		Eine Funktion \(f(x)\) berechnet die reelle Quadratwurzel der übergebenen reellen Zahl \(x\). Somit ist eine Einschränkung von \(x\), dass selbiges positiv sein muss (also \(x \in \mathbb{R} _ +\)). Für die Rückgabe der Funktion können wir garantieren, dass ausschließlich positive reelle Zahlen zurück gegeben werden, also \(f(x) \in \mathbb{R} _ +\)).
		
		Ein Vertrag der Funktion kann nun wie oben in Textform formuliert werden oder mit einer bestimmten Syntax (zum Beispiel \enquote{\( f(x \in \mathbb{R} _ +) \in \mathbb{R} _ + \)} oder \enquote{\( f : \mathbb{R} _ + \rightarrow \mathbb{R} _ + \)}). Dies stellt aber nur ein Beispiel dar und variiert von Sprache zu Sprache.
	% end
% end
% end

\section{Testen} \functionalMark \imperativeMark \oopMark
	\todo{Schreiben}
% end

	% end

    \chapter{Racket}
	    \label{c:racket}
    
        Wir werden uns nun als erstes mit \racket\, auseinandersetzen. \racket\, steht für \enquote{How to Design Programs - Teaching Languages} und stellt eine Sprache bereit, die auf Racket und damit auf LISP aufbaut und für die einzelne Sprachfeatures \enquote{deaktiviert} werden können, um die Programmierung an Anfänger heranzuführen. Die Unterschiede zwischen den einzelnen sogenannten \enquote{Sprachlevels} werden wir uns hier nicht genauer angucken, sondern annehmen, dass wir stetig auf dem höchsten Sprachlevel arbeiten.

Im folgenden schauen wir uns \racket\, an und wie die in \ref{c:abstrakte_konzepte} Konzepte in der Sprache implementiert werden.

\section{Lexikalische Bestandteile}
	\subsection{Datentypen}
	\implements{Datentypen}{datentypen}{\racket}
	
	Im folgenden schauen wir uns an, was es in \racket\, für Datentypen gibt:
	\begin{itemize}
		\item Zahlen
			\begin{itemize}
				\item Ganzzahlen
				\item Fließkommazahlen
				\item Brüche
				\item Irrationale (ungenaue) Zahlen
				\item Komplexe Zahlen
			\end{itemize}
		\item Wahrheitswerte
		\item Symbole
		\item Strings
		\item Structs
		\item Listen
	\end{itemize}

	Dabei ist \racket\, aber nicht statisch typisiert, das heißt die Datentypen nicht mit angegeben werden, sondern es ist ausreichend, wenn zur Laufzeit der korrekte Datentyp in einer Variable gespeichert ist (es ist zum Beispiel nicht möglich, Strings zu addieren). Ist nicht der korrekte Datentyp gespeichert, so tritt ein Fehler auf.
	
	\paragraph{Symbole}
		Symbole sind einfache Zeichenketten, die ausschließlich verglichen werden können und weniger Funktionalität als Strings bieten.
		
		Allerdings ist die Verwendung von Symbolen sehr effizient und zu empfehlen, wenn wir mit der produzierten Zeichenkette nichts weiter tun wollen als sie zu vergleichen (dies tritt erstaunlich oft auf, öfter als man im Allgemeinen denkt).
	% end
	
	\paragraph{Listen}
		Listen ist einer der wichtigsten Datentypen in \racket. Wir werden uns diesen wichtigen Datentyp im Abschnitt \ref{sec:racket_lists} genauer anschauen.
	% end
	
	\paragraph{Sondertyp \textit{Struct}}
		Ein \textit{Struct} (eine Struktur) ist von dem Entwickler definierbar und ermöglicht es, komplexe Datentypen zu speichern. Wir werden uns diesen besonderen Datentyp im Abschnitt \ref{sec:structs} anschauen.
	% end
% end

\subsection{Literale}
	\implements{Literalen}{literale}{\racket}
	
	Wie wir Literale im Code ablegen, hängt von dem Datentyp ab, den wir produzieren wollen:
	
	\begin{table}[H]
		\centering
		\begin{tabular}{l | l}
			\textbf{Datentyp} & \textbf{Schreibweise} \\ \hline
			Ganzzahl & \lstinline[language = Racket]|42| \\
			Fließkommazahl & \lstinline[language = Racket]|21.5| \\
			Bruch & \lstinline[language = Racket]|2/3| \\
			Irrationale (ungenaue) Zahl & \lstinline[language = Racket]|#i2.1415| \\
			Komplexe Zahl & \lstinline[language = Racket]|2+5i| \\
			Wahrheitswert & \lstinline[language = Racket]|true|, \lstinline[language = Racket]|false|, \lstinline[language = Racket]|#t|, \lstinline[language = Racket]|#f|, \lstinline[language = Racket]|#true|, \lstinline[language = Racket]|#false| \\
			Symbol & \lstinline[language = Racket]|'symbol|, \lstinline[language = Racket]|'"string as symbol"| \\
		\end{tabular}
		\caption{\racket: Literale verschiedener Datentypen}
	\end{table}

	\paragraph{Symbol-Literale}
		Wenn wir Symbole verwenden, der Text hinter den Symbolen allerdings ein valides Literal eines anderen Datentyps darstellt, so wird das Symbol in den jeweiligen Datentyp umgeformt. Außerdem können wir auch Leerzeichen und Klammern innerhalb eines Symbols verwenden, wenn wir diesen einen Backslash (\(\backslash\)) voranstellen. Wenn wir viele Leerzeichen innerhalb eines Symbols verwenden wollen, können wir um den Inhalt des Symbols Senkrechtstriche setzen.
		
		Somit ist alles folgende äquivalent:
		\begin{itemize}
			\item \lstinline[language = Racket]|'"string as symbol"| \(\iff\) \lstinline[language = Racket]|"string as symbol"|
			\item \lstinline[language = Racket]|'12.34| \(\iff\) \lstinline[language = Racket]|12.34|
			\item \lstinline[language = Racket]|'\ \(| \(\iff\) \texttt{'| (|}
		\end{itemize}
	% end
% end

\subsection{Bezeichner und Konventionen}
	\implements{Bezeichnern und Konventionen}{identifier}{\racket}

	In \racket\, können annähernd alle Zeichen in Bezeichnern genutzt werden, u.a. \texttt{-}, \texttt{?}, usw.. Nicht möglich ist es, eine Zahl als das erste Zeichen eines Bezeichners zu wählen.
	
	Damit sind beispielsweise folgende Bezeichner gültig:
	\begin{itemize}
		\item \texttt{odd?}
		\item \texttt{-}
		\item \texttt{+-123?!}
	\end{itemize}

	\paragraph{Konventionen}
		Bei der Benennung von Variablen und Funktionen sind folgende Konventionen üblich:
		\begin{itemize}
			\item Es werden nur Kleinbuchstaben verwendet.
			\item Einzelne Wortabschnitte werden mit Bindestrichen getrennt (Beispiel: \texttt{is-this-real}).
			\item Zur Benennung von Funktionen gibt es noch weitere Konventionen:
				\begin{itemize}
					\item Funktionen zur Umwandlung von Datentyp A in Datentyp B werden \texttt{A->B} genannt.
					\item Funktionen, deren Rückgabe ein Wahrheitswert ist, wird in Fragezeichen nachgestellt. Beispiel: \texttt{odd?}
					\item \todo{Weiterführen}
				\end{itemize}
		\end{itemize}
	% end
% end

\subsection{Strukturierung des Codes}
	\todo{Schreiben}
% end

% end

\section{Anweisungen}
	\todo{Schreiben}

\subsection{Methodenaufrufe}
	\todo{Schreiben}
% end

\subsection{Konstanten}
	\todo{Schreiben}
% end

\subsection{Operatoren}
	\todo{Schreiben}

	\subsubsection{Arithmetik}
		\todo{Schreiben}
	% end
% end

\subsection{Abfragen/Vergleiche}
	\todo{Schreiben}

	\subsubsection{Gleichheit, Größer-/Kleiner-Gleich}
		\todo{Schreiben}
	% end
	
	\subsubsection{Prädikate}
		\todo{Schreiben}
		
		\begin{itemize}
			\item \lstinline[language = Racket|number?|
			\item \lstinline[language = Racket|real?|
			\item \lstinline[language = Racket|rational?|
			\item \lstinline[language = Racket|integer?|
			\item \lstinline[language = Racket|natural?|
			\item \lstinline[language = Racket|string?|
			\item \lstinline[language = Racket|cons?|, \lstinline[language = Racket|empty?|
		\end{itemize}
	% end
% end

% end

\section{Kontrollstrukturen}
	\implements{Kontrollstrukturen}{kontrollstrukturen}{\racket}

In diesem Abschnitt schauen wir uns an, wie Kontrollstrukturen in \racketText umgesetzt werden. \racketText kennt dabei die Kontrollstrukturen \textit{If} und \textit{Cond} (von \enquote{Conditional}), wobei \textit{Cond} nur eine Vereinfachung von vielen geschalteten Ifs darstellt.

In \racketText gibt es keine Schleifen, da \racketText eine funktionale Programmiersprache ist! Alle Wiederholungen werden über Rekursion \footnote{Siehe \refImpl{recursion}{\racket}} gelöst.

\subsection{If}
	Das \textit{If} ist die einfachste Form der Verzweigung und hat folgende Form:
	\begin{figure}[H]
		\centering
		\lstinline[language = Racket]|(if <Abfrage> <Wahr-Fall> <Falsch-Fall>)|
	\end{figure}
	Wird der Ausdruck \texttt{<Abfrage>} zu Wahr ausgewertet, so wird das Ergebnis von \texttt{<Wahr-Fall>} zurück gegeben. Ansonsten wir das Ergebnis von \texttt{<Falsch-Fall>} zurück gegeben.
	
	\warning{Bei einem If in \racketText müssen \textit{immer} sowohl Wahr- als auch Falsch-Fall angegeben werden!}
	
	\paragraph{Beispiele}
		\begin{itemize}
			\item \lstinline[language = Racket]|(if (= (modulo x 2) 0) 'even 'odd)| \\
				  Wertet zu \texttt{'even} aus, wenn \texttt{x} gerade ist und sonst zu \texttt{'odd}.
			\item \lstinline[language = Racket]|(if (> x y) x y)| \\
				  Wertet zu dem Maximum von \texttt{x} und \texttt{y} aus (also \( \max \{ x, y \} \)).
		\end{itemize}
	% end
% end

\subsection{Cond}
	Ein \textit{Cond} vereinfacht verschachtelte If-Abfragen immens, wie wir gleich sehen werden. Schauen wir uns dazu folgendes verschachteltes If an:
	\begin{figure}[H]
		\centering
		\begin{lstlisting}[language = Racket]
(if (< x y)
	-1
	(if (> x y)
		1
		0
	)
)
\end{lstlisting}
	\end{figure}

	Und nun noch die allgemeine Syntax von Cond:
	\begin{figure}[H]
		\centering
		\lstinline[language = Racket]|(cond (<Test1> <Ausdruck1>) *@\dots@* (<TestN> <AusdruckN>)) [(else <Ansonsten>)]|
	\end{figure}
	Wobei der gesamte Ausdruck zu \texttt{<AusdruckK>} auswertet genau dann wenn \texttt{<TestK>} Wahr wird und zu \texttt{<Ansonsten>} auswertet, wenn alle Tests negativ ausfallen.
	
	Dann können wir das obige If zu folgendem Code vereinfachen:
	\begin{figure}[H]
		\centering
		\begin{lstlisting}[language = Racket]
(cond
	((< x y) -1)
	((> x y)  1)
	((= x y)  0)
)
\end{lstlisting}
	\end{figure}

	Damit haben wir nun das nötige Handwerkszeug, um komplexe Programme zu schreiben.
% end
% end

\section{Funktionen}
	\todo{Schreiben}

\subsection{Bestandteile}
	\todo{Schreiben}
% end

\subsection{Verträge}
	\todo{Schreiben}
% end

\subsection{Rekursion}
	\todo{Schreiben}
% end

% end

\section{Fehlerbehandlung}
	\implements{Fehlerbehandlung}{fehlerbehandlung}{Racket}

In Racket gibt es die zwei typischen grundlegenden Arten von Fehlerbehandlung:
\begin{itemize}
	\item Exceptions in Form von Errors und
	\item Result Codes.
\end{itemize}

\subsection{Result Codes}
	\implements{Result Codes}{resultcodes}{Racket}
	
	In Racket werden Result Codes nicht besonders implementiert, wir können sie nur durch Fallunterscheidungen nutzen.
	
	\paragraph{Beispiel}
		Als Beispiel implementieren wir eine Funktion, welche die reelle Quadratwurzel einer Zahl berechnet. Ist die gegebene Zahl negativ, so gibt die Funktion \(-1\) zurück (Result Code).
		
		\begin{figure}[H]
			\centering
			\begin{lstlisting}[language = Racket]
(define (square-root-positive x)
	(if (< x 0)
		-1
		(sqrt x) ; Die Funktion sqrt gibt fuer negative Werte ein komplexes Ergebnis.
	)
)
\end{lstlisting}
		\end{figure}
	% end
% end

\subsection{Errors}
	\implements{Errors}{exceptions}{Racket}
	
	Außerdem können wir Errors einsetzen, um Fehler anzuzeigen. Diese sind meistens besser geeignet, da die Ausführung direkt abbricht und der Aufrufer nicht prüfen muss, ob ein Fehler aufgetreten ist.
	
	Ein Error lösen wir wie folgt aus:
	\begin{figure}[H]
		\centering
		\lstinline[language = Racket]|(error <Funktionsname> <Fehlermeldung>)|
	\end{figure}
	Der Funktionsname in dem der Fehler aufgetreten ist wird als Symbol übergeben, die Fehlermeldung als String.
	
	\paragraph{Beispiel}
		Wir implementieren folgende Funktion:
		\begin{figure}[H]
			\centering
			\begin{lstlisting}[language = Racket]
(define (square-root-positive x)
	(if (< x 0)
		(error 'square-root "Illegal value for real square root!")
		(sqrt x) ; Die Funktion sqrt gibt fuer negative Werte ein komplexes Ergebnis.
	)
)
			\end{lstlisting}
		\end{figure}
	
		Rufen wir die Funktion nun mit \lstinline[language = Racket]|(square-root-positive -4)| auf, so bekommen wir folgende Fehlermeldung: \texttt{square-root-real: Illegal value for real square root!}
	% end
% end

% end

\section{Datenstrukturen}
	\implements{Datenstrukturen}{datastruct}{Racket}

In diesem Abschnitt schauen wir uns an, was für Datenstrukturen in Racket implementiert werden und wie wir eigene hinzufügen können.

\subsection{Listen}
	\label{sec:racket_lists}

	Listen sind der zentrale Bestandteil von Racket, wie der Name der Ursprungssprache (LISP / List Processing) schon vermuten lässt.
	
	Listen sind in Racket die einzige Möglichkeit, \enquote{beliebig viele} Daten in einem Feld zu speichern und mit Hilfe von Rekursion über diese zu iterieren. Listen werden dabei als einfach gelinkte Listen abgelegt, das heißt eine Liste besteht aus den Kopf (\texttt{first}) und dem Rest der Liste (\texttt{rest}).
	
	Zum Umgang mit diesen Listen sind folgende Funktionen/Konstanten verfügbar (alle Daten können auch ad-hoc von einem Ausdruck berechnet werden):
	\begin{itemize}
		\item \lstinline[language = Racket]|(cons <Elemente> <Liste>)| \\ Funktion. Hängt das Element vorne an die Liste.
		\item \lstinline[language = Racket]|empty| \\ Konstante. entspricht einer leeren Liste und wird zum anlegen einer neuen Liste benötigt (als zweiter Parameter zur \texttt{cons}),
		\item \lstinline[language = Racket]|(list [Elemente])| \\ Funktion. Erstellt eine eine neue Liste, die alle gegebenen Elemente enthält. Die Elemente werden durch Leerzeichen getrennt.
		\item \lstinline[language = Racket]|(first <Liste>)| \\ Funktion. Gibt das erste Element der Liste zurück, also den Kopf.
		\item \lstinline[language = Racket]|(rest <Liste>)| \\ Funktion. Gibt den Rest der Liste (also die gesamte Liste ohne den Kopf) zurück. Ist die Liste leer, gibt es einen Fehler.
		\item \lstinline[language = Racket]|(second <Liste>)|, \lstinline[language = Racket]|(third <Liste>)|, \lstinline[language = Racket]|(fourth <Liste>)|, \lstinline[language = Racket]|(fifth <Liste>)|, \lstinline[language = Racket]|(sixth <Liste>)|, \lstinline[language = Racket]|(seventh <Liste>)|, \lstinline[language = Racket]|(eighth <Liste>)| \\ Funktionen. Geben das zweite/\dots/achte Element der Liste zurück.
		\item \lstinline[language = Racket]|(cons? <Arg>)| \\ Funktion. Gibt an, ob das gegebene Argument eine Liste ist.
		\item \lstinline[language = Racket]|(empty? <Arg)| \\ Funktion. Gibt an, ob das gegebene Argument eine leere Liste ist.
	\end{itemize}
% end

\subsection{Structs}
	\label{sec:racket_structs}

	Structs ermöglichen uns, viele Daten in einer Konstanten (oder einem Parameter) abzulegen und damit komplexe Datenstrukturen zu erstellen.
	
	\subsubsection{Definition}
		Zur Definition eines Struct-Typs wird folgender Code genutzt:
		\begin{figure}[H]
			\centering
			\lstinline[language = Racket]|(define-struct <Name> ([Attribute]))|
		\end{figure}
		Der Name gibt an, unter welchen Namen wir das Struct referenzieren können. Die Attribute definieren, unter welchem Namen wir Daten in dem Struct speichern können. Auf diese können wir anschließend zugreifen. Unterschiedliche Attribute können wir durch Leerzeichen separieren.
		
		\paragraph{Beispiel}
			Legen wir als Beispiel ein Struct zur Speicherung von Daten über einen Studierenden an:
			\begin{figure}[H]
				\centering
				\begin{lstlisting}[language = Racket]
(define-struct student (name matr-number))
\end{lstlisting}
			\end{figure}
		% end
	% end
	
	\subsubsection{Prädikate}
		Um zu Prüfen, ob eine Konstante \texttt{x} vom Typ des Structs \texttt{<Name>} ist, können wir die automatisch generierte Funktion \texttt{<Name>?} nutzen.
		
		\paragraph{Beispiel}
			Um zu prüfen, ob eine Variable \texttt{x} vom Typ \texttt{student} ist, nutzen wir folgende Code:
			\begin{figure}[H]
				\centering
				\begin{lstlisting}[language = Racket]
(student? x)
\end{lstlisting}
			\end{figure}
		% end
	% end
	
	\subsubsection{Nutzung, Attribute und Zugriff}
		Die Erstellung einer \enquote{Instanz} eines Structs \texttt{<Name>} geschieht wie folgt:
		\begin{figure}[H]
			\centering
			\begin{lstlisting}[language = Racket]
(make-<Name> [Parameter-Daten])
\end{lstlisting}
		\end{figure}
		Für die Parameter müssen wir die Daten in der korrekten Reihenfolge wie in der Struct-Definition übergeben.
		
		Um auf bestimmte Attribute eines Structs \texttt{x} zuzugreifen, nutzen wir folgenden Code:
		\begin{figure}[H]
			\centering
			\begin{lstlisting}[language = Racket]
(<Name>-<Attribut> x)
\end{lstlisting}
		\end{figure}
		Dies gibt den Wert des jeweiligen Attributs zurück.
		
		\paragraph{Beispiel}
			Wir nehmen als Beispiel wieder das Studierenden-Struct her. Nun wollen wir eine Funktion anlegen, die den Namen des Studierenden ausgibt, zwei Structs anlegen und die Funktion aufrufen.
			
			\begin{figure}[H]
				\centering
				\begin{lstlisting}[language = Racket]
(define (print-name x) (print (student-name)))

(define fd (make-student "Fabian Damken"  1234567))
(define fk (make-student "Florian Kadner" 8912345))
(define lr (make-student "Lukas Roehrig"  6789123))

(print-name fd)
(print-name fk)
(print-name lr)
\end{lstlisting}
			\end{figure}
		% end
	% end
% end

% end

\section{Testen}
	\implements{Tests}{testing}{Racket}

Auch in Racket ist es natürlich möglich, Tests zu schreiben.

Hierzu haben wir drei grundlegende Funktionen zur Verfügung, die wie folgt heißen (auch hier wieder nur eine Auswahl, eine gesamte Liste steht in Abschnitt \ref{sec:racket_summary}):
\begin{itemize}
	\item \lstinline[language = Racket]|(check-expect <Ausdruck> <Erwartetes Ergebnis>)| Testet, ob der Ausdruck das erwartete Ergebnis produziert.
	\item \lstinline[language = Racket]|(check-within <Ausdruck> <Erwartetes Ergebnis> <Delta>)| Testet, ob der Ausdruck das erwartete Ergebnis \( \pm\text{Delta} \) produziert.
	\item \lstinline[language = Racket]|(check-error <Ausdruck> <Erwartete Fehlermeldung>)| Testet, ob der Ausdruck einen Fehler mit der erwarteten Meldung produziert.
\end{itemize}

% end

\section{Funktionen als Daten/Parameter}
	\todo{Schreiben}
% end

\section{Funktionen höherer Ordnung}
	\todo{Schreiben}

\subsection{Lambdas}
	\todo{Schreiben}
% end

\subsection{Beispiele}
	\todo{Schreiben}

	\subsubsection{Filter}
		\todo{Schreiben}
	% end
	
	\subsubsection{Map}
		\todo{Schreiben}
	% end
	
	\subsubsection{Fold}
		\todo{Schreiben}
	% end
	
	\subsubsection{Vergleich von zwei Listen}
		\todo{Schreiben}
	% end
% end
% end

\section{Dokumentation}
	\implements{Dokumentation}{doku}{\racket}

\subsection{Veträge}
	\implements{Verträgen}{vetraege}{\racket}
	
	Nehmen wir an, wir haben eine Funktion \texttt{(moving-average data n)} implementiert, welche den gleitenden Durchschnitt einer Liste \texttt{data} mit den vorherigen \texttt{n} Daten berechnet. Diese Funktion gibt eine Liste mit der Länge \( \text{Length}(\texttt{moving-average}) - (n - 1) \) zurück.
	
	Den Vertrag der Funktion schreiben wir nun wie folgt in den Code:
	\begin{figure}[H]
		\centering
		\begin{lstlisting}[language = Racket]
;; moving-average :: (listof number) number -> (listof number)
(define (moving-average data n) *@\dots@*)
\end{lstlisting}
	\end{figure}
% end

\subsection{Funktionsdokumentation}
	Eine vollständige Funktionsdokumentation besteht aus:
	\begin{itemize}
		\item Einer Beschreibung, was die Methode tut.
		\item Dem Vertrag der Methode, wie oben beschrieben.
		\item Mindestens einem Nutzungsbeispiel.
	\end{itemize}

	Somit ist die folgende Methode korrekt dokumentiert:
	\begin{figure}[H]
		\centering
		\begin{lstlisting}[language = Racket]
;; factorial :: number -> number
;;
;; Berechnet die Fakultaet einer gegebenen natuerlichen Zahl.
;;
;; Beispiele:
;;   (factorial 3) -> 2
;;   (factorial 5) -> 120
(define (factorial n)
	(if (= n 1)
		1
		(* n (factorial (- n 1)))
	)
)
\end{lstlisting}
	\end{figure}
% end
% end

\section{Zusammenfassung}
	\label{sec:racket_summary}

\todo{Schreiben}
% end

    % end

    %\chapter{Java}
	%    \label{c:java}
    %
    %    \section{Lexikalische Bestandteile}
	\subsection{Datentypen}
	\implements{Datentypen}{datentypen}{Java}
	
	Java ist eine statisch Typisierte Sprache, das heißt der Typ einer Variable muss jederzeit angegeben werden und dem Compiler bekannt sein. Es ist nicht möglich, eine Variable nacheinander für zum Beispiel Zahlen und Zeichenketten zu verwenden.
	
	In Java existieren viele Datentypen, die in zwei Kategorien unterteilt werden können:
	\begin{itemize}
		\item Primitive Datentypen
		\item Objektreferenzen
	\end{itemize}
	
	\paragraph{Primitive Datentypen}
		Einer der Unterschiede zwischen primitiven Datentypen und Objektreferenzen ist, dass Daten, welche in primitiven Datentypen gespeichert sind, mit Pass-by-Value weitergegeben werden. Das bedeutet, die Daten werden bei einer Übergabe an die Methode kopiert und Änderungen an den Daten an einer Stelle wirken sich nicht auf andere Stelle aus. Außerdem sind ist die Anzahl an primitiven Datentypen begrenzt und die Datentypen sind von vornherein festgelegt. Ferner gibt es große Unterschiede bei der Behandlung von Konstanten, die wir später betrachten werden. \todo{Primitive vs. Objekte: Konstanten} Als ersten Anhaltspunkt eignet sich, dass primitive Datentypen mit einem kleinen Buchstaben und Objektreferenztypen mit einem großen Buchstaben beginnen.
		
		Es existieren folgende primitive Datentypen:
		\begin{table}[H]
			\centering
			\begin{tabular}{l | l | l | l}
				Schlüsselwort       & Typ            & Beschreibung               & Wertebereich                                                      \\ \hline
				\lstinline|byte|    & Ganzzahl       & Vorzeichenbehaftet, 8 Bit  & \( -(2 ^ { 7}) \) bis \( 2 ^ { 7} - 1 \)                          \\
				\lstinline|short|   & Ganzzahl       & Vorzeichenbehaftet, 16 Bit & \( -(2 ^ {15}) \) bis \( 2 ^ {15} - 1 \)                          \\
				\lstinline|int|     & Ganzzahl       & Vorzeichenbehaftet, 32 Bit & \( -(2 ^ {31}) \) bis \( 2 ^ {31} - 1 \)                          \\
				\lstinline|long|    & Ganzzahl       & Vorzeichenbehaftet, 64 Bit & \( -(2 ^ {63}) \) bis \( 2 ^ {63} - 1 \)                          \\
				\lstinline|float|   & Fließkommazahl & einfache Genauigkeit       & \( 1,4 \cdot 10 ^ {-45} \) bis \( \approx 3,4 \cdot 10 ^ {38} \)  \\
				\lstinline|double|  & Fließkommazahl & doppelte Genauigkeit       & \( 4,9 \cdot 10 ^ {324} \) bis \( \approx 1,8 \cdot 10 ^ {308} \) \\
				\lstinline|char|    & Charakter      & Unicode-Code, 16 Bit       & \( 0 \) bis \( 2 ^ {16} - 1 \)                                    \\
				\lstinline|boolean| & Wahrheitswert  &                            & \texttt{true}/\texttt{false}
			\end{tabular}
			\caption{Liste der primitiven Datentypen in Java}
		\end{table}
		
		Hierbei fällt auf, dass es in Java keinen eingebauten Datentyp für vorzeichenfreie Zahlen (\enquote{unsigned}) gibt. Dies kann bei der Verarbeitung von Binärdaten (beispielsweise bei Netzwerkkommunikation) zu Fehlern führen.
	% end
	
	\paragraph{Objektreferenzen}
		Neben primitiven Datentypen gibt es noch die Objektreferenzen. Diese verhalten sich anders als primitive Datentypen, wobei uns vor allem die folgenden Unterschiede auffallen:
		\begin{itemize}
			\item Es gibt als einziges \enquote{echtes} Literal den Wert \lstinline|null|, der aussagt, dass die Objektreferenz kein Objekt referenziert. Jegliche Methodenaufrufe auf dieser Referenz brechen mit einer \lstinline|NullPointerException| ab. Deshalb muss vor jedem Zugriff auf eine solche Referenz geprüft werden, ob sie ungleich \lstinline|null| ist.
			\item Werden Objektreferenzen als Parameter übergeben, so wird hierbei ausschließlich die Referenz übergeben und auf das gleiche Objekt referenziert (Pass-by-Reference). Das bedeutet, eine Änderung an dem Objekt an einer Stelle kann sich an beliebig vielen anderen Stellen auswirken.
		\end{itemize}
		Wir werden uns Objektreferenzen im Abschnitt \refImpl{oop}{Java} über objektorientierte Programmierung in Java nochmals genauer anschauen.
	% end
	
	\paragraph{Sonderfall \lstinline|String|}
		Ein \lstinline|String| ist eine Objektreferenz, kann allerdings in manchen Bereichen als ein primitiver Datentyp angesehen werden. Beispielsweise existieren, wir wir weiter unten noch sehen werden, Literale für diesen Datentyp, welche implizit ein Objekt erzeugen. Auch scheint ein String mit Pass-by-Value übergeben zu werden, da ein String nicht veränderbar ist.
		
		Insgesamt existieren folgende Unterschiede:
		\begin{itemize}
			\item Ein String ist \textit{immutable}, das heißt nicht veränderlich.
			\item Dadurch scheint es, als wird ein String Pass-by-Value übergeben.
			\item Bei Konstanten wird ein String als primitiver Typ angesehen und gleich behandelt.
			\item Es gibt eine syntaktische Form, String-Literale auszudrücken.
			\item Trotz allem ist es möglich, einem String den Wert \lstinline|null| zuzuweisen. Dies ist gleichermaßen praktisch wie nervig.
		\end{itemize}
	% end
% end

\subsection{Literale}
	\implements{Literalen}{literale}{Java}
	
	In Java gibt es Schreibweisen für Literale für alle Datentypen, wobei die Erstellung von Objekten einen Sonderfall darstellt und nicht vollständig als Literal bezeichnet werden kann (es können zwar alle Argumente fest im Code stehen, das Objekt selbst wird allerdings erst zur Laufzeit erstellt).
	
	In der folgenden Tabelle sind sämtliche syntaktische Methoden zu Definition von Literalen gelistet:
	\begin{table}[H]
		\centering
		\begin{tabular}{l | l}
			\textbf{Datentyp} & \textbf{Schreibweise}                          \\ \hline
			\texttt{byte}     & \texttt{123}, \texttt{-123}                    \\
			\texttt{short}    & \texttt{1234}, \texttt{-1234}                   \\
			\texttt{int}      & \texttt{12345}, \texttt{-12345}                 \\
			\texttt{long}     & \texttt{123456L}, \texttt{123456L}               \\
			\texttt{float}    & \texttt{12.34F}, \texttt{0.34F}/\texttt{.34F}  \\
			\texttt{double}   & \texttt{123.456}, \texttt{0.456}/\texttt{.456} \\
			\texttt{char}     & \texttt{'a'}                                   \\
			\texttt{boolean}  & \texttt{true}, \texttt{false}                  \\
			\texttt{String}   & \texttt{"Hello, World!"}                       \\
			\texttt{Object}   & \texttt{null}
		\end{tabular}
	\end{table}
	\begin{itemize}
		\item \texttt{String} ist hier kein primitiver Datentyp, das heißt mit einem String-Literal wird auch immer ein neues Objekt erzeugt.
		\item Das Literal \texttt{null} für \texttt{Object} ist allgemein anwendbar, wenn mit Objekten gearbeitet wird. Allerdings kann dies zu unerwarteten \textit{\texttt{NullPointerExceptions}} führen, welche wir später noch eingehend betrachten werden.
	\end{itemize}
	
	Bei Literalen von Zahlen gibt es außerdem folgende Besonderheiten:
	\begin{itemize}
		\item Bei einem \texttt{float}-Literal muss ein \enquote{\texttt{F}} am Ende des Literals angehängt werden, damit das Literal als \texttt{float} und nicht als \texttt{double} interpretiert wird. Die Groß-/Kleinschreibung ist irrelevant.
		\item Bei einem \texttt{long}-Literal kann ein \enquote{\texttt{L}} am Ende des Literals angehängt werden, damit das Literal als \texttt{long} und nicht als \texttt{int} interpretiert wird. Die Groß-/Kleinschreibung ist irrelevant, aufgrund der Ähnlichkeit von \enquote{\texttt{l}} und \enquote{\texttt{1}} wir allerdings ein großes \enquote{\texttt{L}} empfohlen.
		\item Bei allen Ganzzahlen (\texttt{byte}, \texttt{short}, \texttt{int}, \texttt{long}) können die Zahlen mit den Zahlensystemen Binär, Oktal, Dezimal und Hexadezimal eingegeben werden, wobei Dezimal sinnvollerweise der Standard ist. Zur Nutzung hiervon müssen den Werten bestimmte Zeichenketten vorangestellt werden. Dies sind \texttt{0b} für Binär, \texttt{0} für Oktal, nichts für Dezimal und \texttt{0x} für Hexadezimal. \\ Das heißt, die folgenden Literale sind äquivalent:
			\begin{itemize}
				\item \texttt{0b101010}
				\item \texttt{052}
				\item \texttt{42}
				\item \texttt{0x2A}
			\end{itemize}
			Wobei auch hier die Groß-/Kleinschreibung irrelevant ist, für den Prefix allerdings die Kleinschreibung und für die Zahl die Großschreibung empfohlen wird.
	\end{itemize}
	
	\warning{Wird bei Zahlen eine \texttt{0} vorangestellt, wird die Zahl Oktal interpretiert! Das heißt es gilt \texttt{010 \(\neq\) 10}.}
	
	\paragraph{Escape-Sequenzen}
		Escape-Sequenzen werden innerhalb eines Strings mit einem Backslash (\textbackslash) eingeleitet und bestehen in den meisten Fällen auf einem Zeichen.
		
		In Java sind folgende Escape-Sequenzen verfügbar:
		\begin{table}[H]
			\centering
			\begin{tabular}{c | l}
				\textbf{Escape-Sequenz} & \textbf{Repräsentiertes Zeichen} \\ \hline
				\lstinline|\t|          & Tab.                             \\
				\lstinline|\b|          & Backspace.                       \\
				\lstinline|\n|          & New line.                        \\
				\lstinline|\r|          & Carriage Return.                 \\
				\lstinline|\f|          & Formfeed.                        \\
				\lstinline|\'|          & Single quote.                    \\
				\lstinline|\"|          & Double quote.                    \\
				\lstinline|\\|          & Backslash.
			\end{tabular}
			\caption{Java: Escape-Sequenzen}
		\end{table}
	% end
% end

\subsection{Schlüsselwörter}
	\implements{Schlüsselwörtern}{keywords}{Java}
	
	In Java existieren folgende Schlüsselwörter (kursiv geschriebene Themen werden wir nicht ausführlicher betrachten):
	\begin{description}
        \item[\texttt{abstract}] Markiert eine\dots
	        \begin{description}
	        	\item[Klasse] das heißt, diese kann abstrakte Methoden enthalten.
	        	\item[Methode] die von Unterklassen implementiert werden muss.
	        \end{description}
        \item[\texttt{continue}] Fährt in einer Schleife mit dem nächsten Element fort.
        \item[\texttt{for}] Leitet eine for-Schleife ein.
        \item[\texttt{new}] Operator zur Erstellung eines neuen Objektes einer Klasse.
        \item[\texttt{switch}] Leitet eine switch-Anweisung ein.
        \item[\texttt{assert}] Legt bestimmte Bedingungen fest, die für Parameter gelten müssen. Gelten diese nicht, wird ein Fehler ausgelöst.
        \item[\texttt{default}]
	        \begin{itemize}
	        	\item Default-Fall in einer switch-Anweisung.
	        	\item Definition einer Default-Methode innerhalb eines Interfaces.
	        	\item \textit{Definition des Default-Wertes einer Methode in einer Annotation}
	        \end{itemize}
        \item[\texttt{if}] Leitet eine if-Verzweigung ein.
        \item[\texttt{package}] Definition des Packages einer Klasse.
        \item[\texttt{synchronized}] Markiert eine Methode oder einen Codeblock als synchron, das heißt es kann maximal ein Thread zur gleichen Zeit die Methode \enquote{betreten}.
        \item[\texttt{boolean}] Datentyp.
        \item[\texttt{do}] Leitet eine do-while-Schleife ein.
        \item[\texttt{goto}] Reserviert. Löst ausschließlich einen Compilefehler aus.
        \item[\texttt{private}] Markiert eine Klasse, einen Konstruktor, eine Methode oder ein Attribut als privat.
        \item[\texttt{this}] Referenz auf die Instanz des aktuellen Objektes.
        \item[\texttt{break}] Bricht die Ausführung einer Schleife ab.
        \item[\texttt{double}] Datentyp.
        \item[\texttt{implements}] Implementiert ein Interface.
        \item[\texttt{protected}] Markiert eine Klasse, einen Konstruktor, eine Methode oder ein Attribut als protected.
        \item[\texttt{throw}] Wirft eine Instanz einer Exception.
        \item[\texttt{byte}] Datentyp.
        \item[\texttt{else}] Leitet einen else-Block ein.
        \item[\texttt{import}] Importiert eine Klasse/Methode aus einem anderen Paket.
        \item[\texttt{public}]  Markiert eine Klasse, einen Konstruktor, eine Methode oder ein Attribut als public.
        \item[\texttt{throws}] Definiert, dass ein Konstruktor/eine Methode eine bestimmte Exception werfen kann.
        \item[\texttt{case}] Leitet einen Fall eines switch-Ausdruckes ein.
        \item[\texttt{enum}] Leitet die Definition eines Enums ein.
        \item[\texttt{instanceof}] Operator zum Prüfen, ob eine Instanz eine Instanz einer anderen Klasse ist.
        \item[\texttt{return}] Gibt einen Wert zurück und bricht die Ausführung der Methode/des Konstruktors ab.
        \item[\textit{\texttt{transient}}] \textit{Definiert, dass ein bestimmte Attribut einer Instanz nicht mit serialisiert wird.}
        \item[\texttt{catch}] Leitet einen catch-Block ein.
        \item[\texttt{extends}]
	        \begin{description}
	        	\item[Klasse] Erweitert eine bestehende (möglicherweise abstrakte) Klasse.
	        	\item[Interface] Erweitert ein bestehendes Interface.
	        \end{description}
        \item[\texttt{int}] Datentyp.
        \item[\texttt{short}] Datentyp.
        \item[\texttt{try}] Leitet einen try-Block ein.
        \item[\texttt{char}] Datentyp.
        \item[\texttt{final}]
	        \begin{description}
	        	\item[Klasse] Die Klasse ist nicht vererbbar.
	        	\item[Methode] Die Methode ist nicht überschreibbar.
	        	\item[Variable] Die Variable ist nur einmal zuweisbar.
	        \end{description}
        \item[\texttt{interface}] Leitet die Definition eines Interfaces ein.
        \item[\texttt{static}] Markiert eine innere Klasse, eine Methode oder ein Attribut als statisch.
        \item[\texttt{void}] \enquote{Datentyp} als Platzhalter für \enquote{Nichts}.
        \item[\texttt{class}] Leitet die Definition einer Klasse ein.
        \item[\texttt{finally}] Leitet einen finally-Block ein.
        \item[\texttt{long}] Datentyp.
        \item[\textit{\texttt{strictfp}}] \textit{Legt fest, dass innerhalb einer Methode/einer Klasse nur strikte mathematische Operationen verwendet werden, sodass diese nicht optimiert werden sollen (es wird sich strikt an den Standard gehalten).}
        \item[\textit{\texttt{volatile}}] \textit{Markiert ein Attribut, sodass Modifikationen an diesem atomar durchgeführt werden und für andere Threads direkt sichtbar sind.}
        \item[\texttt{const}] Reserviert. Löst ausschließlich einen Compilefehler aus.
        \item[\texttt{float}] Datentyp.
        \item[\textit{\texttt{native}}] \textit{Markiert die Implementierung einer Methode als nativ, das heißt, die Implementierung liegt in nativem Code (C/C++) vor. Siehe JNI (Java Native Interface).}
        \item[\texttt{super}] Referenz auf die Instanz der Oberklasse des aktuellen Objektes.
        \item[\texttt{while}] Leitet eine while-Schleife ein.
	\end{description}
	
	Die genaue Bedeutung der obigen Schlüsselwörter werden wir in den jeweiligen Kapiteln genauer betrachten.
% end

\subsection{Bezeichner und Konventionen}
	\implements{Bezeichnern und Namenskonventionen}{identifier}{Java}
	
	In Java können Zeichenketten als Bezeichner gelten, wenn sie folgenden Bedingungen genügen:
	\begin{itemize}
		\item Sie bestehen nur aus \texttt{a} bis \texttt{z}, \texttt{0} bis \texttt{9}, \texttt{\_} oder \texttt{\$}.
		\item Sie beginnen nur mit \texttt{a} bis \texttt{z}, \texttt{\_} oder \texttt{\$}.
	\end{itemize}
	
	Zur Benennung sind außerdem folgende Konventionen zu empfehlen:
	\begin{itemize}
		\item Namen von Klassen beginnen mit einem Großbuchstaben.
		\item Namen von Methoden/Parametern/Variablen/etc. beginnen mit einem Kleinbuchstaben.
		\item Namen von Klassen sollen Subjekte und Objekte sein. Beispiel: \enquote{\texttt{User}}
		\item Namen von Methoden sollen mit einem Verb beginnen. Beispiel: \enquote{\texttt{generateAccessToken}}
	\end{itemize}
	
	\info{Die oben Bedingungen, wann eine Zeichenkette als Bezeichner dienen kann, stellen Vereinfachungen dar. Streng genommen können alle Zeichen verwendet werden, für die die Methoden \texttt{Character.isJavaIdentifierStart(char)} bzw. \texttt{Character.isJavaIdentifierPart(char)} den Wert \texttt{true} ergeben. Damit \textit{wären} auch Bezeichner wie \enquote{\texttt{\(\Delta\Psi\)}} möglich.}
% end

\subsection{Operatoren}
	\implements{Operatoren}{lexOperatoren}{Java}
	
	In Java existieren die folgenden Operatoren, die genaue Bedeutung werden wir im Abschnitt \refImpl{operatoren}{Java} behandeln:
	\begin{table}[H]
		\centering
		\begin{tabular}{l | l}
			Kategorie & Ausprägungen \\
			\hline
			Arithmetische Verknüpfungen & \texttt{*}, \texttt{/}, \texttt{\%}, \texttt{+}, \texttt{-} \\
			Unäre Arithmetik            & \texttt{expr++}, \texttt{expr--}, \texttt{++expr}, \texttt{--expr}, \texttt{+expr}, \texttt{-expr} \\
			Logik                       & \texttt{!}, \texttt{\&\&}, \texttt{||}, \texttt{\textasciicircum}, \texttt{?:} \\
			Bitweise Logik              & \texttt{\textasciitilde}, \texttt{\&}, \texttt{|}, \texttt{\textasciicircum} \\
			Verschiebung (Shift)        & \texttt{<{}<}, \texttt{>{}>}, \texttt{>{}>{}>} \\
			Vergleiche                  & \texttt{<}, \texttt{>}, \texttt{<=}, \texttt{>=}, \texttt{==}, \texttt{!=}, \texttt{instanceof} \\
			Zuweisungen                 & \texttt{=}, \texttt{+=}, \texttt{-=}, \texttt{*=}, \texttt{/=}, \texttt{\%=}, \texttt{\textasciicircum=}, \texttt{|=}, \texttt{<{}<=}, \texttt{>{}>=}, \texttt{>{}>{}>=} \\
		\end{tabular}
		\caption{Java: Operatoren}
	\end{table}
% end

\subsection{Strukturierung des Codes, Packages und Imports}
	\implements{Paketen und Code-Strukturierung}{namespaces}{Java}
	
	\subsubsection{Kommentare}
		Es gibt drei verschiedene Arten von Kommentaren in Java:
		\begin{itemize}
			\item Einzeilige Kommentare \\ Der Kommentar ist nur in der aktuellen Zeile gültig.
			\item Blockkommentare \\ Der Kommentar ist gültig, bis der Kommentar explizit beendet wird (auch über Zeilenumbrüche hinweg).
			\item Javadoc \\ Dies ist eine besondere Form eines Blockkommentars, der so nur vor einer Methode, einem Feld oder einem Typ stehen kann (und, um es ganz genau zu nehmen, vor der \lstinline|package|-Deklaration in einer Datei \texttt{package-info.java}). Diese besondere Form von Kommentaren wird genutzt, um den Code zu Dokumentieren und anschließend eine HTML-Dokumentation daraus zu generieren. Wir werden dies im Abschnitt \refImpl{doku}{Java} über Javadoc intensiver anschauen.
		\end{itemize}
	% end
	
	\subsubsection{Whitespaces}
		Jegliche Whitespaces (Tab, Zeilenumbruch, Leerzeichen) sind in Java optional und dienen nur der Strukturierung des Codes. Trotz dass dies überflüssig ist, empfiehlt es sich, wenn wir unseren Code einrücken, gezielt Zeilenumbrüche setzen und so unseren Code lesbar machen.
	% end
	
	\subsubsection{Klammerung}
		In Java werden jegliche verfügbaren Klammern genutzt ((), {}, [], <>). Sie haben die folgenden Zwecke:
		\begin{description}
			\item[\texttt{()}] Klammerung von Ausdrücken (um die Operatorenpräzedenz festzulegen), Aufrufen von Methoden/Konstruktoren, Notwendig bei If-Ausdrücken, Switch-Case und Schleifen.
			\item[\texttt{\{\}}] Kennzeichnung von Codeblöcken (Klassen-/Methodendefinition, If-Ausdrücke, Schleifen, Switch-Case, \dots).
			\item[\texttt{[]}] Zugriff auf die Elemente eines Arrays und Erstellung von Arrays.
			\item[\texttt{<>}] Vergleichsoperatoren und Generics.
		\end{description}
	% end
	
	\subsubsection{Packages und Imports}
		Als übergeordnete Strukturierung unserer Klassen existiert das Konstrukt von \textit{Packages}, mit denen Klassen gruppiert und somit logisch zusammengefasst werden können. Beispielsweise können wir alle Klassen, die mit dem Zugriff auf eine Datenbank zu tun haben in ein Package \texttt{database} legen, und die Klassen, die mit dem User Interface zu tun haben in ein Package \texttt|ui|. Der Name eines Packages muss ein gültiger Bezeichner sein. Um eine Klasse in einem Package abzulegen, muss die Klasse zum einen mit der Zeile \lstinline|package /* Packagename */;| starten und außerdem im korrekten Ordner liegen (das heißt eine Klasse im Package \texttt{ui} muss in einem Ordner \texttt{ui} liegen).
		
		Damit entsteht eine logische Trennung und das Projekt wird übersichtlicher. Der \textit{voll-qualifizierte Klassen} ist dann der Name der Klasse mit dem vorangestellten Package-Namen. Hierdurch werden Kollisionen in der Klassenbenennung vermieden. Beispiel: Unsere Klasse \texttt{Connection} liegt im Package \texttt{database}. Dann ist der voll-qualifizierte Name dieser Klasse \texttt{database.Connection}. Nutzen wir innerhalb einer Klasse eine andere Klasse, die nicht im selben Package und nicht im Package \texttt{java.lang} liegt, so müssen wir entweder den voll-qualifizierten Klassennamen bei jeder Verwendung angeben oder die Klasse mit dem Ausdruck \lstinline|import /* Voll-qualifizierter Klassenname */;| importieren. Dann können wir die Klasse überall nutzen, als wäre sie im gleichen Package. Standardmäßig sind alle Klassen aus \lstinline{java.lang} importiert.
		
		Um Packages von Unterpackages zu trennen, können wir Punkte innerhalb des Package-Namen nutzen. Somit können wir beispielsweise Klassen, die mit der Datenbank und konkret mit MySQL zu tun haben, in ein Package \lstinline{database.mysql} legen.
		
		\paragraph{Konvention}
			Zur Vermeidung von Kollisionen ist es üblich, allen Packages innerhalb eines Projektes den umgekehrten Namen der Domain voranzustellen, für die das Projekt entwickelt wird (Bindestriche oder andere nicht-Java-konforme Zeichen werden dabei durch einen Unterstrich ersetzt, siehe §3.8 Java-Spezifikation). Danach folgt der Projektname.
			
			Bei der Entwicklung von Nabla für das Fachgebiet Algorithmik am Fachbereich Informatik an der TU Darmstadt sollte also der Packagename \lstinline|de.tu_darmstadt.informatik.algo.nabla| vorangestellt werden.
		% end
	% end
% end

% end

\section{Anweisungen}
	Schauen wir uns nun an, wie man Dinge in Java tut, also wie wir Anweisungen und Ausdrücke formulieren können.

\subsection{Variablen}
	\implements{Variablen}{variablen}{Java}
	
	Die allgemeine Syntax zur Deklaration einer Variablen ist:
	\begin{figure}[H]
		\centering
		\lstinline|<modifier> <typ> <name>;|
	\end{figure}
	Dabei ist \texttt{<modifier>} eine Reihe von Schlüsselwörtern, welche das Verhalten der Variablen modifizieren (genannt \enquote{Modifier}). Diese werden wir uns weiter unten genau anschauen. \texttt{<typ>} ist der Datentyp der Variablen (dies kann ein primitiver Datentyp aber auch ein Referenztyp sein). Der Name der Variablen wird mit \texttt{<name>} festgelegt.
	
	\subsubsection{Modifier}
		Für eine lokale Variable (das heißt eine Variable innerhalb eines Codeblocks oder als Parameter) existiert ausschließlich folgender Modifier:
		\begin{description}
			\item[\texttt{final}] Sorgt dafür, dass die Variable nur einmal zugewiesen werden kann (zum Beispiel direkt nach oder noch während der Deklaration). Wenn möglich sollte eine Variable immer als \lstinline|final| markiert werden, um versehentliches Überschreiben des Wertes zu verhindern.
		\end{description}
		Handelt es sich bei der Variablen um eine Instanz- oder Klassenvariable, sind zusätzlich folgende Modifier verfügbar:
		\begin{description}
			\item[\texttt{volatile}] Bei der Zuweisung der Variablen geschieht die Zuweisung \textit{atomar}. Dieser Modifier kann nicht mit \lstinline|final| modifiziert werden.
			\item[\texttt{transient}] Bei der Serialisierung einer Instanzvariablen wird dieses Feld nicht serialisiert.
			\item[\(\bullet\)] Sämtliche Sichtbarkeitsmodifizierer (siehe \ref{sec:visibility}).
		\end{description}
		Alle Modifier können wir mit kleinen Einschränkungen beliebig kombinieren.
		
		Beispiel: Eine Definition einer privaten Klassenvariable \texttt{timestamp}, die atomar Zugewiesen werden soll und nicht mit serialisiert werden soll sieht so aus:
		\begin{figure}[H]
			\centering
			\lstinline|private static transient volatile long timestamp;|
		\end{figure}
	% end
	
	% TODO: Schreiben
	%\subsubsection{Lokale Variablen, Konstanten, Attribute, Arraykomponenten}
	%	\todo{Schreiben}
	%% end
	
	\subsubsection{Null- und Defaultwerte}
		Klassenvariablen, die nicht \lstinline|final| sind, werden bestimmte Default-Werte zugewiesen (sofern die Variable nicht während der Deklaration direkt zugewiesen wird):
		\begin{table}[H]
			\centering
			\begin{tabular}{l | l}
				\textbf{Typ} & \textbf{Default-Wert} \\ \hline
				\lstinline|byte| & \lstinline|0| \\
				\lstinline|short| & \lstinline|0| \\
				\lstinline|int| & \lstinline|0| \\
				\lstinline|long| & \lstinline|0| \\
				\lstinline|float| & \lstinline|0.0F| \\
				\lstinline|double| & \lstinline|0.0| \\
				\lstinline|boolean| & \lstinline|false| \\
				\lstinline|char| & \lstinline|'\000'| (Null-Byte) \\
				\lstinline|Object| und Unterklassen & \lstinline|null| \\
			\end{tabular}
			\caption{Java: Defaultwerte}
		\end{table}
	% end
% end

\subsection{Zuweisungen}
	\implements{Zuweisungen}{zuweisungen}{Java}
	
	Um eine Variable zuzuweisen, wird folgender Ausdruck verwendet:
	\begin{figure}[H]
		\centering
		\lstinline|<variable> = <ausdruck>;|
	\end{figure}
	Dabei ist der linke Teil \texttt{<variable>} der Name der Variablen, welcher der Wert des Ausdrucks \texttt{<ausdruck>} zugewiesen wird. Der Ausdruck kann dabei beliebig komplex sein.
	
	Wie können den Wert auch zeitgleich mit der Deklaration zuweisen, die Syntax ist dann wie folgt:
	\begin{figure}[H]
		\centering
		\lstinline|<modifier> <typ> <name> = <ausdruck>;|
	\end{figure}

	Eine Besonderheit ist hier, dass der Ausdruck einer normalen Zuweisung den Wert der Zuweisung zurück gibt (das heißt es gilt \texttt{(<variable> = <ausdruck>) == <ausdruck>}).
% end

\subsection{Methodenaufrufe}
	\implements{Methodenaufrufen}{methodenNutzung}{Java}
	
	Der allgemeine Ausdruck, um eine Methode in Java aufzurufen ist:
	\begin{figure}[H]
		\centering
		\lstinline|<objekt>.<methodenname>([parameter], [parameter], ...)|
	\end{figure}
	Der Methodenname muss immer gegeben sein, ebenso wie das Objekt (beziehungsweise bei einer statischen Methode die Klasse), welches/welche das Objekt enthält. Die Parameter müssen gegeben sein, wenn die aufgerufene Methode dies fordert, es gibt aber auch Methoden, die keine Parameter erfordern.
	
	Wir können die Rückgabe der Methode auch einer Variablen zuweisen, die Syntax ist dann wie folgt:
	\begin{figure}[H]
		\centering
		\lstinline|<variable> = <objekt>.<methodenname>([parameter], [parameter], ...)|
	\end{figure}
	Dies ist nur möglich, wenn die Methode einen Rückgabetyp hat, das heißt der Rückgabetyp nicht \lstinline|void| ist.
% end

\subsection{Operatoren}
	\implements{Operatoren}{operatoren}{Java}
	
	\subsubsection{Arithmetik-Operatoren}
		Es existieren die folgenden arithmetischen Operatoren, die allesamt alle primitiven und numerischen Datentypen (\lstinline|byte|, \lstinline|short|, \lstinline|int|, \lstinline|long|, \lstinline|float|, \lstinline|double|) annehmen:
		\begin{table}[H]
			\centering
			\begin{tabular}{c | l | l}
				\textbf{Operator} & \textbf{Syntax}                & \textbf{Beschreibung}                                    \\ \hline
				\texttt{++}       & \texttt{a++}, \texttt{++a}     & \texttt{a} wird um 1 \textit{inkrementiert}.             \\
				\texttt{-{}-}     & \texttt{a-{}-}, \texttt{a-{}-} & \texttt{a} wird um 1 \textit{dekrementiert}.             \\
				\texttt{*}        & \texttt{a * b}                 & \texttt{a} und \texttt{b} werden \textit{multipliziert}. \\
				\texttt{/}        & \texttt{a / b}                 & \texttt{a} wird durch \texttt{b} \textit{dividiert}.     \\
				\texttt{\%}       & \texttt{a \% b}                & Es wird \( \texttt{a} \textbf{ mod } \texttt{b} \) berechnet (d.h. \( \texttt{a} - \big\lfloor \frac{\texttt{a}}{\texttt{b}} \big\rfloor \texttt{b} \)) (\textit{Modulo}). \\
				\texttt{+}        & \texttt{a + b}                 & \texttt{a} und \texttt{b} werden \textit{addiert}.       \\
				\texttt{-}        & \texttt{a - b}                 & \texttt{b} wird von \texttt{a} \textit{subtrahiert}.     \\
				\texttt{-}        & \texttt{-a}                    & Negiert das Vorzeichen von \texttt{a}.
			\end{tabular}
		\end{table}
		Bei den Inkrementierungs-/Dekrementierungs-Operatoren ist der Unterschied zwischen den Syntaxen \texttt{a++} und \texttt{++a} (beziehungsweise \texttt{a-{}-} und \texttt{-{}-a}), dass das Ergebnis von ersterem Ausdruck den Wert von \texttt{a} vor der Inkrementierung/Dekrementierung und \texttt{++a}/\texttt{-{}-a} den Wert nach der Inkrementierung/Dekrementierung als Ergebnis liefert (Postfix vs. Prefix Operatoren). Das bedeutet, dass \texttt{a++ == a}, \texttt{a-{}- == a}, \texttt{++a == a + 1} und \texttt{-{}-a == a - 1} gelten.
		
		\paragraph{Kommazahlen und Division}
			Eine Division wird immer als \textit{Ganzzahldivision} durchgeführt, wenn nicht mindestens einer der Parameter eine Fließkommazahl ist. Das bedeutet, dass Nachkommastellen nur berechnet werden, wenn mindestens einer der Parameter ein \lstinline|float| oder \lstinline|double| ist.
			
			Eine Ganzzahldivision von \(a\) und \(b\) entspricht \( \big\lfloor \frac{a}{b} \big\rfloor \), dass heißt, die Nachkommastellen werden abgeschnitten.
		% end
	% end
	
	\subsubsection{Logik- und Vergleichs-Operatoren}
		Es existieren die folgenden logischen Operatoren und Vergleichsoperatoren, die alle als Ergebnis ein \lstinline|boolean| zurück geben.
		\begin{table}[H]
			\centering
			\begin{tabular}{c | l | l | l}
				\textbf{Operator} & \textbf{Syntax}   & \textbf{Parametertyp} & \textbf{Beschreibung}                                         \\ \hline
				   \texttt{<}     & \texttt{a < b}    & primitive Zahl        & Ist \texttt{a} kleiner \texttt{b}?                            \\
				   \texttt{>}     & \texttt{a > b}    & primitive Zahl        & Ist \texttt{a} größer \texttt{b}?                             \\
				   \texttt{<=}    & \texttt{a <= b}   & primitive Zahl        & Ist \texttt{a} kleiner-gleich \texttt{b}?                     \\
				   \texttt{>=}    & \texttt{a >= b}   & primitive Zahl        & Ist \texttt{a} größer-gleich \texttt{b}?                      \\
				   \texttt{==}    & \texttt{a == b}   & Beliebig              & Ist \texttt{a} identisch zu \texttt{b}?                       \\
				   \texttt{!=}    & \texttt{a != b}   & Beliebig              & Ist \texttt{a} nicht identisch zu \texttt{b}?                 \\
				  \texttt{\&\&}   & \texttt{a \&\& b} & Wahrheitswert         & Verknüpft \texttt{a} und \texttt{b} mit einem logischem UND.  \\
				   \texttt{||}    & \texttt{a || b}   & Wahrheitswert         & Verknüpft \texttt{a} und \texttt{b} mit einem logischem ODER. \\
				   \texttt{\^}    & \texttt{a \^{} b} & Wahrheitswert         & Verknüpft \texttt{a} und \texttt{b} mit einem logischem XOR.  \\
				   \texttt{!}     & \texttt{!a}       & Wahrheitswert         & Negiert den Wahrheitswert von \texttt{a}
			\end{tabular}
		\end{table}
		\textit{Identisch} bedeutet für Zahlen, dass diese bis auf die letzte Nachkommastelle gleich sind. Für Objekte bedeutet dies, dass es ein und das selbe Objekt sind (das heißt, dass die Speicheradresse identisch ist). Eine Änderung auf \texttt{a} ändert somit auch \texttt{b}, wenn \texttt{a == b} gilt (nur bei Objekten!). Aufgrund dessen ist es auch nicht möglich, Strings mit \texttt{==} zu vergleichen, da dies bei Benutzereingaben oder ähnlichem immer \lstinline|false| liefern würde, da die Objekte nur den gleichen Inhalt haben und nicht identisch sind (siehe auch \ref{sec:equals_identity}).
	% end
	
	\subsubsection{Bitlogik-Operatoren}
		Die bitlogischen Operatoren können auf primitive Ganzzahlen (\lstinline|byte|, \lstinline|short|, \lstinline|int|, \lstinline|long|) angewendet werden. Diese wenden die üblichen logischen Verknüpfungen auf Bit-Ebene an, dass heißt, die Zahl wird in Binärdarstellung überführt und die Verknüpfung der Reihe nach auf jedes Bit einzeln angewendet (bei ungleich großen Datentypen werden die fehlenden Stellen bei dem kleineren mit Nullen aufgefüllt). Der Rückgabetyp entspricht immer dem größeren Datentyp. Es existieren die folgenden Operatoren:
		\begin{table}[H]
			\centering
			\begin{tabular}{c | l | l}
				\textbf{Operator} & \textbf{Syntax}      & \textbf{Beschreibung}                                                                               \\ \hline
				  \texttt{<{}<}   & \texttt{a <{}< b}    & Verschiebt die Bits von \texttt{a} um \texttt{b} Stellen nach links.                                \\
				  \texttt{>{}>}   & \texttt{a >{}> b}    & Verschiebt die Bits von \texttt{a} um \texttt{b} Stellen nach rechts.                               \\
				\texttt{>{}>{}>}  & \texttt{a >{}>{}> b} & Verschiebt die Bits von \texttt{a} um \texttt{b} Stellen nach rechts und behält das Vorzeichen bei. \\
				   \texttt{\&}    & \texttt{a \& b}      & Verknüpft die Bits von \texttt{a} und \texttt{b} mit einer UND-Verknüpfung.                         \\
				   \texttt{\^}    & \texttt{a \^{} b}    & Verknüpft die Bits von \texttt{a} und \texttt{b} mit einer XOR-Verknüpfung.                         \\
				   \texttt{|}     & \texttt{a | b}       & Verknüpft die Bits von \texttt{a} und \texttt{b} mit einer ODER-Verknüpfung.                        \\
				\texttt{\(\sim\)} & \texttt{\(\sim\)a}   & Negiert die Bits von \texttt{a}.
			\end{tabular}
		\end{table}
	% end
	
	\subsubsection{Spezielle Operatoren}
		Zusätzlich zu den oben genannten Operatoren gibt es noch die Operatoren \lstinline|new|, \lstinline|instanceof| und der Ternäre Operator, die etwas anders funktionieren.
		\begin{description}
			\item[\texttt{\color{lstkeywords} new}] Mit diesem Operator können neue Instanzen (Objekte) einer Klasse erstellt werden und die allgemeine Syntax lautet \lstinline|new <klasse>([parameter], [parameter], ...)|; diesen Operator werden wird im Abschnitt \ref{sec:constructor} genauer betrachten.
			\item[\texttt{\color{lstkeywords} instanceof}] Mit diesem Operator kann geprüft werden, ob ein Objekt eine Instanz einer bestimmten Klasse darstellt, die allgemeine Syntax hierfür lautet \lstinline|<objekt> instanceof <klasse>|. Beispielsweise wäre für eine Variable \lstinline|Number x = 1.2| der Ausdruck \lstinline|x instanceof Double| wahr, der Ausdruck \lstinline|x instanceof Integer| jedoch falsch.
			\item[Ternärer Operator] Mit diesem Operator können, ähnlich wie bei einem If, Fallunterscheidungen vorgenommen werden. Die allgemeine Syntax lautet \lstinline|<test> ? <wahr-fall> : <sonst-fall>|. Dabei wird zuerst der Test ausgewertet, ist dieser Wahr, so wird das Ergebnis von dem wahr-Fall zurück gegeben, sonst das Ergebnis von dem sonst-Fall. Dabei muss der Test zu einem Wahrheitswert auswerten und die beiden Fälle zu dem gleichen Typ, beziehungsweise einem kompatiblen Typ für den äußeren Ausdruck.
		\end{description}
	% end
	
	\subsubsection{Bindungsstärke der Operatoren}
		Die Bindungsstärke der Operatoren in Java gliedert sich wie folgt, wobei die oberste Zeile die stärkste Bindungsstärke hat und mehrere Elemente auf einer Zeile die gleiche Bindungsstärke:
		\begin{enumerate}
			\item \texttt{expr++}, \texttt{expr--}
			\item \texttt{++expr}, \texttt{--expr}, \texttt{+expr}, \texttt{-expr}, \texttt{\(\sim\)}, \texttt{!}
			\item \texttt{*}, \texttt{/}, \texttt{\%}
			\item \texttt{+}, \texttt{-}
			\item \texttt{<{}<}, \texttt{>{}>}, \texttt{>{}>{}>}
			\item \texttt{<}, \texttt{>}, \texttt{<=}, \texttt{>=}, \texttt{instanceof}
			\item \texttt{==}, \texttt{!=}
			\item \texttt{\&}
			\item \texttt{\^}
			\item \texttt{|}
			\item \texttt{\&\&}
			\item \texttt{||}
			\item \texttt{? :}
			\item \texttt{=}, \texttt{+=}, \texttt{-=}, \texttt{*=}, \texttt{/=}, \texttt{\%=}, \texttt{\&=}, \texttt{\^{}=}, \texttt{|=}, \texttt{<{}<=}, \texttt{>{}>=}, \texttt{>{}>{}>=}
		\end{enumerate}
	% end
	
	\subsubsection{Klammerung}
		Um die Bindungsstärke von Operatoren zu beeinflussen, können Ausdrücke wie in der Mathematik geklammert werden, wobei die innerste Klammer immer zuerst ausgewertet wird. Hierfür dürfen ausschließlich runde Klammern (\texttt{(}, \texttt{)}) genutzt werden.
	% end
% end

\subsection{Implizite und Explizite Typenkonversion (Casts)}
	Schauen wir uns zuerst einmal an, was wir unter einer Typenkonversion verstehen: Wenn wir eine Variable \lstinline|int a = 41| haben, können wir diese Problemlos einer anderen Variable mit dem Datentyp \lstinline|long| zuweisen (\lstinline|long b = a|). Hier liegt uns eine \textit{implizite Typenkonversion} vor, bei der der Datentyp \lstinline|int| zu einem \lstinline|long| umgewandelt wird. Wir gehen nun getrennt auf primitive Typenkonversionen, Wrapper-Typen und Objektkonversionen ein.
	
	\subsubsection{Primitive Typen}
		Eine primitive Typenkonversion haben wir bereits gesehen. Eine implizite Typenkonversion ist immer dann möglich, wenn der neue Datentyp eine größere oder gleiche Datenmenge halten kann wie der alte Datentyp (das heißt es ist zum Beispiel nicht implizit möglich, eine Fließkommazahl in eine Ganzzahl zu konvertieren).
		
		\begin{figure}[H]
			\centering
			\begin{tikzpicture}[main/.style = { draw, rectangle, minimum height = 0.9cm, minimum width = 2cm }]
				\node [main] (byte) {\lstinline|byte|};
				\node [main, right = 2 of byte] (short) {\lstinline|short|};
				\node [main, right = 2 of short] (int) {\lstinline|int|};
				\node [main, right = 2 of int] (long) {\lstinline|long|};
				\node [main, below = 2 of int] (float) {\lstinline|float|};
				\node [main, right = 2 of float] (double) {\lstinline|double|};
				\node [main, above = 2 of short] (char) {\lstinline|char|};
				
				\draw [->] (char) -| (int);
				
				\draw [->] (byte) -- (short);
				\draw [->] (short) -- (int);
				\draw [->] (int) -- (long);
			
				\coordinate [below = 1 of long] (needle);
				\draw (long) -- (needle);
				\draw [->] (needle) -| (float);
				
				\draw [->] (float) -- (double);
			\end{tikzpicture}
		\end{figure}
		Der Pfeil \( A \rightarrow B \) bedeutet, dass \(A\) implizit in \(B\) konvertiert werden kann. Der Rückweg ist ausgeschlossen. Außerdem ist die Konvertierung transitiv, dass bedeutet, wenn \( A \rightarrow B \) und \( B \rightarrow C \), dann geht auch \( A \rightarrow C \).
		
		Eine explizite Konvertierung wird vorgenommen, indem der neue Typ in Klammern vor die Variable (oder den Ausdruck) des alten Typs gesetzt wird:
		\begin{figure}[H]
			\centering
			\lstinline|(<neuer-typ>) <ausdruck>|
		\end{figure}
		Beispielsweise Wertet der Ausdruck \( 1 / 2.0 \) zu einem \lstinline|double| aus und das Ergebnis muss explizit in ein \lstinline|int| konvertiert werden: \lstinline|(int) (1 / 2.0)|. Das Ergebnis wäre in diesem Falle \lstinline|0|, da bei einer Typenkonvertierung von einer Fließkommazahl in eine Ganzzahl die Nachkommastellen abgeschnitten werden.
	% end
	
	\subsubsection{Wrappertypen}
		Wie wir im Abschnitt zu Generics (\ref{sec:generics}) sehen werden, sind primitive Typen nicht immer hilfreich. Manchmal möchten wir auch Zahlen oder ähnliches in Objekten speichern können. Hier kommen die sogenannten \textit{Wrappertypen} ins Spiel, die ebenso wir Strings immutable, das heißt nicht veränderlich, sind.
		
		Wrappertypen sind Klassen, die eine primitive Variable speichern und diese bei Bedarf zur Verfügung stellt. Die Verwendung dieser Wrapper Typen erfolgt durch \textit{Autoboxing} transparent, das heißt, eine Variable wird automatisch in einem Wrappertyp gespeichert und gelesen.
		
		Die Namen der Wrappertypen entsprechen zu großen Teilen dem Namen des primitiven Typs mit einem großem Anfangsbuchstaben (die Klassen liegen allesamt in dem Package \lstinline|java.lang|):
		\begin{table}[H]
			\centering
			\begin{tabular}{l | l}
				\textbf{Primitiver Typ} & \textbf{Wrappertyp}   \\ \hline
				\lstinline|byte|        & \lstinline|Byte|      \\
				\lstinline|short|       & \lstinline|Short|     \\
				\lstinline|int|         & \lstinline|Integer|   \\
				\lstinline|long|        & \lstinline|Long|      \\
				\lstinline|float|       & \lstinline|Float|     \\
				\lstinline|double|      & \lstinline|Double|    \\
				\lstinline|char|        & \lstinline|Character| \\
				\lstinline|boolean|     & \lstinline|Boolean|
			\end{tabular}
		\end{table}
	
		\paragraph{Autoboxing}
			Weisen wir einer Variable \lstinline|Object obj| einen primitiven Wert (zum Beispiel \lstinline|1.2|) zu, so wird dieser primitive Typ automatisch in den entsprechenden Wrappertyp konvertiert und der Variable zugewiesen. Ebenfalls wird an Stellen, an denen primitive Typen gebraucht werden (zum Beispiel in arithmetischen Operationen oder Vergleichen) der Wrappertyp zurück in einen primitiven Wert gewandelt.
			
			Beispiel:
			\begin{figure}[H]
				\centering
				\begin{lstlisting}
double primitive = 1.2;
int wholeNumber = (int) x;
Double wrapper = primitive;   // Autoboxing.
if (wrapper > wholeNumber) {  // Autounboxing.
	...
}
\end{lstlisting}
			\end{figure}
		
			\warning{Im Gegensatz zu primitiven Typen können Variablen von Wrappertypen \lstinline|null| sein. Wird versucht, Autounboxing auf \lstinline|null|-Werten anzuwenden, so wird eine \lstinline|NullPointerException| geworfen.}
		% end
	% end
	
	\subsubsection{Objekte (\enquote{Downcast})}
		\label{sec:downcast}
	
		Auch bei Objekten müssen wir manchmal eine Typenkonvertierung vornehmen. Implizite Typenkonvertierungen sind hier genau dann möglich, wenn der neue Typ eine Oberklasse des alten Typs ist. Eine explizite Typenkonvertierung wird benötigt, wenn in der Klassenhierarchie \enquote{nach unten} gegangen werden soll (dies wird \textit{Downcast} genannt). Eine explizite Typenkonvertierung findet wie bei primitiven Typen statt indem der neue Typ in Klammern vor den Ausdruck geschrieben wird.
		
		Schauen wir uns dies am Beispiel eines Strings an:
		\begin{figure}[H]
			\centering
			\begin{tikzpicture}
				\umlemptyclass{Object}
				\umlemptyclass[below = 1 of Object]{CharSequence}
				\umlemptyclass[below = 1 of CharSequence]{String}
				
				\umlinherit{String}{CharSequence}
				\umlinherit{CharSequence}{Object}
			\end{tikzpicture}
		\end{figure}
		Eine implizite Typenkonvertierung ist nun immer nach oben in der Hierarchie möglich (also \( \texttt{String} \rightarrow \texttt{CharSequence} \rightarrow \texttt{Object} \)).
		
		Beispiel:
		\begin{figure}[H]
			\centering
			\begin{lstlisting}
String s = "Hello, World!";
Object o = s;                // Implizite Typenkonvertierung.
String casted = (String) o;  // Explizite Typenkonvertierung (Downcast).
\end{lstlisting}
		\end{figure}
	% end
% end

% TODO: Schreiben
%\subsection{Links-/Rechtsausdrücke}
%	\todo{Schreiben}
%% end

% TODO: Schreiben
%\subsection{Seiteneffekte}
%	\todo{Schreiben}
%% end

% end

\section{Kontrollstrukturen}
	\subsection{Verzweigungen}
	\implements{Verzweigungen}{verzweigungen}{Java}

	In Java gibt es als grundlegende Art der Verzweigung nur das einfache If. Ein Switch, welches wir uns später anschauen werden, baut sehr direkt auf einem If auf reduziert größtenteils die Tipparbeit.

	\subsubsection{If}
		In \texttt{if} hat in Java die folgende Form:
		\begin{figure}[H]
			\centering
			\begin{lstlisting}
if (/* Test */) {
	/* then-Fall */
} else if (/* else-Test */) {
	/* else-then-Fall */
} else {
	/* else-Fall */
}
			\end{lstlisting}
			\caption{Java: \texttt{if}-Verzweigung}
		\end{figure}
		
		Dabei kann es beliebig viele Else-Ifs geben, oder diese können ganz weg gelassen werden. Ebenfalls kann der Else-Fall weggelassen werden, einzig und allein der Then-Fall ist nötig (dieser kann theoretisch auch leer sein, dies ergibt aber in den meisten Fällen keinen Sinn).
		
		Somit ist die einfache Form des \texttt{if}s:
		\begin{figure}[H]
			\centering
			\begin{lstlisting}
if (/* Test */) {
	/* then-Fall */
}
			\end{lstlisting}
			\caption{Java: Einfache \texttt{if}-Verzweigung}
		\end{figure}
		
		Alle Tests (die Bedingungen für das If) \textit{müssen} zu einem \texttt{boolean} auswerten. Alle anderen Datentypen werden nicht akzeptiert und der Code wird nicht kompilieren.
	% end
	
	\subsubsection{Switch}
		Ein \texttt{switch} stellt eine Vereinfachung von vielen If-Else-Verzweigungen dar, welche alle die gleiche Operation (beispielsweise das Vergleichen von zwei Objekten) ausführen.
		
		Schauen wir uns als Motivation den folgenden Code an, welcher ausgibt, wie viele Primzahlen \texttt{p} mit \texttt{p <= x <= 10} existieren.
		\begin{figure}[H]
			\centering
			\begin{lstlisting}
if (x == 1) {
	prime = 0;
} else if (x == 2) {
	prime = 1;
} else if (x == 3) {
	prime = 2;
} else if (x == 4) {
	prime = 2;
} else if (x == 5) {
	prime = 3;
} else if (x == 6) {
	prime = 3;
} else if (x == 7) {
	prime = 4;
} else if (x == 8) {
	prime = 4;
} else if (x == 9) {
	prime = 4;
} else if (x == 10) {
	prime = 4;
} else {
	// Error.
}
			\end{lstlisting}
			\caption{Java: \texttt{switch} Motivation}
		\end{figure}
		
		Mit Hilfe eines Switches können wir den Code nun äquivalent umformen:
		\begin{figure}[H]
			\centering
			\begin{lstlisting}
switch (x) {
case 1:
	prime = 0;
	break;
case 2:
	prime = 1;
	break;
case 3:
	prime = 2;
	break;
case 4:
	prime = 2;
	break;
case 5:
	prime = 3;
	break;
case 6:
	prime = 3;
	break;
case 7:
	prime = 4;
	break;
case 8:
	prime = 4;
	break;
case 9:
	prime = 4;
	break;
case 10:
	prime = 4;
	break;
default:
	// Error.
	break;
}
			\end{lstlisting}
			\caption{Java: \texttt{switch}}
		\end{figure}
		
		Der Code ist äquivalent zu der If-Else-Kaskade und ist einfacher zu verstehen. Allerdings müssen wir in jedem Case (ein einzelner Fall in dem Switch-Konstrukt) ein \texttt{break} platzieren, welches die Ausführung des Switches abbricht. Wird das \texttt{break} weggelassen, so \textit{fällt die Ausführung durch}, das heißt es wird einfach mit dem nächsten Case fortgefahren.
		
		Um den Nutzen hiervon verstehen zu können müssen wir uns darüber im klaren sein, wie ein Switch ausgewertet wird:
		\begin{itemize}
			\item Im ersten Schritt wird die \textit{Vergleichsvariable} in den Klammern hinter dem Schlüsselwort \texttt{switch} ausgewertet. Diese Variable darf nur zu einem String, einem primitiven Wert oder dem Wert eines Enums auswerten.
			\item Anschließend wird der erhaltene Wert mit dem Wert eines jeden Cases verglichen. Hierfür ist es nötig, das hinter dem Schlüsselwort \texttt{case} ausschließlich Literale oder Konstanten mit dem gleichen Typ stehen. Es kann niemals zwei Cases mit dem gleichen Wert (auch genannt \textit{Label}) geben!
			\item Nun wird zur ersten Zeile des Cases gesprungen, dessen Wert gleich dem Vergleichswert ist. Existiert kein solcher Case, so wird zu dem Default-Case gesprungen, welcher aber nicht existieren muss. Wird kein Code zur Ausführung gefunden, so wird das gesamte Switch übersprungen.
			\item Sämtlicher nachfolgender Code wird nun ausgeführt, bis ein \texttt{break} gefunden wird. Dann wird aus dem Switch gesprungen un nach dem Switch fortgefahren. Außerdem beenden Returns und Exceptions wie üblich die Ausführung.
		\end{itemize}
		
		Das heißt: Beim Start des Cases wird zu einer Stelle im Code gesprungen und dieser so lange ausgeführt, bis die Ausführung \textit{explizit} beendet wird oder kein Code mehr im Switch existiert.
		
		Mit dieser Kenntnis kann obiger Code des Switches deutlich vereinfacht werden, indem wir einfach bei jeder Primzahl den Zähler um eins erhöhen:
		\begin{figure}[H]
			\centering
			\begin{lstlisting}
prime = 0;
switch (x) {
	case 10:
	case 9:
	case 8:
	case 7:
		prime++;
	case 6:
	case 5:
		prime++;
	case 4:
	case 3:
		prime++;
	case 2:
		prime++;
	case 1:
		break;
	default:
		// Error.
		break;
}
			\end{lstlisting}
			\caption{Java: \texttt{switch} mit Fall-Thru}
		\end{figure}
		
		\warning{Ein Case in einem Switch öffnet keinen neuen Scope! Somit können Variablen nur einmal genutzt werden oder es muss ein Block um den Case geschrieben werden.}
	% end
% end

\subsection{Schleifen}
	\implements{Schleifen}{schleifen}{Java}
	
	\subsubsection{While-Schleife}
		In Java sieht die zuvor vorgestellte While-Schleife wie folgt aus:
		\begin{figure}[H]
			\centering
			\begin{lstlisting}
while (/* Test */) {
	/* Code */
}
			\end{lstlisting}
			\caption{Java: \texttt{while}-Schleife}
		\end{figure}
		
		Wie auch schon beim If gesehen, darf auch hier der Test nur zu einem \texttt{boolean} und nicht zu anderen Datentypen ausgewertet werden. Der Test wird \textit{vor} jedem Schleifendurchlauf ausgeführt und bricht ab, sobald er zu \texttt{false} auswertet.
	% end
	
	\subsubsection{Do-While-Schleife}
		Als Spezialfall einer While-Schleife gibt es in Java die Do-While-Schleife, welche immer \textit{mindestens einmal} ausgeführt wird:
		\begin{figure}[H]
			\centering
			\begin{lstlisting}
do {
	/* Code */
} while (/* Test */);
			\end{lstlisting}
			\caption{Java: \texttt{do-while} Schleife}
		\end{figure}
		
		Wie bei allen Schleifen und Bedingungen darf auch hier der Test nur zu einem \texttt{boolean} auswerten. Im Gegensatz zur While-Schleife wird der Test hier allerdings \textit{nach} jedem Schleifendurchlauf ausgeführt, wodurch die Schleife immer mindestens einmal ausgeführt wird.
	% end
	
	\subsubsection{For-Schleife}
		Da oftmals über Elemente einer Liste oder eines Arrays iteriert wird und der Code hierfür immer gleich ist (eine Zählvariable wird in jedem Schritt hochgezählt):
		\begin{figure}[H]
			\centering
			\begin{lstlisting}
int i = 0;
while (i < array.length) {
	/* Code */

	i++;
}
			\end{lstlisting}
			\caption{Java: \texttt{for each}-Schleife Motivation}
		\end{figure}
		kann dieser Code zu folgendem, äquivalentem, Code umgewandelt werden:
		\begin{figure}[H]
			\centering
			\begin{lstlisting}
for (int i = 0; i < array.length; i++) {
	/* Code */
}
			\end{lstlisting}
			\caption{Java: \texttt{for each}-Schleife Motivation}
		\end{figure}
		womit Code eingespart wird und die Iteration deutlich übersichtlicher ist.
		
		Die einzelnen Bestandteile des Schleifenkopfes, welche mit Semikola getrennt werden müssen, haben folgende Namen und Funktionen:
		\begin{description}
			\item[\texttt{int i = 0}] \textit{Initialisierung} - Der Code an dieser Stelle wird \textit{einmalig vor} Durchlauf der Schleife ausgeführt.
			\item[\texttt{i < array.length}] \textit{Test} - Ein zu \texttt{boolean} auswertender Ausdruck, welcher \textit{vor jedem} Durchlauf ausgewertet wird. Wird die Bedingung zu \texttt{false} ausgewertet, wird die Schleife beendet.
			\item[\texttt{i++}] \textit{Schritt} - Der Code an dieser Stelle wird \texttt{nach jedem} Durchlauf ausgeführt.
		\end{description}
		
		Ferner können alle Bestandteile der For-Schleife ausgelassen werden (unter Beibehaltung der Semikola!), wobei Initialisierung und Schritt einfach nicht ausgeführt werden und die Bedingung immer zu \texttt{true} auswertet. Somit sind \enquote{\texttt{while (true) \{ \}}} und \enquote{\texttt{for (;;) \{ \}}} äquivalent.
	% end
	
	\subsubsection{\enquote{Erweiterte} For-Schleife}
		Statt wie üblich über Arrays und Listen zu iterieren:
		\begin{figure}[H]
			\centering
			\begin{lstlisting}
for (int i = 0; i < array.length; i++) {
	Object element = array[i];

	/* Code */
}
			\end{lstlisting}
			\caption{Java: Erweiterte For-Schleife Motivation}
		\end{figure}
		kann seit Java 5 auch die \textit{erweiterte For-Schleife} verwendet werden, um über Arrays oder Instanzen von \texttt{java.util.Iterable} iterieren (alle Standard-Klassen für Listen implementieren dieses Interface):
		\begin{figure}[H]
			\centering
			\begin{lstlisting}
for (Object element : array) {
	/* Code */
}
			\end{lstlisting}
			\caption{Java: Erweiterte For-Schleife}
		\end{figure}
		Dabei wird immer über den konkreten Typ, der im Array (oder der Liste) gespeichert ist, iteriert.
		
		Anders ausgedrückt: Der Code im Schleifenkörper wird für jedes Element des Arrays oder der Liste ausgeführt.
	% end
	
	\subsubsection{\texttt{break}, \texttt{continue}}
		Um eine Schleifenausführung vorzeitig auszuführen, gibt es die folgenden Schlüsselwörter:
		\begin{description}
			\item[\texttt{break}] Bricht die gesamte Schleifenausführung der innerstmöglichen Schleife ab.
			\item[\texttt{continue}] Fährt mit der nächsten Iteration der innerstmöglichen Schleife fort.
		\end{description}
		
		\paragraph{Beispiel}
			Schauen wir uns abschließend folgendes schwachsinniges Beispiel an, um die Funktionalität von \texttt{break} und \texttt{continue} zu verdeutlichen. Der folgende Code summiert die ungeraden Elemente eines Arrays, wobei keine weiteren Element aufsummiert werden, sobald die Summe einmal \texttt{10} überschritten hat.
			\begin{figure}[H]
				\centering
				\begin{lstlisting}
int sumOdd = 0;
for (int x : array) {
	if (x % 2 == 0) {
		// Element is even --> continue with next element.
		continue;
	}

	if (sumOdd > 10) {
		// Sum is over 10 --> stop loop.
		break;
	}

	sumOdd += x;
}
System.out.println(sumOdd);
				\end{lstlisting}
				\caption{Java: \texttt{break}, \texttt{continue} Beispiel}
			\end{figure}
			Mit den Werten \texttt{array = new int[] \{ 1, 2, 3, 4, 5, 7, 8, 9 \}} wird \texttt{16} ausgegeben, wobei der Code wie folgt ausgeführt wird:
			\begin{figure}[H]
				\centering
				\begin{lstlisting}
(Zeile =  1; sumOdd =  0)

(Zeile =  2; sumOdd =  0; x = 1)
(Zeile =  3; sumOdd =  0; x = 1; (x % 2 == 0) = false)
(Zeile =  8; sumOdd =  0; x = 1; (sumOdd > 10) = false)
(Zeile = 13; sumOdd =  1; x = 1)

(Zeile =  2; sumOdd =  1; x = 2)
(Zeile =  3; sumOdd =  1; x = 2; (x % 2 == 0) = true)

(Zeile =  2; sumOdd =  1; x = 3)
(Zeile =  3; sumOdd =  1; x = 3; (x % 2 == 0) = false)
(Zeile =  8; sumOdd =  1; x = 3; (sumOdd > 10) = false)
(Zeile = 13; sumOdd =  4; x = 3)

(Zeile =  2; sumOdd =  4; x = 4)
(Zeile =  3; sumOdd =  4; x = 4; (x % 2 == 0) = true)

(Zeile =  2; sumOdd =  4; x = 5)
(Zeile =  3; sumOdd =  4; x = 5; (x % 2 == 0) = false)
(Zeile =  8; sumOdd =  4; x = 5; (sumOdd > 10) = false)
(Zeile = 13; sumOdd =  9; x = 5)

(Zeile =  2; sumOdd =  9; x = 6)
(Zeile =  3; sumOdd =  9; x = 6; (x % 2 == 0) = true)

(Zeile =  2; sumOdd =  9; x = 7)
(Zeile =  3; sumOdd =  9; x = 7; (x % 2 == 0) = false)
(Zeile =  8; sumOdd =  9; x = 7; (sumOdd > 10) = false)
(Zeile = 13; sumOdd = 16; x = 7)

(Zeile =  2; sumOdd = 16; x = 8)
(Zeile =  3; sumOdd = 16; x = 8; (x % 2 == 0) = true)

(Zeile =  2; sumOdd = 16; x = 9)
(Zeile =  3; sumOdd = 16; x = 9; (x % 2 == 0) = false)
(Zeile =  8; sumOdd = 16; x = 9; (sumOdd > 10) = true)

(Zeile = 15; sumOdd = 16)
				\end{lstlisting}
				\caption{Java: \texttt{break}, \texttt{continue} Beispielausführung}
			\end{figure}
		% end
	% end
% end

% end

\section{Methoden}
	\implements{Methoden}{methoden}{Java}

\textit{In Java sind Methoden immer an ein Objekt oder eine Klasse gebunden. Die Unterschiede hierzwischen werden wir uns später im Abschnitt \refImpl{oop}{Java} zu objektorientierter Programmierung in Java anschauen. In diesem Kapitel werden wir annehmen, dass sich alle Methoden in einer Klasse befinden, eine Instanz der Klasse vorliegt und die Methoden an diese Instanz gebunden sind.}

Betrachten wir zur Einführung die folgende Methode:
\begin{figure}[H]
	\centering
	\begin{lstlisting}
int add(int a, int b) {
	return a + b;
}
	\end{lstlisting}
\end{figure}
die die Summe der Zahlen \texttt{a} und \texttt{b} berechnet.

Dabei entspricht \textit{int add(int a, int b)} dem Kopf der Methode und alles in den geschweiften Klammern (also \texttt{return a + b;}) dem Körper der Methode.

\subsubsection{Der Methodenkopf}
	Ein Methodenkopf hat folgenden allgemeinen Aufbau:
	\begin{center}
		\texttt{[MODIFIER] [GENERICS] <RÜCKGABETYP> <METHODENNAME>([PARAMETER])}
	\end{center}
	dabei ist die Angabe von Modifizierern (\textit{Modifier}), Generics und Parametern optional, wobei beliebig viele Parameter angegeben werden können. Die Klammern hinter dem Methodennamen müssen dennoch vorhanden sein, auch wenn keine Parameter angegeben werden.
	
	\paragraph{Modifizierer}
		Die Modifizierer, die an einer Methode angegeben werden können, werden wir uns im Kapitel über objektorientierte Programmierung genauer anschauen, da die in Java vorhanden Modifizierer nur in diesem Kontext Sinn ergeben.
		
		\textbf{Erweitertes Wissen:} Eine Ausnahme stellt der Modifizierer \texttt{strictfp} dar, der der JVM aufträgt, arithmetische Operationen exakt wie in der Spezifikation der JVM vorzunehmen und nicht zu optimieren.
	% end
	
	\paragraph{Generics}
		Siehe \ref{sec:generics}.
	% end
	
	\paragraph{Rückgabetyp}
		Hiermit geben wir den Typ an, den das Ergebnis unserer Methode hat. Dies kann ein primitiver Datentyp oder eine Klasse sein. Wird nichts zurückgegeben, muss \texttt{void} angegeben werden, was so viel wie \enquote{nichts} heißt.
	% end
	
	\paragraph{Methodenname}
		Dies ist der Name der Methode, mit dem wir selbige referenzieren können. Der Name muss sich an die Grundregeln von Bezeichnern in Java halten (siehe \refImpl{identifier}{Java}).
	% end
	
	\paragraph{Parameter}
		Die ist eine Komma-separierte Liste von Parametern, die unsere Funktion erwartet.
		
		Einer dieser Parameter ist dabei aufgebaut wir eine normale Variablendeklaration, das heißt \texttt{[final] <DATENTYP> <NAME>}. Ein hier verwendeter Name kann im Körper der Methode nicht erneut für Variablen genutzt werden, der Zugriff auf den Wert des Parameters erfolgt, als wäre dieser eine ganz normale Variable. Ein in der Parameterliste angegebenes \texttt{final} verhält sich entsprechend.
		
		\subparagraph{Varargs}
			Varargs sind eine spezielle Form der Parameter, die dem Aufrufer erlauben, beliebig viele Parameter zu übergeben.
			
			Betrachten wir hierzu folgendes Beispiel, um beliebig viele Zahlen zu addieren:
			\begin{figure}[H]
				\centering
				\begin{lstlisting}
int add(int[] numbers) {
	int result = 0;
	for (int x : numbers) {
		result += x;
	}
	return result;
}
				\end{lstlisting}
			\end{figure}
			Ein Aufrufer müsste die Funktion zum Beispiel so aufrufen: \lstinline|add(new int[] { 1, 2, 3, 4, 5 })|, das heißt der müsste erst ein Array erstellen und dieses der Funktion übergeben. Dies stellt einen erheblichen Schreibaufwand dar.
			
			Hätten wir stattdessen unsere Funktion wie folgt mit Varargs gestaltet, vereinfacht sich der Aufruf, wie wir gleich sehen werden:
			\begin{figure}[H]
				\centering
				\begin{lstlisting}
int add(int... numbers) {
	int result = 0;
	for (int x : numbers) {
		result += x;
	}
	return result;
}
				\end{lstlisting}
			\end{figure}
			nun Vereinfacht sich der funktional identische Aufruf zu \lstinline|add(1, 2, 3, 4, 5)|.
			
			Wir sehen auch, dass sich an dem Körper unserer Funktion nichts geändert hat, einzig und allein die manuelle Erstellung des Arrays verschwindet. Konkret heißt dies, dass Java uns die Arbeit abnimmt, das Array manuell zu erstellen, sondern dies im Hintergrund erledigt. Wenn der Aufrufer unbedingt will, kann er dennoch ein einfaches Array übergeben.
			
			\warning{Es ist nicht möglich, nach einem Vararg-Parameter noch weitere Parameter anzugeben, da Java sonst nicht wüsste, welche Parameter noch zum Varargs gehören und welche nicht. Vor einem Vararg-Parameter ist dies problemlos möglich.}
		% end
	% end
% end

\subsubsection{Signatur}
	Die Signatur einer Methode muss innerhalb einer Klasse eindeutig sein. Zu der Signatur einer Methoden gehören
	\begin{itemize}
		\item der Methodenname und
		\item die Typen der Parameter.
	\end{itemize}

	Somit sind bei den folgenden Methoden:
	\begin{enumerate}
		\item \lstinline|int add(int a, int b)|
		\item \lstinline|float add(int a, int b)|
		\item \lstinline|float add(float a, float b)|
	\end{enumerate}
	die Methoden 1 und 2 der Signatur nach identisch, die 3. Methode hingegen verschieden. Somit dürften in einer Klasse nur folgende Kombinationen vorkommen:
	\begin{equation*}
		\emptyset \,,\quad \{ 1 \} \,,\quad \{ 1, 3 \} \,,\quad \{ 2 \} \,,\quad \{ 2, 3 \}
	\end{equation*}
% end
% end

\section{Scoping}
	\implements{Scoping und Scopes}{scoping}{Java}

In Java wird immer dann ein neuer Scope geöffnet, wenn eine geschweifte Klammer auf geht und ein Scope geschlossen, wenn die geschweifte Klammer zu geht. Das ist zum Beispiel bei Methoden und Schleifen der Fall.

Das bedeutet: Eine Variable, die wir innerhalb einer Methode definieren (inklusive der Parameter) ist von außerhalb nicht zugreifbar. Das gleiche gilt für Variablen, die innerhalb eines If-Blocks oder dem Körper einer Schleife definiert wurden.

\textbf{Beispiel:} \\
\begin{lstlisting}
int multiply(int a, int b) {
	int result = 0;

	int bAbs = Math.abs(b);
	for (int i = 0; i < Math.abs(b); i++) {
		result += a;
	}

	if (b < 0) {
		return -result;
	} else {
		return result;
	}
}
\end{lstlisting}
Auf die Variablen \texttt{a}, \texttt{b} und \texttt{result} kann von außerhalb der Methode nicht zugegriffen werden. Ebenfalls kann nicht auf \texttt{i} zugegriffen werden, dies ist aber schon außerhalb der Schleife (also in Zeile 1 bis 4 und 8 bis 14) nicht möglich.

Der Wert der Variable \texttt{result} wird zurück gegeben, womit der Aufrufer Zugriff auf den Wert der Variable hat, aber ohne Kenntnis darüber, wie das Ergebnis zustande gekommen ist.
% end

\section{Generische Programmierung}
	\introduces{von generischer Programmierung}{generischeProgrammierung}

\todo{Schreiben}
% end

\section{Datenstrukturen}
	\introduces{von Datenstrukturen}{datastruct}

Eine Datenstruktur ist ein Objekt zur Speicherung und Organisation von Daten, indem diese in einer bestimmten Art und Weise angeordnet sind und es klare Namen gibt, sodass unterschiedliche Entwickler sich über  Datenstrukturen unterhalten können, ohne an eine bestimmte Programmiersprache gebunden zu sein.

\info{In dieser Veranstaltung werden wir Datenstrukturen nur grob behandeln, genauer wird dies in der Veranstaltung \enquote{Algorithmen und Datenstrukturen} behandelt.}

Im allgemeinen unterscheidet man zwischen folgenden Typen von Datenstrukturen:
\begin{description}
	\item[Indexbasiert] Jedes Element innerhalb einer Datenstruktur 
	\item[Nicht Indexbasiert]
\end{description}

\subsection{Arrays, Listen, Mengen} \functionalMark \imperativeMark \oopMark
	Bevor wir uns einige Implementierungen von Arrays, Listen und Mengen anschauen, stellt sich zuerst die Frage, was das eigentlich alles ist?
	
	Alle drei stellen eine Auflistung von Elementen eines beliebigen Typs dar und dienen dazu, beliebig viele und noch nicht zur Compile-Zeit bekannte Elemente in einer Variablen zusammenzufassen. Dabei wird unterschieden zwischen \textit{indexbasierten} und \textit{nicht indexbasierten} Strukturen, wobei bei ersteren jedes Element mit einem Index (einer Zahl) identifiziert werden kann, bei letzteren nicht.
	
	\subsubsection{Array}
		\introduces{von Arrays}{datastructArray}
	
		Ein Array ist eine solche Auflistung, welche in den meisten Sprachen nicht zur Laufzeit vergrößert werden kann und in einigen Sprachen (beispielsweise C) sogar schon zur Laufzeit feststehen muss. Grob gesagt kann man sich ein Array als beschriftetes Kistensystem vorstellen, bei dem jede Kiste eine Zahl zugewiesen bekommt:
		\begin{figure}[H]
			\centering
			\begin{tikzpicture}[->, shorten >= 2pt, every node/.style = { minimum width = 1cm }, elem/.style = { draw, rectangle, minimum width = 2cm }]
				\node (i0) {\texttt{0}};
				\node [elem, right = of i0] (e0) {\enquote{One}};
				
				\node [right = 2 of e0] (i1) {\texttt{1}};
				\node [elem, right = of i1] (e1) {\enquote{Two}};
				
				\node [below = of i0] (i2) {\texttt{2}};
				\node [elem, right = of i2] (e2) {\enquote{Three}};
				
				\node [right = 2 of e2] (i3) {\texttt{3}};
				\node [elem, right = of i3] (e3) {\enquote{Four}};
				
				\draw (i0) -- (e0);
				\draw (i1) -- (e1);
				\draw (i2) -- (e2);
				\draw (i3) -- (e3);
			\end{tikzpicture}
			\caption{Datenstruktur: Array}
		\end{figure}
		
		\warning{In annähernd allen Programmiersprachen werden Arrays ab dem Index \texttt{0} indiziert! Methoden zum Anzeigen der Länge zeigen jedoch die Anzahl der Elemente an und nicht den letzten Index!}
	% end
	
	\subsubsection{Liste}
		\introduces{von Listen}{datastructList}
		
		Eine Liste ist einem Array sehr ähnlich, die Größe kann im Allgemeinen aber zur Laufzeit angepasst werden und es existieren viele verschiedene Implementierungen (unter anderem Implementierungen zur Abbildung auf Arrays, sogenannte Array-Listen). Oftmals ist auch eine Liste indexbasiert, dies muss aber nicht immer der Fall sein (beispielsweise bei gelinkten Listen, die wir später behandeln werden).
	% end
	
	\subsubsection{Menge}
		\introduces{von Sets/Mengen}{datastructSet}
	
		Eine Menge ist, lapidar gesagt, eine Liste ohne Duplikate. Ansonsten gelten genau die gleichen Fakten wie bei einer Liste: Es kann indexbasiert sein, muss es aber nicht, \dots.
	% end
	
	\subsubsection{Linked List (Gelinkte Liste)}
		Eine \textit{gelinkte Liste} ist eine indexlose Liste, in der die Datenspeicherung wie folgt abgebildet wird:
		\begin{itemize}
			\item Jedes Element der Liste enthält die Referenz auf:
				\begin{itemize}
					\item die eigentlichen Nutzdaten (data),
					\item den Nachfolger des Elementes (next) und
					\item im Fall von einer doppelt gelinkten Liste, eine Referenz auf das vorherige Element (previous).
				\end{itemize}
			\item Existiert kein nachfolgendes/vorheriges Element, so wir nichts in das Feld eingetragen.
			\item Manchmal gibt es noch eine Schnittstelle, die eine Referenz auf das erste Element enthält und einige Methoden zur Verfügung stellt (\texttt{first}, \texttt{second}, \texttt{third}, \dots). Diese ist aber nicht vorgeschrieben.
		\end{itemize}
		
		Visualisiert sieht eine einfach gelinkte Liste wie folgt aus:
		\begin{figure}[H]
			\centering
			\begin{tikzpicture}[every node/.style = { align = center }, elem/.style = { draw, rectangle, minimum width = 3cm }]
				\node [elem] (el0) {\texttt{data: } \enquote{One} \\ \texttt{next}};
				\node [elem, right = 1.5 of el0] (el1) {\texttt{data: } \enquote{Two} \\ \texttt{next}};
				\node [elem, right = 1.5 of el1] (el2) {\texttt{data: } \enquote{Three} \\ \texttt{next}};
				\node [elem, right = 1.5 of el2] (el3) {\texttt{data: } \enquote{Four} \\ \texttt{next}};
				
				\draw (el0.west) -- (el0.east);
				\draw (el1.west) -- (el1.east);
				\draw (el2.west) -- (el2.east);
				\draw (el3.west) -- (el3.east);
				
				\coordinate (el0n) at ($(el0.east)!0.5!(el0.south east)$);
				\coordinate (el0a) at ($(el0.north)!0.5!(el1.north)+(0,0.5)$);
				\draw (el0n) -| (el0a);
				\draw [->] (el0a) -| (el1.north);
				
				\coordinate (el1n) at ($(el1.east)!0.5!(el1.south east)$);
				\coordinate (el1a) at ($(el1.north)!0.5!(el2.north)+(0,0.5)$);
				\draw (el1n) -| (el1a);
				\draw [->] (el1a) -| (el2.north);
				
				\coordinate (el2n) at ($(el2.east)!0.5!(el2.south east)$);
				\coordinate (el2a) at ($(el2.north)!0.5!(el3.north)+(0,0.5)$);
				\draw (el2n) -| (el2a);
				\draw [->] (el2a) -| (el3.north);
			\end{tikzpicture}
			\caption{Datenstruktur: Einfach gelinkte Liste}
		\end{figure}
		
		Und das gleiche als doppelt gelinkte Liste:
		\begin{figure}[H]
			\centering
			\begin{tikzpicture}[every node/.style = { align = center }, elem/.style = { draw, rectangle, minimum width = 3cm }]
				\node [elem] (el0) {\texttt{data: } \enquote{One} \\ \texttt{next} \\ \texttt{previous}};
				\node [elem, right = 1.5 of el0] (el1) {\texttt{data: } \enquote{Two} \\ \texttt{next} \\ \texttt{previous}};
				\node [elem, right = 1.5 of el1] (el2) {\texttt{data: } \enquote{Three} \\ \texttt{next} \\ \texttt{previous}};
				\node [elem, right = 1.5 of el2] (el3) {\texttt{data: } \enquote{Four} \\ \texttt{next} \\ \texttt{previous}};
				
				\coordinate (el0tl) at ($(el0.north west)!1/3!(el0.south west)$);
				\coordinate (el0bl) at ($(el0.north west)!2/3!(el0.south west)$);
				\coordinate (el0tr) at ($(el0.north east)!1/3!(el0.south east)$);
				\coordinate (el0br) at ($(el0.north east)!2/3!(el0.south east)$);
				\draw (el0tl) -- (el0tr);
				\draw (el0bl) -- (el0br);
				
                \coordinate (el1tl) at ($(el1.north west)!1/3!(el1.south west)$);
                \coordinate (el1bl) at ($(el1.north west)!2/3!(el1.south west)$);
                \coordinate (el1tr) at ($(el1.north east)!1/3!(el1.south east)$);
                \coordinate (el1br) at ($(el1.north east)!2/3!(el1.south east)$);
				\draw (el1tl) -- (el1tr);
				\draw (el1bl) -- (el1br);
				
                \coordinate (el2tl) at ($(el2.north west)!1/3!(el2.south west)$);
                \coordinate (el2bl) at ($(el2.north west)!2/3!(el2.south west)$);
                \coordinate (el2tr) at ($(el2.north east)!1/3!(el2.south east)$);
                \coordinate (el2br) at ($(el2.north east)!2/3!(el2.south east)$);
				\draw (el2tl) -- (el2tr);
				\draw (el2bl) -- (el2br);
				
                \coordinate (el3tl) at ($(el3.north west)!1/3!(el3.south west)$);
                \coordinate (el3bl) at ($(el3.north west)!2/3!(el3.south west)$);
                \coordinate (el3tr) at ($(el3.north east)!1/3!(el3.south east)$);
                \coordinate (el3br) at ($(el3.north east)!2/3!(el3.south east)$);
				\draw (el3tl) -- (el3tr);
				\draw (el3bl) -- (el3br);
				
				
				\coordinate (el0n) at ($(el0tr)!0.5!(el0br)$);
				\coordinate (el0a) at ($(el0.north)!0.5!(el1.north)+(0,0.5)$);
				\draw (el0n) -| (el0a);
				\draw [->] (el0a) -| (el1.north);
				
				\coordinate (el1n) at ($(el1tr)!0.5!(el1br)$);
				\coordinate (el1a) at ($(el1.north)!0.5!(el2.north)+(0,0.5)$);
				\draw (el1n) -| (el1a);
				\draw [->] (el1a) -| (el2.north);
				
				\coordinate (el2n) at ($(el2tr)!0.5!(el2br)$);
				\coordinate (el2a) at ($(el2.north)!0.5!(el3.north)+(0,0.5)$);
				\draw (el2n) -| (el2a);
				\draw [->] (el2a) -| (el3.north);
				
				
				\coordinate (el1o) at ($(el1bl)!0.5!(el1.south west)$);
				\coordinate (el1b) at ($(el0.south)!0.5!(el1.south)-(0,0.5)$);
				\draw (el1o) -| (el1b);
				\draw [->] (el1b) -| (el0.south);
				
				\coordinate (el2o) at ($(el2bl)!0.5!(el2.south west)$);
				\coordinate (el2b) at ($(el1.south)!0.5!(el2.south)-(0,0.5)$);
				\draw (el2o) -| (el2b);
				\draw [->] (el2b) -| (el1.south);
				
				\coordinate (el3o) at ($(el3bl)!0.5!(el3.south west)$);
				\coordinate (el3b) at ($(el2.south)!0.5!(el3.south)-(0,0.5)$);
				\draw (el3o) -| (el3b);
				\draw [->] (el3b) -| (el2.south);
			\end{tikzpicture}
			\caption{Datenstruktur: Doppelt gelinkte Liste}
		\end{figure}
	% end
% end

\subsection{Map}
	\introduces{von Maps/Dictionaries}{datastructMap}

	Eine \textit{Map}, oder auf \textit{Dictionary} genannt, ist eine Art Liste, welche als Indizes aber jeden beliebigen Typ haben kann (beispielsweise Strings). Damit ist beispielsweise eine Zuordnung von Namen zu Objekten möglich.
	
	\begin{figure}[H]
		\centering
			\begin{tikzpicture}[->, shorten >= 2pt, every node/.style = { minimum width = 2cm }, elem/.style = { draw, rectangle, minimum width = 1cm }]
				\node (i0) {\texttt{"{}One{}"}};
				\node [elem, right = of i0] (e0) {\enquote{0}};
				
				\node [right = 2 of e0] (i1) {\texttt{"{}Two{}"}};
				\node [elem, right = of i1] (e1) {\enquote{1}};
				
				\node [below = of i0] (i2) {\texttt{"{}Three{}"}};
				\node [elem, right = of i2] (e2) {\enquote{2}};
				
				\node [right = 2 of e2] (i3) {\texttt{"{}Four{}"}};
				\node [elem, right = of i3] (e3) {\enquote{3}};
				
				\draw (i0) -- (e0);
				\draw (i1) -- (e1);
				\draw (i2) -- (e2);
				\draw (i3) -- (e3);
			\end{tikzpicture}
		\caption{Datenstruktur: Map/Dictionary}
	\end{figure}
% end

% end

\section{Fehlerbehandlung}
	\introduces{von Fehlerbehandlung}{fehlerbehandlung}

Während der Ausführung von Code kann es immer zu Fehler kommen, welche es zu behandeln gilt (beispielsweise bei einer Division durch \text{0}). Hierbei stellt sich die Frage, wie wir dem Nutzer (oder dem Aufrufer einer Methode) erkenntlich machen, dass es zu einem Fehler gekommen ist.

Dabei gibt es unterschiedliche Möglichkeiten, welche sich grob wie folgt einteilen lassen:
\begin{itemize}
	\item Exceptions und
	\item Result Codes.
\end{itemize}

Diese zwei Unterarten werden wir in den folgenden Abschnitten behandeln.

\subsection{Result Code}
	\introduces{von Result Codes}{resultcodes}

	Beschäftigen wir uns zuerst mit der einfachsten Methode, Fehler anzuzeigen: Wir sagen dem Aufrufer über den Rückgabewert der Methode Bescheid, ob alles korrekt abgelaufen ist.
	
	An einem konkreten Beispiel heißt dies:
	\begin{itemize}
		\item Szenario: Wir haben eine Methode \texttt{indexOf(val: String, el: char): int}, welche uns die Position des ersten Vorkommens von \texttt{el} in \texttt{val} zurück gibt. \\ Beispiel: \texttt{indexOf("asdfgas", 's')} gibt \texttt{1} zurück.
		\item Ist das Zeichen nicht in dem String vorhanden, stellt dies einen Fehler dar.
		\item Mit Fehlermeldung über Result Codes könnten wir nun beispielsweise \texttt{-1} zurück geben, da dies kein valider Index ist (welche immer \texttt{\(\geq\) 0} sein müssen).
		\item Damit sieht der Aufrufer, dass der Methodenaufruf schief gegangen ist und kann entsprechend reagieren.
	\end{itemize}
	
	\paragraph{Vorteile}
		\begin{itemize}
			\item Es werden keine expliziten Verfahren zum Melden von Fehlern benötigt.
			\item Die genutzt Technologie wird in vielen Sprachen eingesetzt.
		\end{itemize}
	% end
	
	\paragraph{Nachteile}
		\begin{itemize}
			\item Manchmal ist es nicht möglich, Fehler so anzuzeigen (beispielsweise wenn alle Werte gültig sind).
			\item Der Aufrufer muss extra prüfen und daran denken, ob und welche Codes zurück kommen könnten.
		\end{itemize}
	% end
% end

\subsection{Exceptions}
	\introduces{von Exceptions}{exceptions}

	Schauen wir uns nun \textit{Exception} an, ein sehr viel mächtigeres System als Result Codes.
	
	Die Grundidee einer Exception ist, dass die Ausführung des Codes an einer beliebigen Stelle abgebrochen wird und dem Aufrufer über einen weiteren Mechanismus (den Exceptions) aufgezeigt wird, dass es Fehler vorlag. Der Aufrufer kann den Fehler anschließend behandeln.
	
	Das System besteht aus den folgenden Teilen, welche wir in den nächsten Abschnitten näher betrachten:
	\begin{itemize}
		\item Werfen von Exceptions und
		\item Fangen von Exceptions.
	\end{itemize}
	
	\subsubsection{Werfen von Exceptions}
		Eine Methode, welche beispielsweise die Berechnung \(\frac{a}{b}\) durchführt, muss einen Fehler auslösen, wenn \(b = 0\) gilt.
		
		Im Kontext von Exceptions wird dieses Auslösen eines Fehler \textit{werfen} einer Exception genannt, das heißt der Code bricht ab, es wird kein Wert zurück gegeben und der Aufrufer \enquote{erhält} den Fehler, welcher eine genauere Beschreibung enthalten kann (beispielsweise die Nachricht \enquote{MathException: Cannot divide by 0.}).
		
		\paragraph{Beispiel}
			\begin{figure}[H]
				\centering
				\begin{lstlisting}[language = Java, style = base]
divide(int a, int b) {
	if (b == 0) {
		// Nach der folgenden Zeile wird zum Aufrufer zurueck gekehrt, dieser
		// erhaelt die Nachricht und die Ausfuehrung der Methode wird
		// abgebrochen.
		throw "MathException: Cannot divide by 0." // Werfen der Exception.
	}

	// Somit koennen wir uns nun sicher sein, dass 'b != 0' gilt und einfach mit
	// der Division forfahren.
	return /* Divisions-Algorithmus */
}
				\end{lstlisting}
				\caption{Exceptions Werfen: Beispiel}
				\label{fig:throw_exceptions}
			\end{figure}
		% end
	% end
	
	\subsubsection{Fangen von Exceptions}
		Rufen wir eine Methode auf, welche eine Exception werfen kann (dies wird je nach Sprache in der Signatur der Methode dokumentiert), so müssen wir diese \textit{fangen}. Das bedeutet, wir müssen die Exception empfangen und den Fehler behandeln (wie auch immer).
		
		Dies geschieht zumeist mit einem Try-Catch-Konstrukt, welcher in zwei Blöcke aufgeteilt ist:
		\begin{itemize}
			\item Der \textit{Try-Block} enthält den Code, der die Exception auslösen kann. Tritt irgendwo eine Exception auf, so bricht die Ausführung dieses Blocks ab.
			\item Der \textit{Catch-Block} fängt eine mögliche Exception und wird nur ausgeführt, wenn im Try-Block ein Fehler aufgetreten ist. Sofern der Catch-Block nicht für einen Abbruch der Ausführung sorgt, wird nach seiner Ausführung einfach mit der ersten Zeile nach dem Try-Catch fortgefahren.
			\item In den meisten Sprachen gibt es noch einen Finally-Block, diesen werden wir aber erst im Java-Abschnitt zu Exceptions behandeln.
		\end{itemize}
		
		\paragraph{Beispiel}
			Sei wieder die Methode aus Abbildung \ref{fig:throw_exceptions} gegeben, nur rufen wir diese diesmal auf.
			\begin{figure}[H]
				\centering
				\begin{lstlisting}[language = Java, style = base]
// Try-Catch-Konstrukt
try { // Try-Block

	// Dieser Aufruf geht noch gut, denn '2 != 0'.
	divide(4, -2)
	// Dieser Aufruf wird fehlschlagen und der Catch-Block wird ausgefuehrt.
	divide(5, 0)
	// Dieser Aufruf wird nicht mehr ausgefuehrt, da der vorherige Aufruf
	// fehlgeschlagen ist.
	divide(6, 2)
} catch (String exception) { // Catch-Block

	// Der String 'exception' enthaelt nun den Wert "MathException: Cannot divide
	// by 0.", welcher von der Methode divide(int, int) als Fehlermeldung
	// uebergeben wurde.
	// Wir geben den Fehler hier einfach aus und behandeln ihn nicht weiter.
	print(exception)
}
				\end{lstlisting}
				\caption{Exceptions Fangen: Beispiel}
			\end{figure}
		% end
	% end
	
	\subsubsection{Exception-Typen}
		Es wird im allgemeinen zwischen den folgenden Exception-Typen unterschieden:
		\begin{description}
			\item[Geprüft] Diese Exceptions müssen von dem Aufrufer gefangen und behandelt oder weitergeleitet werden. Das Ignorieren der Exception führt zu einem Compiler Fehler.
			\item[Nicht Geprüft] Diese Exceptions müssen nicht von dem Aufrufer gefangen werden und können ignoriert werden. Allerdings stürzt das Programm ab, sollte doch ein solcher Fehler auftreten.
		\end{description}
		
		\info{Die Diskussion, ob man nur geprüfte Exceptions, nur ungeprüfte Exceptions oder beides verwenden sollte, ist sehr langwierig und es gibt für beide Seiten gute Argumente. Meiner Meinung nach ist die Mischform der beste Weg, da dieser am meisten Flexibilität bietet.}
	% end
% end

% end

\section{Dokumentation mit JavaDoc}
	\todo{Schreiben}

% Klassen
% Methoden
% Konstruktoren
% etc.

% end

    %% end
    
    \chapter{Java}
    	\label{c:java}
    	
    	Dieses Kapitel wird in den nächsten Wochen folgen.
    % end
    
    %\chapter{Abstraktion}
	%    \label{c:abstraktion}
    %
	%    \todo{Schreiben}

\section{Funktionale Abstraktion}
	\todo{Schreiben}
% end

\section{Objektorientierte Abstraktion}
	% Ein Konstruktor ist eine besondere Methode innerhalb einer Klasse, welche keinen expliziten Rückgabetyp besitzt und den gleichen Namen trägt wie die Klasse. In dieser Methode werden alle zur Initialisierung der Klasse nötigen Aktionen vollführt, zum Beispiel Objektvariablen belegen. Jede Klasse besitzt einen Default-Konstruktor, welcher keine Parameter annimmt

\subsection{Klassen, Objekte und Methoden}
	\todo{Schreiben}
	
	\subsubsection{Statische Methoden und Attribute}
		\todo{Schreiben}
	% end
	
	\subsubsection{Sichtbarkeit}
		\todo{Schreiben}
	% end
	
	\subsubsection{Abgrenzung: Objektvariable \(\leftrightarrow\) Objektkonstante \(\leftrightarrow\) Klassenvariable \(\leftrightarrow\) Klassenkonstante}
		\todo{Schreiben}
	% end
	
	\subsubsection{Abgrenzung: Objektmethode \(\leftrightarrow\) Klassenmethode}
		\todo{Schreiben}
	% end
	
	\subsubsection{Konstruktoren}
		\todo{Schreiben}
		
		\paragraph{Überladen von Konstruktoren}
			\todo{Schreiben}
		% end
		
		\paragraph{\texttt{this}}
			\todo{Schreiben}
		% end
		
		\paragraph{\texttt{super}}
			\todo{Schreiben}
		% end
	% end
	
	\subsubsection{Initializer-Block}
		\todo{Schreiben}
	% end
	
	\subsubsection{Static-Initializer-Block}
		\todo{Schreiben}
	% end
% end

\subsection{Referenzen}
	\todo{Schreiben}
	
	\subsubsection{Vergleich zu primitiven Daten}
		\todo{Schreiben}
	% end
	
	\subsubsection{Literal \texttt{null}}
		\todo{Schreiben}
	% end
	
	\subsubsection{Sonderfall \texttt{String}}
		\todo{Schreiben}
	% end
	
	\subsubsection{Zuweisen vs. Kopieren}
		\todo{Schreiben}
	% end
	
	\subsubsection{Test auf Gleichheit und Identität}
		\label{sec:equals_identity}
	
		\todo{Schreiben}
	% end
	
	\subsubsection{Downcasts}
		\todo{Schreiben}
	% end
% end

\subsection{Vererbung}
	\todo{Schreiben}
	
	\subsubsection{Methoden}
		\todo{Schreiben}
	% end
	
	\subsubsection{Variation der Sichtbarkeit}
		\todo{Schreiben}
	% end
	
	\subsubsection{Variation von Rückgabetyp und Exceptions}
		\todo{Schreiben}
	% end
	
	\subsubsection{Attribute}
		\todo{Schreiben}
	% end
	
	\subsubsection{Finale Klassen}
		\todo{Schreiben}
	% end
% end

\subsection{Abstrakte Klassen}
	\todo{Schreiben}
% end

\subsection{Interfaces}
	\todo{Schreiben}
	
	\subsubsection{Default-Methoden}
		\todo{Schreiben}
	% end
	
	\subsubsection{Funktionale Interfaces}
		\todo{Schreiben}
		
		\paragraph{Interfaces in \texttt{java.util.function}}
			\todo{Schreiben}
		% end
	% end
% end

\subsection{Polymorphie und späte Bindung}
	\todo{Schreiben}
	
	% statischer/dynamischer Typ
% end

\subsection{Verschachtelte Klassen}
	\todo{Schreiben}
	
	\subsubsection{Statische verschachtelte Klassen}
		\todo{Schreiben}
	% end
	
	\subsubsection{Innere Klassen}
		\todo{Schreiben}
	% end
	
	\subsubsection{Anonyme Innere Klassen}
		\todo{Schreiben}
	% end
% end

\subsection{Lambda-Ausdrücke}
	\todo{Schreiben}
	
	\subsubsection{Methoden-Referenzen}
		\todo{Schreiben}
	% end
% end

\subsection{Enumerations}
	\todo{Schreiben}
	
	\subsubsection{Klasse \texttt{java.lang.Enum}}
		\todo{Schreiben}
	% end
	
	\subsubsection{Vererbung}
		\todo{Schreiben}
	% end
% end

\subsection{Metadaten}
	\todo{Schreiben}
	
	% Zur Klasse
	% Zu den Attributen
	% Zu den Methoden
	% Methodentabelle
% end

\subsection{Speicherverwaltung}
	\todo{Schreiben}
% end

% end
    %% end
    
    \chapter{Abstraktion}
    	\label{c:abstraktion}
    	
    	Dieses Kapitel wird in den nächsten Wochen folgen.
    % end
    
    %\chapter{Komplexität und Landau-Symbolik}
	%    \label{c:komplexitaet}
    %
	%    Wir haben in den vorhergehenden Kapiteln viel über Programmiersprachen und deren Eigenarten gehört und widmen uns nun einem etwas theoretischeren Thema als die Programmierung an sich: Der Komplexitätstheorie, genauer der Landau-Symbolik.

Betrachten wir hier zum Einstieg folgende Algorithmen, die zum einen die Fakultät einer Ganzzahl berechnen und zum anderen die Fakultät-Fakultät einer Ganzzahl berechnen (d.h. das Produkt der Fakultäten)
\begin{figure}[H]
	\centering
	\caption{Komplexität: Berechnung der Fakultät}
\end{figure}

\todo{Stopped here.}
    %% end

    %\chapter{Glossar}
	%    \label{c:glossar}
    %
    %    \todo{Skript zur Generierung des Glossars.}
    %% end
    
    \bibliography{../cite}
\end{document}
